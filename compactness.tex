\chapter{Compactness and finiteness}

\section{Compactness theorem}

\section{Convergence of compact spaces.}

\begin{thm}{Definition}
Let $\spc{X}$ and $\spc{Y}$ be metric space,
a map $f\:\spc{X}\to\spc{Y}$
is called \emph{$\eps$-isometry}\index{isometry!$\eps$-siometry}
if the following two condition hold:
\begin{subthm}{}
$\Im f$ is an $\eps$-net of $\spc{Y}$
\end{subthm}

\begin{subthm}{}
$\dist{f(x)}{f(x')}{\spc{Y}}\lege\dist{x}{x'}{\spc{X}}\pm\eps$ for any $x,x'\in\spc{X}$
\end{subthm}

\end{thm}

\begin{thm}{Definition}
Let $\spc{X}_n$ be a sequence of compact metric spaces and $\spc{X}_\infty$ is an other compact metric space.
We say that $\spc{X}_n$ converges to $\spc{X}_\infty$ in the sence of Gromov--Hausdorff if there is a sequence of $\eps_n$-isometries 
$f_n\:\spc{X}_\infty\to \spc{X}_n$ for a sequence of positive numbers $\eps_n$ converging to $0$ as $n\to\infty$.
\end{thm}


The following theorem roughly states that isometry class of Gromov--Hausdorff limit is uniquely defined once it is compact. 

\begin{thm}{Theorem} Let $\spc{X}_1,\spc{X}_2,\dots$ and $\spc{X}_\infty$ and $\bar{\spc{X}}_\infty$ be metric spaces
such that $\spc{X}_n\xto{\GH}\spc{X}_\infty$, 
$\spc{X}_n\xto{\bar\GH}\bar{\spc{X}}_\infty$.

Assume that $\bar{\spc{X}}_\infty$ is compact.
Then $\spc{X}_\infty\iso \bar{\spc{X}}_\infty$.
\end{thm}


\parit{Proof.}
For each $x_\infty\in\spc{X}_\infty$,
fix its liftings $x_n\in\spc{X}_n$.

Choose some ultrafilter $\o$.
Define $\bar x_\infty\in \bar{\spc{X}}_\infty$ such that $\bar x_n\xGHto{\bar\GH}\bar x_\infty$;
since $\bar{\spc{X}}_\infty$ is compact, the $\o$-limit is defined.
We claim that the map $x_\infty\to \bar x_\infty$ is an isometry.

Indeed, according to the definition of Gromov--Hausdorff convergence, 
\[\dist{\bar x_\infty}{\bar y_\infty}{\bar{\spc{X}}_\infty}
=
\lim_{n\to\o}\dist{x_n}{y_n}{\spc{X}_n}
=
\dist{x_\infty}{y_\infty}{\spc{X}_\infty}.
\]
Thus, the map $x_\infty\to\bar x_\infty$ gives a distance preserving map
$\map\:\spc{X}_\infty\hookrightarrow\bar{\spc{X}}_\infty$.
In particular,  
$\spc{X}_\infty$ is compact.
Switching $\spc{X}_\infty$ and $\bar{\spc{X}}_\infty$ and applying the same argument, 
we get an isometiric embedding 
$\bar{\spc{X}}_\infty\hookrightarrow\spc{X}_\infty$.
Thus, the result follows from Lemma~\ref{lem:>=-isometry}.
\qeds




\begin{thm}{Definition}
 Given two compact spaces $\spc{X}$ and $\spc{Y}$, we will write 
\begin{itemize}
\item $\spc{X}\le \spc{Y}$ if there is a non-contracting map $\map\:\spc{X}\to \spc{Y}$.
\item $\spc{X}\le \spc{Y}+\eps$ if there is a map $\map\:\spc{X}\to \spc{Y}$ such that for any $x,x'\in \spc{X}$ we have
\[\dist{x}{x'}{}\le \dist{\map(x)}{\map(x')}{}+\eps.\]
\end{itemize}

Further, we define \index{Gromov--Hausdorff distance}\emph{Gromov--Hausdorff distance} between $\spc{X}$ and $\spc{Y}$, $\GHdist(\spc{X},\spc{Y})$ as infimum of all $\eps>0$ such that
$\spc{X}\le \spc{Y}+\eps$ and $\spc{Y}\le \spc{X}+\eps$.
\end{thm}

\parbf{Remark for AKP.}
This notation is a bit slang-like, maybe better to use $\succcurlyeq$ instead of $\ge$.
Yet $\spc{Y}+\eps$ migh be needed to be changed;
one might think that $\spc{Y}+\eps$ is a metric space $(\ushort{\spc{Y}},\dist{}{}{}+\eps \rho)$, where $\rho$ is a discreate metric on $\ushort{\spc{Y}}$ (i.e. $\rho(x,y)=1$ if $x\not=y$). 
It will be totally correct to introduce some operator say $\op{Add}_\eps\spc{Y}$, but it is also less intuitive...
One may also write something like 
$\spc{X}\succcurlyeq \spc{Y}^{+\eps}$ or $\spc{X}\succcurlyeq^{\eps+} \spc{Y}$...

\medskip

The Gromov--Hausdorff distance turns the set of all isometry classes of compact metric spaces into a metric space.
The following theorem shows that convergence in this space coinsides with the Gromov--Hausdorff convergence defined above.

\begin{thm}{Theorem} Let $\spc{X}_1,\spc{X}_2,\dots$ and $\spc{X}_\infty$ be compact metric spaces.
Then 
\[\spc{X}_n\xto{}\spc{X}_\infty
\ \ \Leftrightarrow\ \ 
\GHdist(\spc{X}_n,\spc{X}_\infty)\to 0.\]

\end{thm}

\parit{Proof; $(\Leftarrow)$.}
Let $\spc{X}_n\xto{\GH} \spc{X}_\infty$
and $b_n\:\spc{X}_n\to \spc{X}_\infty$ be a sequence of maps such that
\[\dist{a_n(x)}{a_n(y)}{\spc{X}_\infty}
\ge
\dist{x}{y}{\spc{X}_n}-\eps_n,\]
\[\dist{b_n(v)}{b_n(w)}{\spc{X}_n}
\ge
\dist{v}{w}{\spc{X}_\infty}-\eps_n\]
for any $x,y\in \spc{X}_n$ and $v,w\in\spc{X}_\infty$ and some sequence $\eps_n\to0+$.

Let us show that $\spc{X}_n\xto{\GH}\spc{X}_\infty$. 

Fix $\eps>0$ and choose a maximal $\eps$-packing $\{x^1,x^2,\dots,x^\kay\}$ in $\spc{X}_\infty$ such that the value $\sum_{i<j}\dist{x^i}{x^j}{}$ is maximal.
Note that 
\[\dist{a_n\circ b_n(x^i)}{a_n\circ b_n(x^j)}{}>\dist{x^i}{x^j}{}-2\cdot\eps_n.\]
Thus, 
$\dist{a_n\circ b_n(x^i)}{a_n\circ b_n(x^j)}{}-\dist{x^i}{x^j}{}
\to_{n\to\infty} 
0$ for all $i$ and $j$.
For all large $n$,
we have $2\cdot\eps_n<\dist{x^i}{x^j}{}-\eps$.
Thus $\dist{b_n(x^i)}{b_n(x^j)}{\spc{X}_n}>\eps$ 
and \[\dist{a_n\circ b_n(x^i)}{a_n\circ b_n(x^j)}{\spc{X}_n}>\eps\] 
for all $i\not=j$.
Thus, for each large $n$, 
the set $\{a_n\circ b_n(x^i)\}$ forms a maximal $\eps$-packing and therefore an $\eps$-net in $\spc{X}_\infty$.
Since $\eps>0$ is arbitrary, we get \ref{def:GH:appr1}.

Since $\{a_n\circ b_n(x^i)\}$ is an  $\eps$-net in $\spc{X}_\infty$,
for any $y_n\in\spc{X}_n$, there is $x^i$ such that $\dist{a_n\circ b_n(x^i)}{a_n(y_n)}{}<\eps$.
Thus, $\dist{b_n(x^i)}{y_n}{}<\eps+\eps_n$.

Further, note that the set $\{b_n(x^i)\}$ forms an $(\eps+\eps_n)$-net in $\spc{X}_n$.
Otherwise, if for some $y_n\in \spc{X}_n$, we have $\dist{y_n}{b_n(x^i)}{}$ for all $i$,
then $\dist{a_n\circ b_n(x^i)}{a_n(y_n)}{}>\eps$; 
i.e. $\kay<\pack_\eps{X}_\infty$, a contradiction.

\parit{$(\Rightarrow)$.}
Assume $\spc{X}_n\xto{\GH}\spc{X}_\infty$.
Fix $\eps>0$ and choose a maximal $\eps$-packing $\{x^1,x^2,\dots,x^\kay\}$ in $\spc{X}_\infty$.
For each $x^i$, 
choose a sequence $x^i_n\in\spc{X}_n$ such that $a_n(x^i_n)\to x^i$.
Note that for all large $n$, we have $\dist{x^i_n}{x^j_n}{}>\eps$.
For each point $z\in \spc{X}_\infty$, choose $x^i$ so that $\dist{z}{x^i}{}<\eps$ and define map $b_n\:\spc{X}_\infty\to\spc{X}_n$ such that 
$b_n(z)=x^i_n$.
\qeds



\begin{thm}{Lemma}\label{lem:>=-isometry}
Let $\spc{X}$ and $\spc{Y}$ be two metric spaces 
and $\spc{X}$ is compact then
\[
\spc{X}\ge\spc{Y}\ge\spc{X}
\ \ \Leftrightarrow\ \ 
\spc{X}\iso\spc{Y}.
\]

\end{thm}

The following proof was suggested by Travis Morrison.

\parit{Proof.}
Let $f\: \spc{X} \to \spc{Y}$ 
and $g\: \spc{Y} \to \spc{X}$ be non contracting mappings.
It is sufficient to prove that $h  = g\circ f\:\spc{X}\to \spc{X}$ is an isometry. 

Given any pair of points $x,y\in \spc{X}$, 
set $x_n=h^{\circ n}(x)$ and $y_n=h^{\circ n}(y)$.
Since $\spc{X}$ is compact, one can choose an incresing sequence of integers $n_\kay$
such that both sequences $(x_{n_i})_{i=1}^\infty$ and $(y_{n_i})_{i=1}^\infty$
converge.
In particular, both of these sequences  are Cauchy;
i.e.,
\[
\dist{x_{n_i}}{x_{n_j}}{},\dist{y_{n_i}}{y_{n_j}}{}\to 0
\]
as $\min\{i,j\}\to\infty$.
Since $h$ is noncontracting, we get
\[
\dist{x}{x_{|n_i-n_j|}}{}\le \dist{x_{n_i}}{x_{n_j}}{}.
\]
It follows that  
there is a sequence $m_i\to\infty$ such that
\[
x_{m_i}\to x\ \ \text{and}\ \ y_{m_i}\to y\ \ \text{as}\ \ \kay\to\infty.
\eqlbl{eq:x_l->x}
\]

Let $\ell_n=\dist{x_n}{y_n}{}$.
Since $h$ is noncontracting, $(\ell_n)$ is a nondecreasing sequence.
On the other hand, 
from \ref{eq:x_l->x}, it follows that $\ell_{m_i}\to\dist{x}{y}{}=\ell_0$ as $m_i\to\infty$;
i.e., $(\ell_n)$ is a constant sequece.
In particular $\ell_0=\ell_1$ for any $x$ and $y\in \spc{X}$;
i.e., $h$ is distance preserving map.

Therefore $h(X)$ is isometric to $\spc{X}$.
From \ref{eq:x_l->x}, we get that $h(\spc{X})$ is everywhere dense.
Since $X$ is compact, we get $h(\spc{X})=\spc{X}$.
\qeds

\section{Strongly convex functions}

\section{A finiteness theorem}

\section{Controlled concavity}
\addtocounter{subsection}{1}
\setcounter{thm}{0}

In this and the next sections we introduce a couple of techniques which use
comparison of $m$-dimensional Alexandrov's space with a model space of the same
dimension \label{lob-k-m}$\Lob{m}\kappa$ (that is $m$-dimensional simply connected Riemannian manifold with constant curvature $\kappa$).
These techniques were introduced by Grigori Perelman in \cite{perelman:morse}
and \cite{perelman:DC}.

We start with the local existence of a strictly concave function on an Alexandrov's space.

\begin{thm}{Theorem~\cite[3.6]{perelman:morse}}
\label{thm:strictly-concave}
Let $A\in\CBB{}{}$. 

For any point $p\in A$ there is a strictly concave function $f$ defined in an
open neighborhood of $p$.

Moreover, given $v\in T_p$, the differential, $d_p f(x)$, can be chosen
arbitrarily close to $x\mapsto -\<v,x\>$
\end{thm}

\begin{wrapfigure}{r}{35mm}
\begin{lpic}[t(-10mm),b(0mm),r(0mm),l(0mm)]{pics/strictly-concave(0.25)}
\lbl[tr]{125,05;$q$}
\lbl[br]{32,166;$\gamma(t)$}
\lbl[l]{44,159;$\alpha(t)$}
\end{lpic}
\end{wrapfigure}

\parit{Proof.} 
Consider the real-to-real function 
$$\phi_{r,c}(x)=(x-r)- c{(x-r)^2}/r,$$
so we have 
$$\phi_{r,c}(r)=0,\ \ \phi_{r,c}'(r)=1\ \ \phi_{r,c}''(r)=- {2c}/{r}.$$ 

Let $\gamma$ be a unit-speed geodesic, fix a point $q$ and set 
$$\alpha(t)=\angle(\gamma^+(t),\dir{\gamma(t)}{q}).$$
If $r>0$ is sufficiently small and $\dist{q}{\gamma(t)}{}$ is sufficiently close to
$r$, then direct calculations show that
$$(\phi_{r,c}\circ\dist{q}{}{}\circ\gamma)''(t)
\le 
\frac{3-c\cdot \cos^2\alpha(t)}{r}.$$

Now, assume $\{q_i\}$, $i=\{1,..,N\}$ is a finite set of points such that $\dist{p}{q_i}{}=r$ for any $i$. 
For $x\in A$ and $\xi_x\in \Sigma_x$, set $\alpha_i(\xi_x)=\angle(\xi_x,\dir{p}{q_i})$. 
Assume we have a collection $\{q_i\}$ such
that for any $x\in B_\eps(p)$ and $\xi_x\in \Sigma_x$ 
we have  $\max_i\{\dist{\alpha_i(\xi_x)}{\tfrac\pi2}{}\}\ge\eps>0$. 
Then  taking in the above inequality $c>3N/\cos^2\eps$, we get that the function
$$f=\sum_i \phi_{r,c}\circ\dist{q_i}{}{}$$
is strictly concave in $B_{\eps'}(p)$ for some positive $\eps'<\eps$.

To construct the needed collection $\{q_i\}$, note that for small $r>0$ one can
construct $N_\delta\ge \Const/\delta^{(m-1)}$ points $\{q_i\}$ such that $\dist{p}{q_i}{}=r$
and $\tilde\angle_\kappa q_i p q_j>\delta$ (here $\Const=\Const(\Sigma_p)>0$).
On the other hand, the set of directions which is orthogonal to a given direction
is smaller than $S^{m-2}$ and therefore contains at most
$\Const(m)/\delta^{(m-2)}$ directions with angles at least $\delta$. 
Therefore, for small enough $\delta>0$, $\{q_i\}$ forms the needed collection.

If $r$ is small enough, points $q_i$ can be chosen so that all directions
$\dir{p}{q_i}$ will be $\eps$-close to a given direction $\xi$ and
therefore the second property follows.
\qeds

Note that in the theorem~\ref{thm:strictly-concave} (as well as in
theorem~\ref{exist-control}), the function $f$ can be chosen to have maximum value $0$ at $p$,
$f(p)=0$ and with $d_p f(x)$ arbitrary close to $-|x|$.
It can be constructed by taking the minimum of the functions in these theorems. Whence the claim below follows.


\begin{thm}{Claim}\label{cor:convex-nbhd}
For any point of an Alexandrov's space there is an arbitrary small closed convex
neighborhood.
\end{thm}

Applying rescaling and passing to the limit, 
one can estimate the size of the convex
hull in an Alexandrov's space in terms of the volume of a ball containing it:

\begin{thm}{Lemma on strictly concave convex hulls
\cite[4.3]{perelman-petrunin:extremal}}
For any $v>0$, $r>0$ and $\kappa\in \RR$, $m\in\NN$ there is $\eps>0$ such that,
if $A\in\CBB{m}\kappa$ and $\vol B_{r}(p)\ge v$ then for any $\rho<\eps\cdot r$, 
$$\diam[\Conv B_\rho(p)]\le \rho/\eps.$$

In particular, for any compact Alexandrov's $A$ space there is $\Const\in \RR$ such that for
any subset $X\subset A$
$$\diam \left(\Conv X \right)\le \Const\cdot\diam X.$$
\end{thm}





\subsection{General definition.} 
The above construction can be generalized and optimized in many ways to fit
particular needs. 
Here we introduce one such variation which is not the most general, but general
enough to work in most applications.

Let $A$ be an Alexandrov's space and $f\:A\to \RR$,
$$f=\Theta(\dist[2]{q_1}{}{},\dist[2]{q_2}{}{},\dots,\dist[2]{q_N}{}{})$$ be a \index{simple
function}\emph{simple
function} (see section~\ref{adm}).
If $A$ is $m$-dimensional, we say that such a function $f$ has \index{controlled
concavity}\emph{controlled
concavity of type} $(\lambda,\kappa)$ at $p\in A$, if for any $\eps>0$ there is
$\delta>0$, such that for any collection of points $\{\tilde p,\tilde q_i\}$ in
the \index{model space}\emph{model $m$-space}%
\footnote{that is, a simply connected $m$-manifold with
constant curvature $\kappa$.} $\Lob{m}\kappa$  satisfying 
\[\dist{\tilde q_i}{\tilde q_j}{}
>
\dist{q_i}{q_j}{}-\delta
\ \ \text{and}\ \  
\bigl|\dist{\tilde
p}{\tilde q_i}{}-\dist{p}{q_i}{}\bigr|<\delta
\ \ \text{for all}\ \ i,j,\] 
we have that the function 
$\tilde f\:\Lob{m}\kappa\to \RR$ defined by 
$$\tilde f=\Theta(\dist[2]{\tilde q_1}{}{},\dist[2]{\tilde q_2}{}{},..,\dist[2]{\tilde
q_n}{}{})$$
is $(\lambda-\eps)$-concave in a small neighborhood of $\tilde p$.

The following lemma states that the conrolled concavity is stronger than the usual
concavity.

\begin{thm}{Lemma} \label{contr-concave}
Let $A\in\CBB{m}\kappa$.

If a simple function 
$$f=\Theta(\dist[2]{q_1}{}{},\dist[2]{q_2}{}{},..,\dist[2]{q_N}{}{}),\ \ f\:A\to\RR$$ 
has a conrolled concavity type $(\lambda,\kappa)$ at each point $p\in \Omega$, then
$f''\le \lambda$ in $\Omega$. 
\end{thm}

The proof is just a direct calculation similar to that in the proof
of~\ref{thm:strictly-concave}. 
Note also, that the function constructed in the proof of
theorem~\ref{thm:strictly-concave} has controlled concavity.
In fact from the
same proof follows:

\begin{thm}{Existence} \label{exist-control}
Let $A\in\CBB{m}{}$, $p\in A$, $\lambda,\kappa\in\RR$.
Then there is a function $f$ of controlled concavity $(\lambda,\kappa)$
at $p$.

Moreover, given $v\in T_p$, the function $f$ can be chosen so that its differential $d_p f(x)$ will be arbitrary close to $x\mapsto -\<v,x\>$.
\end{thm}

Since functions with a conrolled concavity are simple they admit liftings, and
from the definition it is clear that these liftings also have controlled
concavity of the same type.
More precisely, we get the following.

\begin{thm}{Concavity of lifting} \label{lem:lifting}
Let $A\in\CBB{m}{}$.

Assume a simple function 
$$f\:A\to \RR,\ \ f=\Theta(\dist[2]{q_1}{}{},\dist[2]{q_2}{}{},..,\dist[2]{q_N}{}{})$$
has controlled concavity type $(\lambda,\kappa)$ at $p$.

Let $A_n\in\CBB{m}\kappa$,
$A_n\GHto A$ (so, no collapse) and $\{p_n\},\{q_{i,n}\}\in
A_n$ be sequences of points such that $p_n\to p\in A$ and  $q_{i,n}\to q_i\in A$
for each $i$.

Then for all large $n$, the liftings of $f$,
$$f_n\:A_n\to \RR,\ \
f_n=\Theta(\dist[2]{q_{1,n}}{}{},\dist[2]{q_{2,n}}{}{},..,\dist[2]{q_{N,n}}{}{})$$
have controlled concavity type $(\lambda,\kappa)$ at $p_n$. 
\end{thm}

\section{Homotopy stability}

\begin{thm}{Proposition}\label{prop:homotopy-eq}
Let $\spc{L}_n$ be a sequence of $m$-dimensional $\CBB{}{\kappa}$ spaces.
Assume that as $n\to\infty$, the sequence $\spc{L}_n$ converges to a compact  $m$-dimensional $\CBB{}{\kappa}$ space $\spc{L}_\infty$.
Then $\spc{L}_n$ is homotopically equivalent to $\spc{L}_\infty$ for all large $n$.
\end{thm}

The proposition above together with compactness theorem
implies the following.

\begin{thm}{Corollary}
Given a positive integer $m$, a  positive reals $D$ and $v$ and there finitely many homotopy types of compact $m$-dimensional manifolds with diameter at most $D$, volume at least $v$ and sectional curvature at least $-1$.
\end{thm}

\parit{Proof.}
Assume contrary,
that is there is a sequence of Riemannian manifolds $\spc{M}_n$ with pairwise distinct homotopy types and the curvature, diameter, volume bounds as int the formulation of corollary.

By compactness theorem, 
we can pass to a converging subsequence of $\spc{M}_n$.
Denote by $\spc{M}_\infty$ its limit.

By Proposition~\ref{prop:homotopy-eq}, $\spc{M}_n$ is homotopically equivalent to $\spc{M}_\infty$ for all large $n$,
a contradiction.
\qeds


\parit{Proof of Proposition~\ref{prop:homotopy-eq}.}
According to ??? for any point $p\in\spc{L}_\infty$ 
one can choose a strictly concave function $f_p$ 
which is defined in a neighborhood $\Omega\ni p$ 
and has a maximum at $p$.

We can assume that the set 
\[K_p=\set{x\in\Omega}{f_p(x)\ge0}\] 
forms s compact set; if not add an appropriate real number to $f_p$.

Since $\spc{L}_\infty$ is compact, there is an array of points $p_1,\dots,p_n$
such that the interiors of $K_{p_1},\dots,K_{p_n}$ cover $\spc{L}_\infty$.

According to ???, each $f_p$ is a limit of strictly concave functions defined on $\spc{L}_n$.
???
\qeds