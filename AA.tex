\begin{thm}{Exercise}\label{ex:short-retracts}
Let $\spc{U}\in \Cat{}{0}$
and $\phi^1,\phi^2,\dots,\phi^k\:\spc{U}\to \spc{U}$ be commuting short retractions; 
that is 
\begin{itemize}
\item $\phi^i\circ\phi^i=\phi^i$ for each $i$;
\item $\phi^i\circ\phi^j=\phi^j\circ\phi^i$ for any $i$ and $j$;
\item $\dist{\phi^i(x)}{\phi^i(y)}{\spc{U}}\le \dist{x}{y}{\spc{U}}$ for each $i$ and any $x,y\in\spc{U}$.
\end{itemize}
Set $A^i=\Im \phi^i$ for all $i$.

Assume $\Gamma$ is a finite graph 
(without loops and multiple edges) 
with edges labeled by $1,2,\dots, n$.
Denote by $\spc{U}^\Gamma$ the space obtained by taking 
a copy of $\spc{U}$ for each vertex of $\Gamma$, and 
gluing two such copies along $A^i$ if the corresponding vertices are joined by an edge labeled by $i$.

Show that $\spc{U}^\Gamma$ is a $\Cat{}{0}$ space.
\end{thm}

\parbf{Exercise~\ref{ex:short-retracts}.}
Start with a copy of $\spc{U}$ for each vertex of $\Gamma$
and glue them recursively using the following operations and stages.

At  stage $n$, only glue  spaces along $A^n$.

In one step of the stage, choose two connected components of partially glued copies and glue them along the corresponding $A^n$'s.
Apply Reshetnyak gluing theorem to show that each connected component after the step is $\Cat{}{0}$, assuming that all connected components were $\Cat{}{0}$ before the step.
\qeds

%???remove



Given a metric graph $\Gamma$,  define $P_k(\Gamma)$ as follows. Let $\Gamma^b$ be the barycentric subdivision of $\Gamma$ with the natural metric. For any two adjacent vertices $p,q\in\Gamma^b$, substitute for the edge $[pq]$ a countable collection of intervals $\{I_i\}_{i\ge 1}$ of length $\dist{p}{q}{}+\frac{\dist{p}{q}{}}{2^ki}$ where one end of each $I_i$ is glued to $p$ and the other to $q$. Note that the resulting space $P_k(\Gamma)$ is again a metric graph  with length metric. 

Let $\spc{X}_0=[0,1]$ and define $\spc{X}_k$ for $k\ge 1$ inductively by $\spc{X}_k=P_k(\spc{X}_{k-1})$.

Let $\spc{Y}_k$ be the set of vertices of $\spc{X}_k$ with the induced metric. By construction the inclusion $\spc{Y}_k\subset \spc{Y}_{k+1}$ is distance preserving.

Let $\spc{Y}_\infty=\cup_{k\ge 1}\spc{Y}_k$ with the obvious metric and let $\spc{Y}=\bar {\spc{Y}}_\infty$ be its metric completion. Then $\spc{Y}$ is a length space since it satisfies the almost midpoint property. But it is not hard to see that no two distinct points in $\spc{Y}$ can be connected by a shortest geodesic. \qeds














A unit-speed geodesic $\gamma\:\RR_{\ge0}\to \spc{X}$ is called a \index{ray}\emph{ray}.

A unit-speed geodesic  $\gamma\:\RR\to \spc{X}$ is called a \index{line}\emph{line}.

\begin{thm}{Proposition}\label{prop:busemann}
Suppose $\spc{X}$ is a metric space and $\gamma\:[0,\infty)\to \spc{X}$ is a ray. 
Then the \index{Busemann function}\emph{Busemann function} $\bus_\gamma\:\spc{X}\to \RR$ 
\[\bus_\gamma(x)=\lim_{t\to\infty}\dist{\gamma(t)}{x}{}- t\eqlbl{eq:def:busemann*}\]
is defined
and $1$-Lipschitz.
\end{thm}

\parit{Proof.}
As  follows from the triangle inequality, the function \[t\mapsto\dist{\gamma(t)}{x}{}- t\] is nonincreasing in $t$.  
Clearly $\dist{\gamma(t)}{x}{}- t\ge-\dist{\gamma(0)}{x}{}$.
Thus the limit in \ref{eq:def:busemann*} is defined.
\qeds




















\section{Main definition}

For a quadruple of points $x,y,p,q$ in a space $\spc{X}$,
consider two model triangles 
$[\~p\~x\~y]$ and $[\~q\~x\~y]$ in the plane $\EE^2$ with shared side $[\~x\~y]$.
We say that the quadruple  $x,y,p,q$ satisfies two-plus-two comparison 
if for any for any point $\~z\in [\~x\~y]$ we have
\[\dist{\~p}{\~z}{\EE^2}+\dist{\~q}{\~z}{\EE^2}\ge\dist{p}{q}{\spc{X}}.\]

If in a space $\spc{X}$ the two-plus-two comparison holds for any quadruple,
we say that $\spc{X}$ is $\cCat{}{0}$.

If instead of Euclidean plane we use unit sphere or Lobachevsky plane,
we get a definition of $\cCat{}{1}$ and $\cCat{}{-1}$ spaces correspondingly.
The only difference comes from the fact that the model triangle for given triple of points in a sphere is not always defined; 
if this is the case for one of two triples $x,y,p$ or $x,y,q$, we assume that spherical two-plus-two comparison holds fro the quadruple of points $x,y,p,q$.

\section{Constructions}

Product, Cone, Spherical suspension.