\chapter{Preliminaries}

In this chapter we fix some conventions and remind the main definitions.
For more background in metric geometry, reader advised to read the book of Burgos and Ivanov \cite{BBI}.
 

\section{Metric spaces}
\label{sec:metric spaces}


The distance between two points $x$ and $y$ in a metric space $\spc{X}$ will be denoted by $\dist{x}{y}{}$ or $\dist{x}{y}{\spc{X}}$.
The latter notation is used if we need to emphasize 
that the distance is taken in the space ${\spc{X}}$.

The function $\dist{x}{}{}\:y\mapsto \dist{x}{y}{}$ is called \emph{distance function}\index{distance function} from $x$. 

\begin{itemize}
\item The \emph{diameter}\index{Diameter} of metric space $\spc{X}$ is defined as
\[\diam \spc{X}=\sup\set{\dist{x}{y}{\spc{X}}}{x,y\in \spc{X}}.\]

\item Given $R\in[0,\infty]$ and $x\in \spc{X}$, the sets
\begin{align*}
\oBall(x,R)&=\{y\in \spc{X}\mid \dist{x}{y}{}<R\},
\\
\cBall[x,R]&=\{y\in \spc{X}\mid \dist{x}{y}{}\le R\}
\end{align*}
are called respectively the  \emph{open}\index{open ball} and  the \emph{closed  balls}\index{closed ball}   of radius $R$ with center $x$.
Again, if we need to emphasize that these balls are taken in the metric space $\spc{X}$,
we write 
\[\oBall(x,R)_{\spc{X}}\quad\text{and}\quad\cBall[x,R]_{\spc{X}}\]
correspondingly.
\end{itemize}

A metric space $\spc{X}$ is called \emph{proper}\index{proper space} if all closed bounded sets in $\spc{X}$ are compact. 
This condition is equivalent to each of the following statements:
\begin{enumerate}
\item For some (and therefore any) point $p\in \spc{X}$ and any $R<\infty$, 
the closed ball $\cBall[p,R]\subset\spc{X}$ is compact. 
\item The function $\dist{p}{}{}\:\spc{X}\to\RR$ is proper for some (and therefore any) point $p\in \spc{X}$;
that is, for any compact set $K\subset \RR$, its inverse image 
$\set{x\in \spc{X}}{\dist{p}{x}{\spc{X}}\in K}$
is compact.
\end{enumerate}

\section{Constructions}\label{sec:constructions}

\parbf{Product space.}
Given two metric spaces $\spc{U}$ and $\spc{V}$, the \emph{product space} 
$\spc{U}\times\spc{V}$ is defined as the set of all pairs $(u,v)$ in the set $\spc{U}\times\spc{V}$ 
with the metric defined by formulas
\[\dist{(u^1,v^1)}{(u^2,v^2)}{\spc{U}\times\spc{V}}=\sqrt{\dist[2]{u^1}{u^2}{\spc{U}}+\dist[2]{v^1}{v^2}{\spc{V}}}.\]

Given a metric spaces $\spc{U}$,
we say that $\spc{V}$ is Euclidean cone over $\spc{U}$
if the underlying set of $\spc{V}$ is formed 

\parbf{Cone.}
The \emph{cone} $\spc{V}=\Cone\spc{U}$ over metric space $\spc{U}$
is defined as the metric space with underlying set formed by the equivalence classes on
$[0,\infty)\times \spc{U}$ with the minimal equivalence relation ``$\sim$'' such that $(0,p)\sim (0,q)$ for any points $p,q\in\spc{U}$
and the metric given by cosine rule
\[
\dist{(p,s)}{(q,t)}{\spc{V}} 
=
\sqrt{s^2+t^2-2\cdot s\cdot t\cdot \cos\alpha},
\]
where $\alpha= \max\{\pi, \dist{p}{q}{\spc{U}}\}$.

The point in the cone $\spc{V}$ formed by the equivalence class of $0\times\spc{U}$ is called \emph{tip of the cone} and denoted by $0$ or $0_{\spc{V}}$.
The distance $\dist{0}{v}{\spc{V}}$ is called norm of $v$ and denoted as $|v|$ or $|v|_{\spc{V}}$.

\parbf{Suspension.}
The \emph{suspension} $\spc{V}=\Susp\spc{U}$ over metric space $\spc{U}$
is defined as the metric space with underlying set formed by the equivalence classes on
$[0,\infty]\times \spc{U}$ with the equivalence relation ``$\sim$'' defined by $(0,p)\sim (0,q)$ and $(\pi,p)\sim (\pi,q)$ for any points $p,q\in\spc{U}$
and the metric given by spherical cosine rule
\[
\cos\dist{(p,s)}{(q,t)}{\Cone\spc{U}} 
=
\cos s\cdot\cot t+\sin s\cdot\sin t\cdot\cos\alpha,
\]
where $\alpha= \max\{\pi, \dist{p}{q}{\spc{U}}\}$.

The points in the cone $\spc{V}$ formed by the equivalence class of $0\times\spc{U}$ and $\pi\times\spc{U}$ is called \emph{north} \emph{south pole} of the suspension.

\begin{thm}{Exercise}\label{ex:product-cone}
Let $\spc{U}$ be a metric space.
Show that the spaces 
\[\RR\times \Cone\spc{U}\quad\text{and}\quad\Cone[\Susp\spc{U}]\]
are isometric.
\end{thm}




\section{Geodesics, triangles and hinges}
\label{sec:geods}

\parbf{Geodesics.}
Let $\spc{X}$ be a metric space 
and $\II$\index{$\II$} be a real interval. 
A globally isometric map $\gamma\:\II\to \spc{X}$ is called a \emph{unit-speed geodesic}\index{unit-speed geodesic}%
\footnote{Various authors call it differently: \emph{shortest path}, \emph{minimizing geodesic}.}; 
in other words, $\gamma\:\II\to \spc{X}$ is a unit-speed geodesic if 
\[\dist{\gamma(s)}{\gamma(t)}{\spc{X}}=|s-t|\]
for any pair $s,t\in \II$.

A unit-speed geodesic $\gamma\:\RR_{\ge0}\to \spc{X}$ is called a \emph{ray}\index{ray}.

A unit-speed geodesic  $\gamma\:\RR\to \spc{X}$ is called a \emph{line}\index{line}.

\begin{thm}{Proposition}\label{prop:busemann}
Suppose $\spc{X}$ is a metric space and $\gamma\:[0,\infty)\to \spc{X}$ is a ray. 
Then the \index{Busemann function}\emph{Busemann function} $\bus_\gamma\:\spc{X}\to \RR$ 
\[\bus_\gamma(x)=\lim_{t\to\infty}\dist{\gamma(t)}{x}{}- t\eqlbl{eq:def:busemann*}\]
is defined
and $1$-Lipschitz.
\end{thm}

\parit{Proof.}
As  follows from the triangle inequality, the function \[t\mapsto\dist{\gamma(t)}{x}{}- t\] is nonincreasing in $t$.  
Clearly $\dist{\gamma(t)}{x}{}- t\ge-\dist{\gamma(0)}{x}{}$.
Thus the limit in \ref{eq:def:busemann*} is defined.
\qeds

A unit-speed geodesic between $p$ and $q$ in $\spc{X}$ will be denoted by $\geod_{[p q]}$\index{$\geod_{[{*}{*}]}$}.
We assume $\geod_{[p q]}$ is parametrized starting at $p$; 
that is, $\geod_{[p q]}(0)=p$ and $\geod_{[p q]}(\dist{p}{q}{})=q$.
The image of $\geod_{[p q]}$ will be denoted by $[p q]$\index{$[{*}{*}]$} and called a \emph{geodesic}\index{geodesic}.
The term \emph{geodesic}\index{geodesic} will also be used for  a linear reparametrization of a unit-speed geodesic;
when a confusion is possible we call the latter a \emph{constant-speed geodesic}\index{constantspeed geodesic}.
%???MAYBE BETTER CALL IT GEODESIC CURVE???
%%%%%%%%???MAYBE BETTER TO KEEP CONSTANTSPEED GEODESIC. IT IS CLEAR.  A GEODESIC CURVE POSSIBLY MIGHT HAVE A DIFFERENT PARAMETER. WE SHOULD NOT EVEN RAISE THIS POSSIBILITY IN THE READER'S MIND.
%%%%%%%%%%%S:  SORRY BUT I NEED OTHER PARAMETERS IN WARPED PRODUCTS.  SO CAN WE CALL THESE ``PREGEODESICS'''AS IN BRIDSON-HAEFLIGER.
With slight abuse of notation, we will use $[p q]$ also for the class of all linear reparametrizations of $\geod_{[p q]}$.

We may write $[p q]_{\spc{X}}$ 
to emphasize that the geodesic $[p q]$ is in the space  ${\spc{X}}$.
Also we use the following short-cut notation:
\begin{align*}
\l] p q \r[&=[pq]\backslash\{p,q\},
&
\l] p q \r]&=[pq]\backslash\{p\},
&
\l[ p q \r[&=[pq]\backslash\{q\}.
\end{align*}



In general, a geodesic between $p$ and $q$ need not exist and if it exists, it need not to be unique.  However,  once we write $\geod_{[p q]}$ or $[p q]$ we mean that we made a choice of geodesic.

A metric space is called \emph{geodesic}\index{geodesic} if any pair of its points can be jointed by a geodesic. 

A constant-speed geodesic $\gamma\:[0,1]\to\spc{X}$ is called a \emph{geodesic path}\index{geodesic path}.
Given a geodesic $[p q]$,
we denote by $\geodpath_{[pq]}$ the corresponding geodesic path;
that is,
$$\geodpath_{[pq]}(t)\z\equiv\geod_{[pq]}(t\cdot\dist[{{}}]{p}{q}{}).$$

A curve $\gamma\:\II\to \spc{X}$  is called a \emph{local geodesic}\index{geodesic!local geodesic}, if for any $t\in\II$ there is a neighborhood $U\ni t$ in $\II$ such that the restriction $\gamma|_U$ is a constant-speed geodesic.  If $\II=[0,1]$, then $\gamma$ is called a \emph{local geodesic path}.


\parbf{Triangles.}
For a triple of points $p,q,r\in \spc{X}$, a choice of triple of geodesics $([q r], [r p], [p q])$ will be called a \emph{triangle}\index{triangle} and we will use short notation 
$\trig p q r=([q r], [r p], [p q])$\index{$\trig {{*}}{{*}}{{*}}$}.
Again given a triple $p,q,r\in \spc{X}$ it may be no triangle 
$\trig p q r$ simply because one of the pairs of these points can not be joined by a geodesic, and also it maybe many different triangles with these vertexes, any of which can be denoted by $\trig p q r$.
Once we write $\trig p q r$, it means we made a choice of such a triangle, 
that is, a choice of each $[q r], [r p]$ and $[p q]$.
The value $\dist{p}{q}{}+\dist{q}{r}{}+\dist{r}{p}{}$ will be called \emph{perimeter of triangle} $\trig p q r$.

\parbf{Hinges.}
Let $p,x,y\in \spc{X}$ be a triple of points such that $p$ is distinct from $x$ and $y$.
A pair geodesics $([p x],[p y])$ will be called \emph{hinge}\index{hinge} and briefly, it will be denoted by 
$\hinge p x y=([p x],[p y])$\index{$\hinge{{*}}{{*}}{{*}}$}.


\parbf{Convex sets.}\label{def:convex-set}
Let $\spc{X}$ be a metric space. 
A set $A\subset\spc{X}$ is called 
\emph{convex}%
\index{convex set}
if for every two points $p,q\in A$, 
every geodesic $[pq]$ of $\spc{X}$ 
lies in $A$.

A set $A\subset\spc{X}$ is called 
\emph{locally convex}
if every point $a\in A$ admits an open neighborhood $\Omega\ni a$
such that for every two points $p,q\in A\cap\Omega$ every geodesic $[pq]\subset \Omega$ lies in $A$.


Note that any open set is locally convex by definition.

\section{Length spaces}\label{sec:intrinsic}

Recall that a curve is a continuous map from a real interval to a space.

\begin{thm}{Definition}
Let $\spc{X}$ be a metric space and
$\alpha\: \II\to \spc{X}$ is a curve.
We define the \emph{length}\index{length} of $\alpha$ as 
\[
\length \alpha \df \sup_{t_0\le t_1\le\ldots\le t_n}\sum_i \dist{\alpha(t_i)}{\alpha_{i+1}}{}
\]
\end{thm}
It is easy to see that if $\tau\co [c,d]\to [a,b]$ is monotonic onto function then $\length \alpha=\length(\alpha\circ\tau)$.


Given two points $x$ and $y$ in a metric space $\spc{X}$
consider the value
\[\yetdist{x}{y}{}=\inf_{\alpha}\{\length\alpha\},\]
where infimum is taken for all paths $\alpha$ from $x$ to $y$.

If the value $\yetdist{x}{y}{}$ is finite for any pair of points $x$ and $y$ then $\yetdist{}{}{}$ defines a metric on  $\spc{X}$;
it will be called the induced \emph{length-metric}\index{length-metric} on $\spc{X}$.

In this book, most of the time we consider length spaces.
If $\spc{X}$ is length space, 
and $A\subset \spc{X}$.
The set $A$ comes with the inherited metric from $\spc{X}$ 
which might be not a length-metric.
The corresponding length-metric on $A$ will be denoted as $\dist{}{}{A}$.

\begin{thm}{Definition}
If $\yetdist{x}{y}{}=\dist{x}{y}{}$ for any pair of points $x,y\in\spc{X}$  then $\spc{X}$ is called a \emph{length space}\index{length space}.
\end{thm}
In other words, a metric space $\spc{X}$ is a
\emph{length space}
if for any $\eps>0$ and any two points $x,y\in \spc{X}$  there is a path $\alpha\:[0,1]\to\spc{X}$ connecting%
\footnote{That is, such that $\alpha(0)=x$ and $\alpha(1)=y$.}
 $x$ to $y$
such that 
\[\length\alpha<\dist{x}{y}{}+\eps.\]

Note that any geodesic space is a length space;
as you see from the following example, the contrary does not hold.


\begin{thm}{Example}
Let $\spc{X}$ be obtained by gluing a countable collection of disjoint intervals $I_i$ of length $1+1/i$ where for each $I_i$ one end is glued to $p=\{0\}$ and the other to $q=\{1\}$.
Then $\spc{X}$ carries a natural complete length metric  with respect to which $\dist{p}{q}{}=1$ but there is no geodesic connecting $p$ to $q$.
\end{thm}



\begin{thm}{Exercise}\label{ex:no-geod}
Give an example of a complete length space for which no pair of distinct points can be joined by a geodesic.
\end{thm}


\begin{thm}{Definition}
Let $\spc{X}$ be a metric space and $x,y\in\spc{X}$.

\begin{enumerate}[(i)]
\item A point $z\in \spc{X}$ is called a \emph{midpoint} of $x$ and $y$
if 
\[\dist{x}{z}{}=\dist{y}{z}{}=\tfrac12\cdot\dist[{{}}]{x}{y}{}.\]
\item Assume $\eps\ge 0$.
A point $z\in \spc{X}$ is called \emph{$\eps$-midpoint} of $x$ and $y$
if 
\[\dist{x}{z}{},\dist{y}{z}{}<\tfrac12\cdot\dist[{{}}]{x}{y}{}+\eps.\]
\end{enumerate}

\end{thm}

Note that a $0$-midpoint is the same as a midpoint.


\begin{thm}{Lemma}\label{lem:mid>geod}
Let $\spc{X}$ be a complete metric space.
\begin{subthm}{lem:mid>length}
Assume that for any pair of points $x,y\in \spc{X}$  
 and any $\eps>0$
there is a $\eps$-midpoint $z$.
Then the space $\spc{X}$ is an $R$-length space.
\end{subthm}

\begin{subthm}{lem:mid>geod:geod}
Assume that for any pair of points $x,y\in \spc{X}$ 
such that $\dist{x}{y}{}<R$
there is a midpoint $z$.
Then the space $\spc{X}$ is an $R$-geodesic space.
\end{subthm}
\end{thm}

\parit{Proof.}
Let $x,y\in \spc{X}$ be a pair of points such that $\dist{x}{y}{}<R$.

Set $\eps_n=\frac\eps{2^{2\cdot n}}$.

Set $\alpha(0)=x$ and $\alpha(1)=y$.

Set $\alpha(\tfrac12)$ to be an $\eps_1$-midpoint of $\alpha(0)$ and $\alpha(1)$.
Further, set $\alpha(\frac14)$ 
and $\alpha(\frac34)$ to be $\eps_2$-midpoints 
for the pairs $(\alpha(0),\alpha(\tfrac12)$ 
and $(\alpha(\tfrac12),\alpha(1)$ respectively.
Applying the above procedure recursively,
on the $n$-th step we define $\alpha(\tfrac{\kay}{2^n})$
for every odd integer $\kay$ such that $0<\tfrac\kay{2^n}<1$, 
as an $\eps_{n}$-midpoint of the already defined
$\alpha(\tfrac{\kay-1}{2^n})$ and $\alpha(\tfrac{\kay+1}{2^n})$.


In this way we define $\alpha(t)$ for $t\in W$,
where $W$ denotes the set of dyadic rationals in $[0,1]$.
For any $t\in[0,1]$ consider a sequence of $t_n\in W$ such that $t_n\to t$ as $n\to\infty$.
Note that the sequence $\alpha(t_n)$ converges;
define $\alpha(t)$ as its limit.
It is easy to see that $\alpha(t)$
does not depend on the choice of the sequence $t_n$
and $\alpha\:[0,1]\to\spc{X}$ is a path from $x$ to $y$.
Moreover,
\[\begin{aligned}
\length\alpha&\le \dist{x}{y}{}+\sum_{n=1}^\infty 2^{n-1}\cdot\eps_n\le
\\
&\le \dist{x}{y}{}+\tfrac\eps2.
\end{aligned}
\eqlbl{eq:eps-midpoint}
\]
Since $\eps>0$ is arbitrary, we get (\ref{SHORT.lem:mid>length}).

To prove (\ref{SHORT.lem:mid>geod:geod}), 
one should repeat the same argument 
taking midpoints instead of $\eps_n$-midpoints.
In this case \ref{eq:eps-midpoint} holds for $\eps_n=\eps=0$.
\qeds

Since in a compact space a sequence of $1/n$-midpoints $z_n$ contains a convergent subsequence, Lemma  \ref{lem:mid>geod} immediately implies

\begin{thm}{Proposition}
A proper length space is geodesic.
\end{thm}

\begin{thm}{Hopf--Rinow theorem}\label{thm:Hopf-Rinow}
Any complete, locally compact length space is proper.
\end{thm}

\parit{Proof.}
Let $\spc{X}$ be a locally compact length space.
Given $x\in \spc{X}$, denote by $\rho(x)$ the supremum of all $R>0$ such that
the closed ball $\cBall[x,R]$ is compact.
Since $\spc{X}$ is locally compact 
$$\rho(x)>0\ \ \text{for any}\ \ x\in \spc{X}.\eqlbl{eq:rho>0}$$
It is sufficient to show that $\rho(x)=\infty$ for some (and therefore any) point $x\in \spc{X}$.

Assume the contrary; that is, $\rho(x)<\infty$.

\begin{clm}{} $B=\cBall[x,\rho(x)]$ is compact for any $x$.
\end{clm}

Indeed, $\spc{X}$ is a length space;
therefore for any $\eps>0$, 
the set $\cBall[x,\rho(x)-\eps]$ forms a compact $\eps$-net in $B$.
Since $B$ is closed and hence complete, it has to be compact.
\claimqeds

\begin{clm}{} $|\rho(x)-\rho(y)|\le \dist{x}{y}{\spc{X}}$,
in particular $\rho\:\spc{X}\to\RR$ is a continuous function.
\end{clm}

Indeed, 
assume the contrary; that is, $\rho(x)+|x-y|<\rho(y)$ for some $x,y\in \spc{X}$. 
Then 
$\cBall[x,\rho(x)+\eps]$ is a closed subset of $\cBall[y,\rho(y)]$ for some $\eps>0$.
Then  compactness of $\cBall[y,\rho(y)]$ implies compactness of $\cBall[x,\rho(x)+\eps]$, a contradiction.\claimqeds

Set $\eps=\min_{y\in B}\{\rho(y)\}$; 
the minimum is defined since $B$ is compact.
From \ref{eq:rho>0}, we have $\eps>0$.

Choose a finite $\tfrac\eps{10}$-net $\{a_1,a_2,\dots,a_n\}$ in $B$.
The union $W$ of the closed balls $\cBall[a_i,\eps]$ is compact.
Clearly 
$\cBall[x,\rho(x)+\frac\eps{10}]\subset W$.
Therefore $\cBall[x,\rho(x)+\frac\eps{10}]$ is compact;
a contradiction.
\qeds

\begin{thm}{Exercise}\label{exercise from BH}
Construct a geodesic space which is locally compact,
but whose completion is neither geodesic nor locally compact.
\end{thm}



%%%%%%%%%%%%%%%%%%%%%%%%%%%%%%%%%%%%%%%%%%%%%%%%%%%%%%%%%%%%%%%%%%%%%%%%%%%%%%%%%%%%%%











\section{Model angles and triangles.}\label{sec:mod-tri/angles}

Let $\spc{X}$ be a metric space, 
$p,q,r\in \spc{X}$. 
Let us define its \emph{model triangle}\index{model triangle} $\trig{\~p}{\~q}{\~r}$ 
(briefly, 
$\trig{\~p}{\~q}{\~r}=\modtrig(p q r)_{\EE^2}$%
\index{$\modtrig$!$\modtrig({*}{*}{*})_{\EE^2}$}) to be a triangle in the plane $\EE^2$ such that
\[\dist{\~p}{\~q}{}=\dist{p}{q}{},
\ \ \dist{\~q}{\~r}{}=\dist{q}{r}{},
\ \ \dist{\~r}{\~p}{}=\dist{r}{p}{}.\]

The same way we can define a \emph{hyperbolic} and \emph{spherical model triangles} $\modtrig(p q r)_{\HH^2}$, $\modtrig(p q r)_{\SS^2}$
in the hyperbolic plane $\HH^2$ and sphere $\SS^2$.
In the latter case the model triangle is said to be defined if in addition
\[\dist{p}{q}{}+\dist{q}{r}{}+\dist{r}{p}{}< 2\cdot\pi.\]
In this case it also exists and unique up to isometry of $\SS^2$.

If 
$\trig{\~p}{\~q}{\~r}=\modtrig(p q r)_{\EE^2}$ 
and $\dist{p}{q}{},\dist{p}{r}{}>0$, 
the angle measure of 
$\trig{\~p}{\~q}{\~r}$ at $\~p$ 
will be called \emph{model angle} of triple $p$, $q$, $r$ and it will be denoted by
$\angk p q r_{\EE^2}$%
\index{$\tangle$!$\angk{{*}}{{*}}{{*}}$}.
The same way we define $\angk p q r_{\HH^2}$ and $\angk p q r_{\SS^2}$;
in the latter case  we assume in addition that the model triangle $\modtrig(p q r)_{\SS^2}$ is defined.


\begin{wrapfigure}[10]{r}{35mm}
\begin{lpic}[t(0mm),b(-10mm),r(0mm),l(0mm)]{pics/lem_alex1(0.4)}
\lbl[br]{17,59;$p$}
\lbl[r]{1,2;$q$}
\lbl[l]{86,13;$r$}
\lbl[lb]{67,32;$z$}
\end{lpic}
\end{wrapfigure}

\begin{thm}{Alexandrov's lemma}
\index{Alexandrov's lemma}
\index{lemma!Alexandrov's lemma}
\label{lem:alex}  
Let $p,q,r,z$ be distinct points in a metric space such that $z\in \l]p r\r[$.
Then 
the following expressions have the same sign:
\begin{subthm}{lem-alex-difference}
$\angk p q z
-\angk p q r$,
\end{subthm} 

\begin{subthm}{lem-alex-angle}
$\angk z q p
+\angk z q r -\pi$.
\end{subthm}

Moreover,
\[\angk q p r \ge \angk q p z +  \angk q z r,\]
with equality if and only if the expressions in (\ref{SHORT.lem-alex-difference}) and (\ref{SHORT.lem-alex-angle}) vanish.

The same holds for hyperbolic and spherical model angles, but in the latter case one has to assume in addition that
\[\dist{p}{q}{}+\dist{q}{r}{}+\dist{r}{p}{}< 2\cdot\pi.\]
\end{thm}

\parit{Proof.} By the triangle inequality, 
\[
\dist{p}{q}{}+\dist{q}{z}{}+\dist{z}{p}{}\le \dist{p}{q}{}+\dist{q}{r}{}+\dist{r}{p}{}< 2\cdot\varpi\kappa.
\]
Therefore the model triangle $\trig{\~p}{\~q}{\~z}=\modtrig(p q z)_{\EE^2}$ is defined.
Take 
a point $\~r$ on the extension of 
$[\~p \~z]$ beyond $\~z$ so that $\dist{\~p}{\~r}{}=\dist{p}{r}{}$ (and therefore $\dist{\~p}{\~z}{}=\dist{p}{z}{}$). 

Since the increasing the opposite side in the plane triangle increase the corresponding angle, 
the following expressions have the same sign:
\begin{enumerate}[(i)]
\item $\mangle\hinge{\~p}{\~q}{\~r}-\angk{p}{q}{r}$;
\item $\dist{\~p}{\~r}{}-\dist{p}{r}{}$;
\item $\mangle\hinge{\~z}{\~q}{\~r}-\angk{z}{q}{r}$.
\end{enumerate}
Since 
\[\mangle\hinge{\~p}{\~q}{\~r}=\mangle\hinge{\~p}{\~q}{\~z}=\angk{p}{q}{z}\]
and
\[ \mangle\hinge{\~z}{\~q}{\~r}
=\pi-\mangle\hinge{\~z}{\~p}{\~q}
=\pi-\angk{z}{p}{q},\]
the first statement follows.

For the second statement, construct $\trig{\~q}{\~z}{\~r'}=\modtrig(qzr)_{\EE^2}$ on the opposite side of $[\~q\~z]$ from $\trig{\~p}{\~q}{\~z}$.  
Since
\[\dist{\~p}{\~r'}{}\le \dist{\~p}{\~z}{} + \dist{\~z}{\~r'}{}=\dist{p}{z}{}+\dist{z}{r}{}=\dist{p}{r}{},\]
then 
\begin{align*}
\angk{q}{p}{z} + \angk{q}{z}{r} 
&
= 
\mangle\hinge{\~q}{\~p}{\~z}+ \mangle\hinge{\~q}{\~z}{\~r'} 
=
\\
&
= 
\mangle\hinge{\~q}{\~p}{\~r'}
\le
\\
&\le  \angk q p r.
\end{align*}
Equality holds if and only  if $\dist{\~p}{\~r'}{}=\dist{p}{r}{}$, 
as required.
\qeds

%%%%%%%%%%%%%%%%%%%%%%%%%%%%%%%%%%%%%%%%%%%%%%%%%%%%%%%%%%%%%%%%%%%%%%%%%

\section{Angles and the first variation.}\label{sec:angles}

Given a hinge $\hinge p x y$, we define its \emph{angle}\index{angle} as 
follows:\index{$\mangle$!$\mangle\hinge{{*}}{{*}}{{*}}$}
\[\mangle\hinge p x y
\df
\lim_{\bar x,\bar y\to p} \angk p{\bar x}{\bar y}_{\EE^2},\eqlbl{eq:def-angle}\]
where $\bar x\in\l]p x\r]$ and $\bar y\in\l]p y\r]$.


\begin{thm}{Lemma}\label{lem:k-K-angle}
For any three points $p,x,y$ in a metric space the following inequalities
\[
\begin{aligned}
|\angk p{x}{y}_{\SS^2}-\angk p{x}{y}_{\EE^2}|
&\le 
\dist[{{}}]{p}{x}{}\cdot\dist[{{}}]{p}{y}{},
\\
|\angk p{x}{y}_{\HH^2}-\angk p{x}{y}_{\EE^2}|
&\le 
\dist[{{}}]{p}{x}{}\cdot\dist[{{}}]{p}{y}{},
\end{aligned}
\eqlbl{eq:k-K}\]
hold whenever the left hand side is defined.
\end{thm}

The lemma above implies that 
the definition of angle \ref{eq:def-angle} one can use $\angk p{\bar x}{\bar y}_{\SS^2}$ or  $\angk p{\bar x}{\bar y}_{\HH^2}$ instead of $\angk p{\bar x}{\bar y}_{\EE^2}$.
In particular, may use Euclidean plane, 
so that the angle can be calculated from the  cosine law:
\[\cos\angk{p}{x}{y}_{\EE^2}
=
\frac{\dist[2]{p}{x}{}+\dist[2]{p}{y}{}-\dist[2]{x}{y}{}}{2\cdot \dist[{{}}]{p}{x}{}\cdot\dist[{{}}]{p}{y}{}}.\]

\parit{Proof.}
Note that 
\[\angk p{x}{y}_{\HH^2}\le\angk{p}{x}{y}_{\EE^2}\le \angk p{x}{y}_{\SS^2}.\]
Therefore
\begin{align*}
0\le \angk p{x}{y}_{\SS^2}-\angk p{x}{y}_{\HH^2}
\le& \angk p{x}{y}_{\SS^2}+\angk {x}p{y}_{\SS^2}+\angk {y}p{x}_{\SS^2}-
\\
&-\angk p{x}{y}_{\HH^2}-\angk {x}p{y}_{\HH^2}-\angk {y}p{x}_{\HH^2}
= 
\\
=&\area\modtrig(pxy)_{\SS^2}+\area\modtrig(pxy)_{\HH^2}.
\end{align*}
Thus, \ref{eq:k-K} follows since 
\begin{align*}
0
&\le
\area\modtrig(pxy)_{\HH^2}\le 
\\
&\le\area\modtrig(pxy)_{\SS^2}\le
\\
&\le\dist[{{}}]{p}{x}{}\cdot\dist[{{}}]{p}{y}{}.
\end{align*}
\qedsf



\begin{thm}{Triangle inequality for angles}
\label{claim:angle-3angle-inq}
Let  $[px^1]$, $[px^2]$ and $[px^3]$ %$\gamma^1, \gamma^2, \gamma^3$ 
be three geodesics in a metric space.
If all of the angles $\alpha^{i j}=\mangle\hinge p {x^i}{x^j}$ are defined then they satisfy the triangle inequality:
\[\alpha^{13}\le \alpha^{12}+\alpha^{23}.\]

\end{thm}




\parit{Proof.} 
Since $\alpha^{13}\le\pi$, we can assume that $\alpha^{12}+\alpha^{23}< \pi$.
Set $\gamma^i=\geod_{[px^i]}$.
Given any $\eps>0$, for all sufficiently small $t,\tau,s\in\RR_+$ we have
\begin{align*}
\dist{\gamma^1(t)}{\gamma^3(\tau)}{}
\le 
&\dist{\gamma^1(t)}{\gamma^2(s)}{}+\dist{\gamma^2(s)}{\gamma^3(\tau)}{}<\\
<
&\sqrt{t^2+s^2-2\cdot t\cdot  s\cdot \cos(\alpha^{12}+\eps)}+
\\
&+\sqrt{s^2+\tau^2-2\cdot s\cdot \tau\cdot \cos(\alpha^{23}+\eps)}\le
\\
\intertext{(Below we define 
$s(t,\tau)$ so that for 
$s=s(t,\tau)$, this chain of inequalities continues the following way.)}
\le
&\sqrt{t^2+\tau^2-2\cdot t\cdot \tau\cdot \cos(\alpha^{12}+\alpha^{23}+2\cdot \eps)}.
\end{align*}
Thus for any $\eps>0$, 
\[\alpha^{13}\le \alpha^{12}+\alpha^{23}+2\cdot \eps.\]
Hence the result.

\begin{wrapfigure}{r}{30mm}
\begin{lpic}[t(-0mm),b(-0mm),r(0mm),l(0mm)]{pics/s-choice(0.33)}
\lbl[rb]{45,101;$t$}
\lbl[rt]{45,30;$\tau$}
\lbl[W]{50,65;$s\ \ $}
\lbl[l]{18,60,-25;$=\alpha^{12}+\eps$}
\lbl[l]{18,69,24;$=\alpha^{23}+\eps$}
\end{lpic}
\end{wrapfigure}

To define $s(t,\tau)$, consider three rays $\~\gamma^1$, $\~\gamma^2$, $\~\gamma^3$ on a Euclidean plane starting at one point, such that $\mangle(\~\gamma^1,\~\gamma^2)=\alpha^{12}+\eps$, $\mangle(\~\gamma^2,\~\gamma^3)=\alpha^{23}+\eps$ and $\mangle(\~\gamma^1,\~\gamma^3)=\alpha^{12}+\alpha^{23}+2\cdot \eps$.
We parametrize each ray by length from the starting point.
Given two positive numbers $t,\tau\in\RR_+$, let $s=s(t,\tau)$ be %a 
the 
number such that 
$\~\gamma^2(s)\in[\~\gamma^1(t)\ \~\gamma^3(\tau)]$. Clearly $s\le\max\{t,\tau\}$, 
so $t,\tau,s$ may be taken sufficiently small.
\qeds 

\begin{thm}{Exercise}\label{ex:adjacent-angles}
Prove that the sum of adjacent angles is at least $\pi$.

More precisely: let $\spc{X}$ be a complete length space and $p,x,y,z\in \spc{X}$.
If $p\in \l] x y \r[$, then 
\[\mangle\hinge pxz+\mangle\hinge pyz\ge \pi\]
whenever  each angle on the left-hand side is defined.
\end{thm}


\begin{thm}{First variation inequality}\label{lem:first-var}
Assume for hinge $\hinge q p x$ 
the angle $\alpha=\mangle\hinge q p x$ is defined then
\[\dist{p}{\geod_{[qx]}(t)}{}
\le
\dist{q}{p}{}-t\cdot \cos\alpha+o(t).\]

\end{thm}

\parit{Proof.} Take sufficiently small $\eps>0$.
For all sufficiently small $t>0$, we have that 
\begin{align*}
 \dist{\geod_{[qp]}(t/\eps)}{\geod_{[qx]}(t)}{}
&\le 
\tfrac{t}{\eps}\cdot \sqrt{1+\eps^2 -2\cdot \eps\cdot \cos\alpha}+o(t)\le
\\
&\le \tfrac{t}{\eps} -t\cdot \cos\alpha + t\cdot \eps.
\end{align*}
Applying triangle inequality, we get 
\begin{align*}
\dist{p}{\geod_{[qx]}(t)}{}
&\le \dist{p}{\geod_{[qp]}(t/\eps)}{}+\dist{\geod_{[qp]}(t/\eps)}{\geod_{[qx]}(t)}{}
\le 
\\
&\le
\dist{p}{q}{} -t\cdot \cos\alpha + t\cdot \eps
\end{align*}
for any $\eps>0$ and all sufficiently small $t$.
Hence the result.
\qeds

\section{Space of directions and tangent space}
\label{sec:tangent-space+directions}

Let $\spc{X}$ be a metric space with defined angles.
Fix a point $p\in \spc{X}$. 

Consider the set $\mathfrak{S}_p$ 
of all nontrivial unit-speed geodesics  which start at $p$.
By \ref{claim:angle-3angle-inq} the triangle inequality holds for $\mangle$ on $\mathfrak{S}_p$;
that is, $(\mathfrak{S}_p,\mangle)$ 
forms a pseudometric space.

The metric space corresponding to  $(\mathfrak{S}_p,\mangle)$ is called \emph{space of geodesic directions} at $p$
and denoted as $\Sigma'_p$ or $\Sigma'_p\spc{X}$.
The elements of $\Sigma'_p$ are called \emph{geodesic directions} at $p$.
Each geodesic direction is formed by an equivalence class of geodesics starting from $p$ 
for the equivalence relation 
\[[px]\sim[py]\ \ \iff\ \ \mangle\hinge pxy=0.\]

%???
%\begin{thm}{Exercise}\label{ex:geod-CBB}
%Assume $\spc{L}$ is a $\CBB{}{}$ space,  and $[px]$, $[py]$ be two geodesics  which correspond to the same geodesic direction $\xi\in \Sigma'_p$.
%Show that 
%\[[px]\subset [py]\ \ \text{or}\ \ [px]\supset [py].\]
%
%\end{thm}

\begin{thm}{Exercise}\label{ex:geod-CBA}
Assume $\spc{U}$ is a $\Cat{}{}$ space
 with extendable geodesics;
that is for any geodesic in $\spc{U}$
is an arc in a both-side infinite local geodesic.

Show that the space of geodesic derections 
$\Sigma_p'$ is complete for any $p\in \spc{U}$.
\end{thm}

The completion of $\Sigma'_p$ is called \emph{space of directions} at $p$ and is denoted as $\Sigma_p$ or $\Sigma_p\spc{X}$.
The elements of $\Sigma_p$ are called \emph{directions} at $p$.

The Euclidean cone $\Cone\Sigma_p$ over the space of directions $\Sigma_p$ is called tangent space at  $p$ and denoted as $\T_p$ or $\T_p\spc{X}$.

The tangent space $\T_p$ could be also defined directly, without introducing the space of direction.
To do so consider the set $\mathfrak{T}_p$ of all geodesics starting at $p$, with arbitrary speed.
Given $\alpha,\beta\in \mathfrak{T}_p$,
set 
\[\dist{\alpha}{\beta}{\mathfrak{T}_p}
=
\lim_{\eps\to0} 
\frac{\dist{\alpha(\eps)}{\beta(\eps)}{\spc{X}}}\eps
\eqlbl{eq:dist-in-T_p}\]
Since the angles in $\spc{X}$ are defined, 
\ref{eq:dist-in-T_p}
defines a pseudometric on $\mathfrak{T}_p$.


The corresponding metric space admits a natuaral isometric identification with the cone $\T'_p=\Cone\Sigma'_p$.
The elements of $\T'_p$ are formed by the equivalence classes for the realtion 
\[\alpha\sim\beta\ \ \iff\ \ \dist{\alpha(t)}{\beta(t)}{\spc{X}}=o(t).\]
The completion of $\T'_p$ is therefore  natuaral isometric to $\T_p$.

The elements of $\T_p$ will be called tangent vector at $p$,
despite that $\T_p$ is only cone --- not a vector space.
The elements of $\T'_p$ will be called geodesic tangent vector at $p$.

\section{Hemisphere lemma}\label{curves-in-model}


\begin{thm}{Hemisphere lemma}
\label{lem:hemisphere}
For $\kappa>0$, any closed path of length $<2\cdot \varpi\kappa$ (respectively, $\le2\cdot \varpi\kappa$) in $\Lob2\kappa$ lies in an open (respectively, closed) hemisphere. 
\end{thm}

\parit{Proof.} By rescaling, we may assume that $\kappa=1$ and thus $\varpi\kappa=\pi$ and $\Lob2{\kappa}=\SS^2$.
Let $\alpha$ be a closed curve in $\SS^2$ of length $2\cdot\ell$.

Assume $\ell<\pi$.
Let $\check\alpha$ be a subarc of $\alpha$ of length $\ell$, with endpoints $p$ and $q$. 
Since $\dist{p}{q}{}\le\ell<\pi$, there is a unique geodesic $[pq]$ in $\SS^2$.  
Let $z$ be the midpoint of  $[pq]$.  
We claim that $\alpha$ lies in the open hemisphere centered at $z$.  
If not, $\alpha$ intersects the boundary  great circle in a point say $r$.
Without loss of generality we may assume that $r\in\check\alpha$. 
The arc $\check\alpha$ together with its reflection in $z$ form a closed curve of length $2\cdot \ell$ which passes through $r$ and its antipodal point $r'$.
Thus $\ell=\length \check\alpha\ge \dist{r}{r'}{}=\pi$, a contradiction.

If $\ell=\pi$, then either $\alpha$ is a local geodesic, and hence a great circle, 
or $\alpha$ may be strictly shortened by substituting a geodesic arc for a subarc of $\alpha$ 
whose endpoints $p^1,p^2$ are arbitrarily close to some point $p$ on  $\alpha$.
In the latter case,  $\alpha$ lies in a closed hemisphere obtained as a limit of closures of open hemispheres  containing the shortened curves as $p^1,p^2$ approach $p$.
\qeds



%\begin{thm}{Lemma}\label{lem:proj-to-conv}Let $K\subset\Lob2\kappa$ be a convex closed set.Then there is a short map $\map\:\Lob2\kappa\to K$ such that $\map(x)=x$ for any $x\in K$. \end{thm}

\begin{thm}{Exercise}\label{exr-crofton}
Build a proof of Hemisphere lemma
\ref{lem:hemisphere} based on Crofton's formula.
\end{thm}
