%%!TEX root = invitation.tex
\chapter{Preliminaries}

In this chapter we fix some conventions and recall the main definitions.
%We suggest to skip this chapter and use it as a quick reference while reading the rest. !!! I don't like telling the reader to skip something altogether. Not sure how to rephrase this !!!
The chapter may be used as a quick reference when reading the book.

To learn background in metric geometry, the reader  may consult the book of Burago, Burago and Ivanov \cite{BBI}.
 

\section{Metric spaces}
\label{sec:metric spaces}


The distance between two points $x$ and $y$ in a metric space $\spc{X}$ will be denoted by $\dist{x}{y}{}$ or $\dist{x}{y}{\spc{X}}$.
The latter notation is used if we need to emphasize 
that the distance is taken in the space ${\spc{X}}$.

The function 
\[\dist{x}{}{}\:y\mapsto \dist{x}{y}{}\]
is called the \emph{distance function}\index{distance function} from $x$. 

\begin{itemize}
\item The \emph{diameter}\index{Diameter} of metric space $\spc{X}$ is defined as
\[\diam \spc{X}=\sup\set{\dist{x}{y}{\spc{X}}}{x,y\in \spc{X}}.\]

\item Given $R\in[0,\infty]$ and $x\in \spc{X}$, the sets
\begin{align*}
\oBall(x,R)&=\{y\in \spc{X}\mid \dist{x}{y}{}<R\},
\\
\cBall[x,R]&=\{y\in \spc{X}\mid \dist{x}{y}{}\le R\}
\end{align*}
are called respectively the  \emph{open}\index{open ball} and  the \emph{closed  balls}\index{closed ball}   of radius $R$ with center $x$.
Again, if we need to emphasize that these balls are taken in the metric space $\spc{X}$,
we write 
\[\oBall(x,R)_{\spc{X}}\quad\text{and}\quad\cBall[x,R]_{\spc{X}}.\]
\end{itemize}

A metric space $\spc{X}$ is called \emph{proper}\index{proper space} if all closed bounded sets in $\spc{X}$ are compact. 
This condition is equivalent to each of the following statements:
\begin{enumerate}
\item For some (and therefore any) point $p\in \spc{X}$ and any $R<\infty$, 
the closed ball $\cBall[p,R]\subset\spc{X}$ is compact. 
\item The function $\dist{p}{}{}\:\spc{X}\to\RR$ is proper for some (and therefore any) point $p\in \spc{X}$;
that is, for any compact set $K\subset \RR$, its inverse image 
$\set{x\in \spc{X}}{\dist{p}{x}{\spc{X}}\in K}$
is compact.
\end{enumerate}

\section{Constructions}\label{sec:constructions}

\parbf{Product space.}
Given two metric spaces $\spc{U}$ and $\spc{V}$, the \emph{product space} 
$\spc{U}\times\spc{V}$ is defined as the set of all pairs $(u,v)$ in the set $\spc{U}\times\spc{V}$ 
with the metric defined by formulas
\[\dist{(u^1,v^1)}{(u^2,v^2)}{\spc{U}\times\spc{V}}=\sqrt{\dist[2]{u^1}{u^2}{\spc{U}}+\dist[2]{v^1}{v^2}{\spc{V}}}.\]


\parbf{Cone.}
The \emph{cone} $\spc{V}=\Cone\spc{U}$ over the metric space $\spc{U}$
is defined as the metric space whose underlying set consists of  equivalence classes in
$[0,\infty)\times \spc{U}$ with the equivalence relation ``$\sim$'' given by$(0,p)\sim (0,q)$ for any points $p,q\in\spc{U}$,
and whose metric is given by the cosine rule
\[
\dist{(p,s)}{(q,t)}{\spc{V}} 
=
\sqrt{s^2+t^2-2\cdot s\cdot t\cdot \cos\alpha},
\]
where $\alpha= \max\{\pi, \dist{p}{q}{\spc{U}}\}$.

The point in the cone $\spc{V}$ formed by the equivalence class of $0\times\spc{U}$ is called the \emph{tip of the cone} and is denoted by $0$ or $0_{\spc{V}}$.
The distance $\dist{0}{v}{\spc{V}}$ is called the norm of $v$ and is denoted by $|v|$ or $|v|_{\spc{V}}$.

\parbf{Suspension.}
The \emph{suspension} $\spc{V}=\Susp\spc{U}$ over the metric space $\spc{U}$
is defined as the metric space whose underlying set consists of equivalence classes in
$[0,\pi]\times \spc{U}$ with the equivalence relation ``$\sim$'' given by $(0,p)\sim (0,q)$ and $(\pi,p)\sim (\pi,q)$ for any points $p,q\in\spc{U}$,
and whose metric is given by the  spherical cosine rule
\[
\cos\dist{(p,s)}{(q,t)}{\Susp\spc{U}} 
=
\cos s\cdot\cot t-\sin s\cdot\sin t\cdot\cos\alpha,
\]
where $\alpha= \max\{\pi, \dist{p}{q}{\spc{U}}\}$.

The points in $\spc{V}$ formed by the equivalence class of $0\times\spc{U}$ and $\pi\times\spc{U}$ are called  the \emph{north} and the  \emph{south poles} of the suspension.

\begin{thm}{Exercise}\label{ex:product-cone}
Let $\spc{U}$ be a metric space.
Show that the spaces 
\[\RR\times \Cone\spc{U}\quad\text{and}\quad\Cone[\Susp\spc{U}]\]
are  isometric.
\end{thm}




\section{Geodesics, triangles and hinges}
\label{sec:geods}

\parbf{Geodesics.}
Let $\spc{X}$ be a metric space 
and $\II$\index{$\II$} be a real interval. 
A globally isometric map $\gamma\:\II\to \spc{X}$ is called a \emph{geodesic}\index{geodesic}%
\footnote{Various authors call it differently: \emph{shortest path}, \emph{minimizing geodesic}.}; 
in other words, $\gamma\:\II\to \spc{X}$ is a geodesic if 
\[\dist{\gamma(s)}{\gamma(t)}{\spc{X}}=|s-t|\]
for any pair $s,t\in \II$.

We say that  $\gamma\:\II\to \spc{X}$ is a geodesic from point $p$ to point $q$ if $\gamma$ is a geodesic and 
$\II=[a,b]$ and $p=\gamma(a)$, $q=\gamma(b)$. 
In this case the image of $\gamma$ is denoted by $[p q]$\index{$[{*}{*}]$} and with an abuse of notations  we also call it a \emph{geodesic}\index{geodesic}.
Given a geodesic $[pq]$ we can parametrize it by distance to $p$;
this parametrization will be denoted as $\geod_{[p q]}(t)$.
Clearly $\geod_{[p q]}(t)$ is a geodesic in the sense of the first definition.

We may write $[p q]_{\spc{X}}$ 
to emphasize that the geodesic $[p q]$ is in the space  ${\spc{X}}$.
Also we use the following short-cut notation:
\begin{align*}
\l] p q \r[&=[pq]\backslash\{p,q\},
&
\l] p q \r]&=[pq]\backslash\{p\},
&
\l[ p q \r[&=[pq]\backslash\{q\}.
\end{align*}



In general, a geodesic between $p$ and $q$ need not exist and if it exists, it need not  be unique.  However,  once we write $[p q]$ we mean that we made a choice of geodesic.

A metric space is called \emph{geodesic}\index{geodesic} if any pair of its points can be joined by a geodesic. 

A \emph{constant-speed geodesic} is a  linear reparameterization of a geodesic, say by $at+b$ where $t$ is the geodesic parameter; then $a$ is the   \emph{speed}\index{speed}.
A constant-speed geodesic parameterized by $[0,1]$ is called a \emph{geodesic path}\index{geodesic path}.
Given a geodesic $[p q]$,
we denote by $\geodpath_{[pq]}$ the corresponding geodesic path;
that is,
$$\geodpath_{[pq]}(t)\z\equiv\geod_{[pq]}(t\cdot\dist[{{}}]{p}{q}{}).$$

A curve $\gamma\:\II\to \spc{X}$  is called a \emph{local geodesic}\index{geodesic!local geodesic}, if for any $t\in\II$ there is a neighborhood $U$ of $t$ in $\II$ such that the restriction $\gamma|_U$ is a  geodesic. 


\parbf{Triangles.}
For a triple of points $p,q,r\in \spc{X}$, a choice of a triple of geodesics $([q r], [r p], [p q])$ will be called a \emph{triangle}\index{triangle}; we will use the short notation 
$\trig p q r=([q r], [r p], [p q])$\index{$\trig {{*}}{{*}}{{*}}$}.
Again, given a triple $p,q,r\in \spc{X}$ there may be no triangle 
$\trig p q r$ simply because one of the pairs of these points can not be joined by a geodesic, and also many different triangles with these vertexes may exist, any of which can be denoted by $\trig p q r$.
Once we write $\trig p q r$, it means that we made a choice of such a triangle, 
that is, a choice of each $[q r], [r p]$ and $[p q]$.
The value $\dist{p}{q}{}+\dist{q}{r}{}+\dist{r}{p}{}$ will be called the \emph{perimeter of the triangle} $\trig p q r$.

\parbf{Hinges.}
Let $p,x,y\in \spc{X}$ be a triple of points such that $p$ is distinct from $x$ and $y$.
A pair of geodesics $([p x],[p y])$ will be called  a \emph{hinge}\index{hinge} and it will be denoted by 
$\hinge p x y=([p x],[p y])$\index{$\hinge{{*}}{{*}}{{*}}$}.


\parbf{Convex sets.}\label{def:convex-set}
Let $\spc{X}$ be a metric space. 
A set $A\subset\spc{X}$ is called 
\emph{convex}%
\index{convex set}
if for every two points $p,q\in A$, 
\emph{every} geodesic $[pq]$ of $\spc{X}$ 
lies in $A$.

A set $A\subset\spc{X}$ is called 
\emph{locally convex}
if every point $a\in A$ admits an open neighborhood $\Omega\ni a$
such that any geodesic lying in $\Omega$ and with ends in $A$ lies completely in $A$.


Note that any open set is locally convex by definition.

\section{Length spaces}\label{sec:intrinsic}

Recall that a path or a  curve is a continuous map from a real interval to a space.

\begin{thm}{Definition}
Let $\spc{X}$ be a metric space and
$\alpha\: \II\to \spc{X}$ be a curve.
We define the \emph{length}\index{length} of $\alpha$ as 
\[
\length \alpha \df \sup_{t_0\le t_1\le\ldots\le t_n}\sum_i \dist{\alpha(t_i)}{\alpha_{i+1}}{}
\]
\end{thm}

Directly from the definition, it follows that if a path $\alpha$ connects two points $x$ and $y$ \footnote{That is, such that $\alpha(0)=x$ and $\alpha(1)=y$.} then 
\[\length\alpha\ge \dist{x}{y}{}.\]

If for any $\eps>0$ and any pair of points $x$ and $y$ in a metric space $\spc{X}$ there is a path $\alpha$ connecting $x$ to $y$ such that
\[\length\alpha< \dist{x}{y}{}+\eps,\]
then $\spc{X}$ is called a \emph{length space}.

Note that any geodesic space is a length space;
as can be seen from the following example, the converse does not hold.


\begin{thm}{Example}
Let $\spc{X}$ be obtained by gluing a countable collection of disjoint intervals $\II_i$ of length $1+1/i$, where for each $\II_i$ one end is glued to $p=\{0\}$ and the other to $q=\{1\}$.
Then $\spc{X}$ carries a natural complete length metric  with respect to which $\dist{p}{q}{}=1$ but there is no geodesic connecting $p$ to $q$.
\end{thm}



\begin{thm}{Exercise}\label{ex:no-geod}
Give an example of a complete length space for which no pair of distinct points can be joined by a geodesic.
\end{thm}

Let $\spc{X}$ be a metric space and $x,y\in\spc{X}$.

\begin{enumerate}[(i)]
\item A point $z\in \spc{X}$ is called a \emph{midpoint} between $x$ and $y$
if 
\[\dist{x}{z}{}=\dist{y}{z}{}=\tfrac12\cdot\dist[{{}}]{x}{y}{}.\]
\item Assume $\eps\ge 0$.
A point $z\in \spc{X}$ is called an \emph{$\eps$-midpoint} between $x$ and $y$
if 
\[\dist{x}{z}{},\dist{y}{z}{}<\tfrac12\cdot\dist[{{}}]{x}{y}{}+\eps.\]
\end{enumerate}


Note that a $0$-midpoint is the same as a midpoint.


\begin{thm}{Lemma}\label{lem:mid>geod}
Let $\spc{X}$ be a complete metric space.
\begin{subthm}{lem:mid>length}
Assume that for any pair of points $x,y\in \spc{X}$  
 and any $\eps>0$
there is a $\eps$-midpoint $z$.
Then $\spc{X}$ is a length space.
\end{subthm}

\begin{subthm}{lem:mid>geod:geod}
Assume that for any pair of points $x,y\in \spc{X}$ 
there is a midpoint $z$.
Then $\spc{X}$ is a geodesic space.
\end{subthm}
\end{thm}

\parit{Proof.}
We first prove (\ref{SHORT.lem:mid>length}).
Let $x,y\in \spc{X}$ be a pair of points.

Set $\eps_n=\frac\eps{2^{2\cdot n}}$.

Set $\alpha(0)=x$ and $\alpha(1)=y$.

Let $\alpha(\tfrac12)$ be an $\eps_1$-midpoint of $\alpha(0)$ and $\alpha(1)$.
Further, let $\alpha(\frac14)$ 
and $\alpha(\frac34)$ be $\eps_2$-midpoints 
for the pairs $(\alpha(0),\alpha(\tfrac12)$ 
and $(\alpha(\tfrac12),\alpha(1)$ respectively.
Applying the above procedure recursively,
on the $n$-th step we define $\alpha(\tfrac{\kay}{2^n})$,
for every odd integer $\kay$ such that $0<\tfrac\kay{2^n}<1$, 
as an $\eps_{n}$-midpoint of the already defined
$\alpha(\tfrac{\kay-1}{2^n})$ and $\alpha(\tfrac{\kay+1}{2^n})$.


In this way we define $\alpha(t)$ for $t\in W$,
where $W$ denotes the set of dyadic rationals in $[0,1]$.
For any $t\in[0,1]$ consider a sequence of $t_n\in W$ such that $t_n\to t$ as $n\to\infty$.
Note that the sequence $\alpha(t_n)$ is Cauchy and hence it converges;
define $\alpha(t)$ as its limit.
It is easy to see that $\alpha(t)$
does not depend on the choice of the sequence $t_n$
and $\alpha\:[0,1]\to\spc{X}$ is a path from $x$ to $y$.
Moreover,
\[\begin{aligned}
\length\alpha&\le \dist{x}{y}{}+\sum_{n=1}^\infty 2^{n-1}\cdot\eps_n\le
\\
&\le \dist{x}{y}{}+\tfrac\eps2.
\end{aligned}
\eqlbl{eq:eps-midpoint}
\]
Since $\eps>0$ is arbitrary, we get (\ref{SHORT.lem:mid>length}).

To prove (\ref{SHORT.lem:mid>geod:geod}), 
one should repeat the same argument 
taking midpoints instead of $\eps_n$-midpoints.
In this case \ref{eq:eps-midpoint} holds for $\eps_n=\eps=0$.
\qeds

Since in a compact space a sequence of $1/n$-midpoints $z_n$ contains a convergent subsequence, Lemma  \ref{lem:mid>geod} immediately implies

\begin{thm}{Proposition}
A proper length space is geodesic.
\end{thm}

\begin{thm}{Hopf--Rinow theorem}\label{thm:Hopf-Rinow}
Any complete, locally compact length space is proper.
\end{thm}

\parit{Proof.}
Let $\spc{X}$ be a locally compact length space.
Given $x\in \spc{X}$, denote by $\rho(x)$ the supremum of all $R>0$ such that
the closed ball $\cBall[x,R]$ is compact.
Since $\spc{X}$ is locally compact 
$$\rho(x)>0\ \ \text{for any}\ \ x\in \spc{X}.\eqlbl{eq:rho>0}$$
It is sufficient to show that $\rho(x)=\infty$ for some (and therefore any) point $x\in \spc{X}$.

Assume the contrary; that is, $\rho(x)<\infty$.

\begin{clm}{} $B=\cBall[x,\rho(x)]$ is compact for any $x$.
\end{clm}

Indeed, $\spc{X}$ is a length space;
therefore for any $\eps>0$, 
the set $\cBall[x,\rho(x)-\eps]$ forms a compact $\eps$-net in $B$.
Since $B$ is closed and hence complete, it has to be compact.
\claimqeds

\begin{clm}{} $|\rho(x)-\rho(y)|\le \dist{x}{y}{\spc{X}}$,
in particular $\rho\:\spc{X}\to\RR$ is a continuous function.
\end{clm}

Indeed, 
assume the contrary; that is, $\rho(x)+|x-y|<\rho(y)$ for some $x,y\in \spc{X}$. 
Then 
$\cBall[x,\rho(x)+\eps]$ is a closed subset of $\cBall[y,\rho(y)]$ for some $\eps>0$.
Then  compactness of $\cBall[y,\rho(y)]$ implies compactness of $\cBall[x,\rho(x)+\eps]$, a contradiction.\claimqeds

Set $\eps=\min_{y\in B}\{\rho(y)\}$; 
the minimum is defined since $B$ is compact.
From \ref{eq:rho>0}, we have $\eps>0$.

Choose a finite $\tfrac\eps{10}$-net $\{a_1,a_2,\dots,a_n\}$ in $B$.
The union $W$ of the closed balls $\cBall[a_i,\eps]$ is compact.
Clearly 
$\cBall[x,\rho(x)+\frac\eps{10}]\subset W$.
Therefore $\cBall[x,\rho(x)+\frac\eps{10}]$ is compact,
a contradiction.
\qeds

\begin{thm}{Exercise}\label{exercise from BH}
Construct a geodesic space that is locally compact,
but whose completion is neither geodesic nor locally compact.
\end{thm}



%%%%%%%%%%%%%%%%%%%%%%%%%%%%%%%%%%%%%%%%%%%%%%%%%%%%%%%%%%%%%%%%%%%%%%%%%%%%%%%%%%%%%%

\section{Model angles and triangles.}\label{sec:mod-tri/angles}

Let $\spc{X}$ be a metric space, 
$p,q,r\in \spc{X}$. 
Let us define its \emph{model triangle}\index{model triangle} $\trig{\~p}{\~q}{\~r}$ 
(briefly, 
$\trig{\~p}{\~q}{\~r}=\modtrig(p q r)_{\EE^2}$%
\index{$\modtrig$!$\modtrig({*}{*}{*})_{\EE^2}$}) to be a triangle in the plane $\EE^2$ such that
\[\dist{\~p}{\~q}{}=\dist{p}{q}{},
\ \ \dist{\~q}{\~r}{}=\dist{q}{r}{},
\ \ \dist{\~r}{\~p}{}=\dist{r}{p}{}.\]

In the same way we can define the \emph{hyperbolic} and the \emph{spherical model triangles} $\modtrig(p q r)_{\HH^2}$, $\modtrig(p q r)_{\SS^2}$
in the hyperbolic plane $\HH^2$ and the unit sphere $\SS^2$.
In the latter case the model triangle is said to be defined if in addition
\[\dist{p}{q}{}+\dist{q}{r}{}+\dist{r}{p}{}< 2\cdot\pi.\]
In this case the model triangle again exists and is unique up to an isometry of $\SS^2$.

If 
$\trig{\~p}{\~q}{\~r}=\modtrig(p q r)_{\EE^2}$ 
and $\dist{p}{q}{},\dist{p}{r}{}>0$, 
the angle measure of 
$\trig{\~p}{\~q}{\~r}$ at $\~p$ 
will be called the \emph{model angle} of triple $p$, $q$, $r$ and it will be denoted by
$\angk p q r_{\EE^2}$%
\index{$\tangle$!$\angk{{*}}{{*}}{{*}}$}.
In the same way we define $\angk p q r_{\HH^2}$ and $\angk p q r_{\SS^2}$;
in the latter case  we assume in addition that the model triangle $\modtrig(p q r)_{\SS^2}$ is defined.
We may use the notation $\angk p q r$ if it is evident which of the model spaces $\HH^2$, $\EE^2$ or $\SS^2$ is meant.

\begin{wrapfigure}[6]{r}{28mm}
\begin{lpic}[t(-4mm),b(6mm),r(0mm),l(0mm)]{pics/lem_alex1(1)}
\lbl[lb]{10,23;$x$}
\lbl[rt]{1.5,.5;$p$}
\lbl[bl]{25,7.5;$y$}
\lbl[lb]{17,15;$z$}
\end{lpic}
\end{wrapfigure}

\begin{thm}{Alexandrov's lemma}
\index{Alexandrov's lemma}
\index{lemma!Alexandrov's lemma}
\label{lem:alex}  
Let $p,x,y,z$ be distinct points in a metric space such that $z\in \l]x y\r[$.
Then 
the following expressions for the Euclidean model angles have the same sign:
\begin{subthm}{lem-alex-difference}
$\angk x p z
-\angk x p y$,
\end{subthm} 

\begin{subthm}{lem-alex-angle}
$\angk z p x
+\angk z p y -\pi$.
\end{subthm}

Moreover,
\[\angk p x y \ge \angk p x z +  \angk p z y,\]
with equality if and only if the expressions in (\ref{SHORT.lem-alex-difference}) and (\ref{SHORT.lem-alex-angle}) vanish.

The same holds for hyperbolic and spherical model angles, 
but in the latter case one has to assume in addition that
\[\dist{p}{z}{}+\dist{p}{y}{}+\dist{x}{y}{}< 2\cdot\pi.\]

\end{thm}

\parit{Proof.} 
Consider the model triangle $\trig{\~x}{\~p}{\~z}=\modtrig(x p z)$.
Take 
a point $\~y$ on the extension of 
$[\~x \~z]$ beyond $\~z$ so that $\dist{\~x}{\~y}{}=\dist{x}{y}{}$ (and therefore $\dist{\~x}{\~z}{}=\dist{x}{z}{}$). 

\begin{wrapfigure}[6]{r}{23mm}
\begin{lpic}[t(-0mm),b(0mm),r(0mm),l(0mm)]{pics/alex-lemma-proof(1)}
\lbl[br]{0,3;$\~x$}
\lbl[lt]{22,1;$\~p$}
\lbl[bl]{18,22;$\~y$}
\lbl[bl]{8,13;$\~z$}
\end{lpic}
\end{wrapfigure}

Since increasing the opposite side in a plane triangle increases the corresponding angle, 
the following expressions have the same sign:
\begin{enumerate}[(i)]
\item $\mangle\hinge{\~x}{\~p}{\~y}-\angk{x}{p}{y}$;
\item $\dist{\~p}{\~y}{}-\dist{p}{y}{}$;
\item $\mangle\hinge{\~z}{\~p}{\~y}-\angk{z}{p}{y}$.
\end{enumerate}
Since 
\[\mangle\hinge{\~x}{\~p}{\~y}=\mangle\hinge{\~x}{\~p}{\~z}=\angk{x}{p}{z}\]
and
\[ \mangle\hinge{\~z}{\~p}{\~y}
=\pi-\mangle\hinge{\~z}{\~x}{\~p}
=\pi-\angk{z}{x}{p},\]
the first statement follows.

For the second statement, construct a model triangle $\trig{\~p}{\~z}{\~y'}\z=\modtrig(pzy)_{\EE^2}$ on the opposite side of $[\~p\~z]$ from $\trig{\~x}{\~p}{\~z}$.  
Note that 
\begin{align*}
\dist{\~x}{\~y'}{}
&\le \dist{\~x}{\~z}{} + \dist{\~z}{\~y'}{}=
\\
&=\dist{x}{z}{}+\dist{z}{y}{}=
\\
&=\dist{x}{y}{}.
\intertext{Therefore}
\angk{p}{x}{z} + \angk{p}{z}{y} 
&
= 
\mangle\hinge{\~p}{\~x}{\~z}+ \mangle\hinge{\~p}{\~z}{\~y'} 
=
\\
&
= 
\mangle\hinge{\~p}{\~x}{\~y'}
\le
\\
&\le  \angk p x y.
\end{align*}
Equality holds if and only  if $\dist{\~x}{\~y'}{}=\dist{x}{y}{}$, 
as required.
\qeds

%%%%%%%%%%%%%%%%%%%%%%%%%%%%%%%%%%%%%%%%%%%%%%%%%%%%%%%%%%%%%%%%%%%%%%%%%

\section{Angles and the first variation.}\label{sec:angles}

Given a hinge $\hinge p x y$, we define its \emph{angle}\index{angle} as 
follows:\index{$\mangle$!$\mangle\hinge{{*}}{{*}}{{*}}$}
\[\mangle\hinge p x y
\df
\lim_{\bar x,\bar y\to p} \angk p{\bar x}{\bar y}_{\EE^2},\eqlbl{eq:def-angle}\]
where $\bar x\in\l]p x\r]$ and $\bar y\in\l]p y\r]$.
The angle under the limit can be calculated from the  cosine law:
\[\cos\angk{p}{x}{y}_{\EE^2}
=
\frac{\dist[2]{p}{x}{}+\dist[2]{p}{y}{}-\dist[2]{x}{y}{}}{2\cdot \dist[{{}}]{p}{x}{}\cdot\dist[{{}}]{p}{y}{}}.\]

The following lemma implies that in 
the angle definition  \ref{eq:def-angle}, one can use $\angk p{\bar x}{\bar y}_{\SS^2}$ or  $\angk p{\bar x}{\bar y}_{\HH^2}$ instead of $\angk p{\bar x}{\bar y}_{\EE^2}$.


\begin{thm}{Lemma}\label{lem:k-K-angle}
For any three points $p,x,y$ in a metric space the following inequalities
\[
\begin{aligned}
|\angk p{x}{y}_{\SS^2}-\angk p{x}{y}_{\EE^2}|
&\le 
\dist[{{}}]{p}{x}{}\cdot\dist[{{}}]{p}{y}{},
\\
|\angk p{x}{y}_{\HH^2}-\angk p{x}{y}_{\EE^2}|
&\le 
\dist[{{}}]{p}{x}{}\cdot\dist[{{}}]{p}{y}{},
\end{aligned}
\eqlbl{eq:k-K}\]
hold whenever the left-hand side is defined.
\end{thm}


\parit{Proof.}
Note that 
\[\angk p{x}{y}_{\HH^2}\le\angk{p}{x}{y}_{\EE^2}\le \angk p{x}{y}_{\SS^2}.\]
Therefore
\begin{align*}
0&\le \angk p{x}{y}_{\SS^2}-\angk p{x}{y}_{\HH^2}\le
\\
&\le \angk p{x}{y}_{\SS^2}+\angk {x}p{y}_{\SS^2}+\angk {y}p{x}_{\SS^2}-\angk p{x}{y}_{\HH^2}-\angk {x}p{y}_{\HH^2}-\angk {y}p{x}_{\HH^2}
= 
\\
&=\area\modtrig(pxy)_{\SS^2}+\area\modtrig(pxy)_{\HH^2}.
\end{align*}
Thus, \ref{eq:k-K} follows since 
\begin{align*}
0
&\le
\area\modtrig(pxy)_{\HH^2}\le 
\\
&\le\area\modtrig(pxy)_{\SS^2}\le
\\
&\le\dist[{{}}]{p}{x}{}\cdot\dist[{{}}]{p}{y}{}.
\end{align*}
\qedsf



\begin{thm}{Triangle inequality for angles}
\label{claim:angle-3angle-inq}
Let  $[px^1]$, $[px^2]$ and $[px^3]$ %$\gamma^1, \gamma^2, \gamma^3$ 
be three geodesics in a metric space.
If all  the angles $\alpha^{i j}=\mangle\hinge p {x^i}{x^j}$ are defined then they satisfy the triangle inequality:
\[\alpha^{13}\le \alpha^{12}+\alpha^{23}.\]

\end{thm}




\parit{Proof.} 
Since $\alpha^{13}\le\pi$, we can assume that $\alpha^{12}+\alpha^{23}< \pi$.
Set $\gamma^i=\geod_{[px^i]}$.
Given any $\eps>0$, for all sufficiently small $t,\tau,s\in\RR_+$ we have
\begin{align*}
\dist{\gamma^1(t)}{\gamma^3(\tau)}{}
\le 
&\dist{\gamma^1(t)}{\gamma^2(s)}{}+\dist{\gamma^2(s)}{\gamma^3(\tau)}{}<\\
<
&\sqrt{t^2+s^2-2\cdot t\cdot  s\cdot \cos(\alpha^{12}+\eps)}+
\\
&+\sqrt{s^2+\tau^2-2\cdot s\cdot \tau\cdot \cos(\alpha^{23}+\eps)}\le
\\
\intertext{(Below we define 
$s(t,\tau)$ so that for 
$s=s(t,\tau)$, this chain of inequalities can be continued as follows:)}
\le
&\sqrt{t^2+\tau^2-2\cdot t\cdot \tau\cdot \cos(\alpha^{12}+\alpha^{23}+2\cdot \eps)}.
\end{align*}
Thus for any $\eps>0$, 
\[\alpha^{13}\le \alpha^{12}+\alpha^{23}+2\cdot \eps.\]
Hence the result.

\begin{wrapfigure}{r}{25mm}
\begin{lpic}[t(-0mm),b(-0mm),r(0mm),l(0mm)]{pics/s-choice(1)}
\lbl[w]{12,35,50;$\,t\,$}
\lbl[w]{12,12,-50;$\,\tau\,$}
\lbl[w]{14,22.5,4;$\,s\,$}
\lbl[lw]{9,20,-25;{\small $=\alpha^{12}+\eps$}}
\lbl[lw]{9,26,28;{\small $=\alpha^{23}+\eps$}}
\end{lpic}
\end{wrapfigure}

To define $s(t,\tau)$ to satisfy the above inequality, consider three rays $\~\gamma^1$, $\~\gamma^2$, $\~\gamma^3$ on a Euclidean plane starting at one point, such that $\mangle(\~\gamma^1,\~\gamma^2)=\alpha^{12}+\eps$, $\mangle(\~\gamma^2,\~\gamma^3)=\alpha^{23}+\eps$ and $\mangle(\~\gamma^1,\~\gamma^3)=\alpha^{12}+\alpha^{23}+2\cdot \eps$.
We parametrize each ray by the distance from the starting point.
Given two positive numbers $t,\tau\in\RR_+$, let $s=s(t,\tau)$ be 
the number such that 
$\~\gamma^2(s)\in[\~\gamma^1(t)\ \~\gamma^3(\tau)]$. Clearly $s\le\max\{t,\tau\}$, 
so $t,\tau,s$ may be taken sufficiently small.
\qeds 

\begin{thm}{Exercise}\label{ex:adjacent-angles}
Prove that the sum of adjacent angles is at least $\pi$.

More precisely: let $\spc{X}$ be a complete length space and $p,x,y,z\in \spc{X}$.
If $p\in \l] x y \r[$ then 
\[\mangle\hinge pxz+\mangle\hinge pyz\ge \pi\]
whenever  each angle on the left-hand side is defined.
\end{thm}


\begin{thm}{First variation inequality}\label{lem:first-var}
Assume that for a  hinge $\hinge q p x$ 
the angle $\alpha=\mangle\hinge q p x$ is defined. Then
\[\dist{p}{\geod_{[qx]}(t)}{}
\le
\dist{q}{p}{}-t\cdot \cos\alpha+o(t).\]

\end{thm}

\parit{Proof.} Take sufficiently small $\eps>0$.
For all sufficiently small $t>0$, we have 
\begin{align*}
 \dist{\geod_{[qp]}(t/\eps)}{\geod_{[qx]}(t)}{}
&\le 
\tfrac{t}{\eps}\cdot \sqrt{1+\eps^2 -2\cdot \eps\cdot \cos\alpha}+o(t)\le
\\
&\le \tfrac{t}{\eps} -t\cdot \cos\alpha + t\cdot \eps.
\end{align*}
Applying the triangle inequality, we get 
\begin{align*}
\dist{p}{\geod_{[qx]}(t)}{}
&\le \dist{p}{\geod_{[qp]}(t/\eps)}{}+\dist{\geod_{[qp]}(t/\eps)}{\geod_{[qx]}(t)}{}
\le 
\\
&\le
\dist{p}{q}{} -t\cdot \cos\alpha + t\cdot \eps
\end{align*}
for any $\eps>0$ and all sufficiently small $t$.
Hence the result.
\qeds

\section{Space of directions and tangent space}
\label{sec:tangent-space+directions}

Let $\spc{X}$ be a metric space with defined angles for all hinges.
Fix a point $p\in \spc{X}$. 

Consider the set $\mathfrak{S}_p$ 
of all nontrivial 
%unit-speed 
geodesics  which start at $p$.
By \ref{claim:angle-3angle-inq}, the triangle inequality holds for $\mangle$ on $\mathfrak{S}_p$;
that is, $(\mathfrak{S}_p,\mangle)$ 
forms a pseudometric space, satisfying all the conditions of a metric space  except that  the angle between distinct geodesics might vanish.

The metric space corresponding to  $(\mathfrak{S}_p,\mangle)$ is called the \emph{space of geodesic directions} at $p$, denoted by $\Sigma'_p$ or $\Sigma'_p\spc{X}$.
Elements of $\Sigma'_p$ are called \emph{geodesic directions} at $p$.
Each geodesic direction is formed by an equivalence class of geodesics in $\mathfrak{S}_p$
for the equivalence relation 
\[[px]\sim[py]\ \ \iff\ \ \mangle\hinge pxy=0.\]

%??? MOVE IT BACK ONCE IF CBB ARE BACK
%\begin{thm}{Exercise}\label{ex:geod-CBB}
%Assume $\spc{L}$ is a $\CBB{}{}$ space,  and $[px]$, $[py]$ be two geodesics  which correspond to the same geodesic direction $\xi\in \Sigma'_p$.
%Show that 
%\[[px]\subset [py]\ \ \text{or}\ \ [px]\supset [py].\]
%
%\end{thm}

\begin{thm}{Exercise}\label{ex:geod-CBA}
Assume $\spc{U}$ is a $\Cat{}{\kappa}$ proper length space
 with extendable geodesics;
that is, any geodesic in $\spc{U}$
is an arc in an infinite local geodesic defined on $\RR$.

Show that the space of geodesic directions 
$\Sigma_p'$ is complete for any $p\in \spc{U}$.
\end{thm}

The completion of $\Sigma'_p$ is called the \emph{space of directions} at $p$ and is denoted by $\Sigma_p$ or $\Sigma_p\spc{X}$.
Elements of $\Sigma_p$ are called \emph{directions} at $p$.

The Euclidean cone $\Cone\Sigma_p$ over the space of directions $\Sigma_p$ is called the tangent space at  $p$ and is denoted by $\T_p$ or $\T_p\spc{X}$.

The tangent space $\T_p$ could also be defined directly, without introducing the space of directions.
To do so, consider the set $\mathfrak{T}_p$ of all constant-speed geodesics starting at $p$. Given $\alpha,\beta\in \mathfrak{T}_p$,
set 
\[\dist{\alpha}{\beta}{\mathfrak{T}_p}
=
\lim_{\eps\to0} 
\frac{\dist{\alpha(\eps)}{\beta(\eps)}{\spc{X}}}\eps
\eqlbl{eq:dist-in-T_p}\]
Since the angles in $\spc{X}$ are defined, 
\ref{eq:dist-in-T_p}
defines a pseudometric on $\mathfrak{T}_p$.


The corresponding metric space admits a natural isometric identification with the cone $\T'_p=\Cone\Sigma'_p$.
The elements of $\T'_p$ are  equivalence classes for the relation 
\[\alpha\sim\beta\ \ \iff\ \ \dist{\alpha(t)}{\beta(t)}{\spc{X}}=o(t).\]
The completion of $\T'_p$ is therefore  naturally isometric to $\T_p$.

Elements of $\T_p$ will be called \index{tangent vector}\emph{tangent vectors} at $p$,
even though $\T_p$ is only a metric cone and need not be a vector space.
Elements of $\T'_p$ will be called \index{geodesic tangent vector}\emph{geodesic tangent vectors} at $p$.

\section{Hausdorff convergence}

It seems that Hausdorff convergence was first introduced by Hausdorff in \cite{hausdorff},
and a couple of years later an equivalent definition was given by Blaschke in \cite{blaschke}.
A refinement of this definition was introduced by Frol\'{\i}k in \cite{frolik},
and then rediscovered by Wijsman in \cite{wijsman}.  However, this refinement was a reversion  to  the \emph{closed convergence} introduced by Hausdorff in the original book. 
For this reason we call it Hausdorff convergence
instead of
\emph{Hausdorff--Blascke--Frol\'{\i}k--Wijsman convergence}.


Let $\spc{X}$ be a metric space and $A\subset \spc{X}$.
We will denote by $\dist{A}{}{}(x)$ the distance from $A$ to a point $x$ in $\spc{X}$;
that is,
$$\dist{A}{}{}(x)\df\inf\set{\dist{a}{x}{\spc{X}}}{a\in A}.$$

\begin{thm}{Definition of Hausdorff convergence}\label{def:hausdorff-coverge}
Given a sequence of closed sets $(A_n)_{n=1}^\infty$ in a metric space $\spc{X}$, 
a closed set $A_\infty\subset \spc{X}$ is called the Hausdorff limit of $(A_n)_{n=1}^\infty$,
briefly $A_n\to A_\infty$, if 
$$\dist{A_n}{}{}(x)\to\dist{A_\infty}{}{}(x)\ \ \t{as}\ \ n\to\infty$$
for any fixed $x\in \spc{X}$.

Then the sequence of closed sets $(A_n)_{n=1}^\infty$ is said to \emph{converge in the sense of Hausdorff}.
\end{thm}

\parbf{Example.}
Let $D_n$ be the disc in the coordinate plane 
with center $(0,n)$ and radius $n$.
Then $D_n$ converges to the upper half-plane as $n\to\infty$.

\begin{thm}{Exercise}\begin{subthm}{}
\label{ex:hausdorff-conv}
In a  metric space $\spc{X}$, show that if $A_n\to A_\infty$ as in Definition \ref{def:hausdorff-coverge}, then $A_\infty$ is the set of all points $p$ such that   $p_n\to p$ for some sequence of points  $p_n\in\spc{X}_n$.
\end{subthm}
\begin{subthm}{}
Show  the converse fails, even if  $\spc{X}$ is proper.\end{subthm}
\end{thm}


\begin{thm}{Selection theorem}
Let $\spc{X}$ be a proper metric space
and $(A_n)_{n=1}^\infty$ be a sequence of closed sets in $\spc{X}$.
Assume that for some (and therefore any) point  $x\in\spc{X}$, 
the sequence $\dist{A_n}{}{}(x)$ is bounded.
Then the sequence  $(A_n)_{n=1}^\infty$ has a convergent subsequence in the sense of Hausdorff.
\end{thm}

\parit{Proof.}
Since $X$ is proper,
we can choose a countable dense set $\{x_1,x_2,\dots\}$ in $\spc{X}$.
Note that the sequence $a_n=\dist{A_n}{}{}(x_\kay)$ is bounded for each $\kay$. 
Therefore, passing to a subsequence of $(A_n)_{n=1}^\infty$,
we can assume that $\dist{A_n}{}{}(x_\kay)$ converges as $n\to\infty$ for any fixed $\kay$.

Note that for each $n$, the function $\dist{A_n}{}{}\:\spc{X}\to\RR$ is 1-Lipschitz and nonnegative.
Therefore the sequence $\dist{A_n}{}{}$ converges pointwise to a 1-Lipschitz nonnegative function $f\:\spc{X}\to\RR$.

Set $A_\infty=f^{-1}(0)$.
Since $f$ is 1-Lipschitz, 
\[\dist{A_\infty}{}{}(y)\ge f(y)\] 
for any $y\in \spc{X}$.
It remains to show that 
\[\dist{A_\infty}{}{}(y)\le f(y)\] 
for any $y$.

Assume the contrary;
that is, 
\[f(z)<R<\dist{A_\infty}{}{}(z)\] 
for some $z\in \spc{X}$ and $R>0$.
Then for any large enough $n$, there is a point $z_n\in A_n$ such that
$\dist{x}{z_n}{}\le R$.
Since $\spc{X}$ is proper, we can pass to a partial limit $z_\infty$ of $z_n$ as $n\to\infty$.

It is clear that $f(z_\infty)=0$, that is, $z_\infty\in A_\infty$.
On the other hand, 
\begin{align*}
\dist{A_\infty}{}{}(y)
\le
\dist{z_\infty}{y}{}
\le R
<
\dist{A_\infty}{}{}(y),
\end{align*}
a contradiction.
\qeds

\section{Gromov--Hausdorff convergence}

\begin{thm}{Definition}\label{def:comp-metr}
Let $\set{\spc{X}_\alpha}{\alpha\in\IndexSet}$ be a set of metric spaces.
A metric $\rho$ on the disjoint union
$$\bm{X}=\bigsqcup_{\alpha\in\IndexSet} \spc{X}_\alpha$$
is called a  \emph{compatible metric}\index{compatible}
if the restriction of $\rho$ to every $\spc{X}_\alpha$ is the original metric on $\spc{X}_\alpha$.
\end{thm}

\begin{thm}{Definition}\label{def:GH}
Let $\spc{X}_1,\spc{X}_2,\dots$ 
and $\spc{X}_\infty$ be proper metric spaces 
and $\rho$ be a compatible metric on their disjoint union $\bm{X}$.
Assume that $\spc{X}_n$ is an open set in 
$(\bm{X},\rho)$ for each $n\ne\infty$, and 
$\spc{X}_n\to \spc{X}_\infty$ in $(\bm{X},\rho)$ as $n\to\infty$ in the sense of Hausdorff (see Definition~\ref{def:hausdorff-coverge}).

Then we say $\rho$ defines a 
\emph{convergence%
\footnote{Formally speaking, \emph{convergence} is the topology induced by $\rho$ on $\bm{X}$.} 
in the sense of Gromov--Hausdorff}%
\index{convergence in the sense of Gromov--Hausdorff},
and write $\spc{X}_n\to \spc{X}_\infty$ or $\spc{X}_n\xto{\rho} \spc{X}_\infty$.
The space $\spc{X}_\infty$ is called the limit space of the sequence $(\spc{X}_n)$ along $\rho$.
\end{thm}

Usually Gromov--Hausdorff convergence is defined differently. We prefer this definition since it induces convergence for a sequence of points $x_n\in\spc{X}_n$ (Exercise \ref{ex:hausdorff-conv}),
as well as 
weak convergence of measures $\mu_n$ on $\spc{X}_n$, and so on,
%as the corresponding
corresponding to convergence 
in the ambient space $(\bm{X},\rho)$.

Once we write $\spc{X}_n\to \spc{X}_\infty$, we mean that we made a choice of convergence.
Note that for a fixed sequence of metric spaces $\spc{X}_1,\spc{X}_2,\dots$, one may construct different Gromov--Hausdorff convergences, say $\spc{X}_n\xto{\rho} \spc{X}_\infty$ and $\spc{X}_n\xto{\rho'} \spc{X}_\infty'$,   whose limit spaces $\spc{X}_\infty$ and $\spc{X}_\infty'$ need not be isometric to each other. 

For example, for the constant sequence $\spc{X}_n\iso\RR_{\ge0}$, 
there is a convergence with limit $\spc{X}_\infty\iso\RR_{\ge0}$; 
guess the metric $\rho$ from the diagram.

\begin{center}
\begin{lpic}[t(-0mm),b(-0mm),r(0mm),l(0mm)]{pics/P2P(1)}
\lbl[b]{55,22;$\spc{X}_1$}
\lbl[b]{55,12;$\spc{X}_2$}
\lbl[b]{55,8;$\dots$}
\lbl[t]{55,0;$\spc{X}_\infty$}
\end{lpic}
\end{center}

For another metric $\rho'$ --- also guess it from the diagram ---
the limit space $\spc{X}_\infty'$ is isometric to the real line.

\begin{center}
\begin{lpic}[t(-0mm),b(-0mm),r(0mm),l(0mm)]{pics/P2R(1)}
\lbl[b]{55,22;$\spc{X}_1$}
\lbl[b]{35,12;$\spc{X}_2$}
\lbl[b]{15,8;$\dots$}
\lbl[t]{55,0;$\spc{X}_\infty'$}
\end{lpic}
\end{center}

%???DO WE NEED LIFTINGS???
\parbf{Liftings.}
Given a Gromov--Hausdorff convergence 
$\spc{X}_n\to \spc{X}_\infty$
and a point $p_\infty\in\spc{X}_\infty$, any sequence of points $p_n\in\spc{X}_n$ such that $p_n\to p$  will be called a \emph{lifting}\index{lifting of a point} of $p_\infty$.
In this case the point $p_n\in \spc{X}_n$ will be called a lifting of $p_\infty$ in $\spc{X}_n$, 
and the distance function $\dist{p_n}{}{}\:\spc{X}_n\to \RR$ 
will be called a \emph{lifting}\index{lifting of a distance function} 
of $\dist{p}{}{}\:\spc{X}\to \RR$ to $\spc{X}_n$.

Note that liftings are not uniquely defined.
We will be interested in the properties of liftings for sufficiently large $n$.

\begin{thm}{Selection theorem}\label{thm:gromov-selection}
Let $\spc{X}_n$ be a sequence of proper metric spaces 
with marked points $x_n\in \spc{X}_n$.
Assume that for any fixed $R,\eps>0$, there is $N=N(R,\eps)\in\NN$ 
such that for each $n$
the ball $\cBall[x_n,R]\subset \spc{X}_n$ admits a finite $\eps$-net with at most $N$ points.
Then there is a subsequence of $\spc{X}_n$ admitting a Gromov--Hausdorff convergence 
such that the sequence of marked points $x_n\in\spc{X}_n$ converges.
\end{thm}

\parit{Proof.}
From the main assumption in the theorem,
in each space $\spc{X}_n$ 
there is a sequence of points $z_{i,n}\in\spc{X}_n$ such that the following condition holds for a fixed sequence of integers $M_1<M_2<\dots$
\begin{itemize}
\item $\dist{z_{i,n}}{x_n}{\spc{X}_n}\le \kay+1$ if $i\le M_\kay$;
\item the points $z_{1,n},\dots,z_{M_\kay,n}$ form an $\tfrac1\kay$-net in $\cBall[x_n,\kay]_{\spc{X}_n}$.
\end{itemize}

Passing to a subsequence, we can assume that the sequence \[\ell_n=\dist{z_{i,n}}{z_{j,n}}{\spc{X}_n}\] 
converges for any $i$ and $j$.

Let us consider a countable set of points $\spc{W}=\{w_1,w_2,\dots\}$
equipped with the pseudometric defined as 
\[\dist{w_i}{w_j}{\spc{W}}
=
\lim_{n\to\infty}\dist{z_{i,n}}{z_{j,n}}{\spc{X}_n}.\]
Let $\hat{\spc{W}}$ be the metric space corresponding to $\spc{W}$;
that is, points in $\hat{\spc{W}}$ are equivalence classes in $\spc{W}$
for the relation $\sim$, where $w_i\sim w_j$ if and only it $\dist{w_i}{w_j}{\spc{W}}=0$
and 
\[\dist{[w_i]}{[w_j]}{\hat{\spc{W}}}\df\dist{w_i}{w_j}{\spc{W}}.\]
Denote by
$\spc{X}_\infty$ the completion of $\hat{\spc{W}}$.

It remains to show that there is a Gromov--Hausdorff convergence 
$\spc{X}_n\to\spc{X}_\infty$ such that the sequence $x_n\in\spc{X}_n$ converges.
To prove it, we need to construct a metric $\rho$ on the disjoint union of \[\bm{X}=\spc{X}_\infty\sqcup\spc{X}_1\sqcup\spc{X}_2\sqcup\dots\]  satisfying definitions \ref{def:comp-metr} and \ref{def:GH}.
The metric $\rho$ can be constructed as the maximal compatible metric
such that 
\[\rho(z_{i,n},w_i)\le\tfrac1m\]
for any $n\ge N_m$ and $i<I_m$ for a suitable choice of two sequences 
$I_m$, $N_m$ with $I_1=N_1=1$.
\qeds



\begin{thm}{Exercise}\label{ex:compact-proper-GH}
Let $\spc{X}_n$ be a sequence of metric spaces that  admit 
two convergences $\spc{X}_n\xto{\rho}\spc{X}_\infty$ and $\spc{X}_n\xto{\rho'}\spc{X}_\infty'$.
\begin{subthm}{}
If  $\spc{X}_\infty$ is compact, then $\spc{X}_\infty\iso\spc{X}_\infty'$.
\end{subthm}

\begin{subthm}{}
If  $\spc{X}_\infty$ is proper and there is a sequence of points $x_n\in \spc{X}_n$ 
which converges in both convergences, 
 then $\spc{X}_\infty\iso\spc{X}_\infty'$.
\end{subthm}
\end{thm}
