\chapter*{Semisolutions}
\parbf{Exercise~\ref{ex:two-components-of-M4}.}
Let $\spc{X}$ be a 4-point metric space.

Fix a tetrahedron $\triangle$ in $\RR^3$.
The verices of $\triangle$ , 
say $x_0$, $x_1$, $x_2$, $x_3$ can be identified with the points of $\spc{X}$.

Note that there is unique qadratic form $W$ on $\RR^3$
such that 
\[W(x_i-x_j)=\dist[2]{x_i}{x_j}{\spc{X}}\]
for all $i$ and $j$;
here $\dist{x_i}{x_j}{\spc{X}}$ denotes the distance beween $x_i$ and $x_j$ in $\spc{X}$.

By the triangle inequality $W(v)\ge 0$ 
for any vector $v$ parallel to one of the faces of $\triangle$.

Note that $\spc{X}$ is isometric to a 4-point subset in the Euclidean space
if and ony if $W(v)\ge 0$ for any vector $v$ in $\RR^3$.

\begin{wrapfigure}{r}{52mm}
\begin{lpic}[t(-5mm),b(3mm),r(0mm),l(0mm)]{pics/quad(1)}
\end{lpic}
\end{wrapfigure}

Therefore, if $\spc{X}$ is not of type $\mathcal{E}_4$ then $W(v)<0$ for some vector $v$.
From above, the vector $v$ must be transveral to each of the 4 faces of $\triangle$.
Therefore if we project $\triangle$ along $v$ to a plane transveral to $v$ we see one of the folwing two pictures.

Note that the combinatorics of the picture does not depend on the choice of $v$. 
Hence $\mathcal{M}_4\backslash\mathcal{E}_4$ not connected. 

It remains to show that if the combinatorics of the pictures for two spaces are the same then one can continuously deform one space into the other.
This can be easily done by deforming $W$ and applying a permutation of $x_0$, $x_1$, $x_2$, $x_3$ if necessary.





\parbf{Exercise~\ref{ex:convex-set}.}
The simplest proof we know requires the construction of tangent cones for Alexandrov spaces.

\parbf{Exercise~\ref{ex:non-contracting-map}.}
Given any pair of point $x_0,y_0\in K$, 
consider two sequences $x_0,x_1,\dots$ and $y_0,y_1,\dots$
such that 
and $x_{n+1}=f(x_n)$ and $y_{n+1}=f(y_n)$ for each $n$.

Since $K$ is compact, 
we can choose an increasing sequence of integers $n_k$
such that both sequences $(x_{n_i})_{i=1}^\infty$ and $(y_{n_i})_{i=1}^\infty$
converge.
In particular, both of these sequences  are Cauchy;
that is,
\[
|x_{n_i}-x_{n_j}|_K, |y_{n_i}-y_{n_j}|_K\to 0
\ \ 
\text{as}
\ \ \min\{i,j\}\to\infty.
\]


Since $f$ is non-contracting, we get
\[
|x_0-x_{|n_i-n_j|}|
\le 
|x_{n_i}-x_{n_j}|.
\]

It follows that  
there is a sequence $m_i\to\infty$ such that
\[
x_{m_i}\to x_0\ \ \text{and}\ \ y_{m_i}\to y_0\ \ \text{as}\ \ i\to\infty.
\leqno({*})\]

Set \[\ell_n=|x_n-y_n|_K.\]
Since $f$ is non-contracting, $(\ell_n)$ is a non-decreasing sequence.

By $({*})$,  $\ell_{m_i}\to\ell_0$ as $m_i\to\infty$.
It follows that $(\ell_n)$ is a constant sequence.

In particular 
\[|x_0-y_0|_K=\ell_0=\ell_1=|f(x_0)-f(y_0)|_K\]
for any pair of points $(x_0,y_0)$ in $K$.
That is, $f$ is distance preserving, in particular injective.

From $({*})$, we also get that $f(K)$ is everywhere dense.
Since $K$ is compact $f\:K\to K$ is surjective. Hence the result follows.\qeds


Exercise~\ref{ex:non-contracting-map} is a basic introductory lemma on  Gromov--Hausdorff distance \cite[7.3.30]{BBI}.
The proof presented here is not standard, 
it was given by Travis Morrison, 
a student in Anton Petrunin's MASS class at Penn State in the  Fall of 2011.

\parbf{Exercise~\ref{ex:product-cone}.} A point in $\RR\times \Cone \spc{U}$ can be described by a triple $(x,r,p)$, where $x\in \RR$, $r\in \RR_{\ge}$ and $p\in \spc{U}$.
Correspondingly, a point in $\Cone[\Susp\spc{U}]$ can be described by a triple $(\rho,\phi,p)$, where $\rho\in \RR_\ge$, $\phi\in [0,\pi]$ and $p\in \spc{U}$.

The map 
$\Cone[\Susp\spc{U}]\to\RR\times\Cone\spc{U}$ defied as
\[(\rho,\phi,p)\mapsto(\rho\cdot\cos\phi,\rho\cdot\sin\phi,p)\] 
is the needed isometry.

\parbf{Exercise~\ref{ex:no-geod}.}
Given a metric graph $\Gamma$ define $P_k(\Gamma)$ as follows. Let $\Gamma^b$ be the barycentric subdivision of $\Gamma$ with the natural metric. For any two adjacent vertices $p,q\in\Gamma^b$ substitute the edge $[pq]$ by  a countable collection of intervals $\{I_i\}_{i\ge 1}$ of length $\dist{p}{q}{}+\frac{\dist{p}{q}{}}{2^ki}$ where one end of each $I_i$ is glued to $p$ and the other to $q$. Note that the resulting space $P_k(\Gamma)$ is again a metric graph  with an inner metric. 

Let $\spc{X}_0=[0,1]$ and define $\spc{X}_k$ for $k\ge 1$ inductively as $\spc{X}_k=P_k(\spc{X}_{k-1})$.

Let $\spc{Y}_k$ be the set of vertices of $\spc{X}_k$ with the induced metric. By construction the inclusion $\spc{Y}_k\subset \spc{Y}_{k+1}$ is distance preserving.

Let $\spc{Y}_\infty=\cup_{k\ge 1}\spc{Y}_k$ with the obvious metric and let $\spc{Y}=\bar {\spc{Y}}_\infty$ be its metric completion. Then $\spc{Y}$ is a length space since it satisfies the almost midpoint property. But it is not hard to see that no two distinct points in $\spc{Y}$ can be connected by a shortest geodesic. \qeds


\parit{Alternative solution.}
Consider the space $c_0$ of all sequences converging to zero equipped with sup-norm.
The unit ball in $c_0$ is almost what you want --- there is no unique shortest curve between points. 
All we need now is to enhance {}\emph{bypasses} and to give disadvantage to \emph{straight lines}. 

This can be done by taking the distance element to be 
\[ds:=\frac1{2+\frac{x_1}2+\frac{x_2}4+\frac{x_3}8+\dots}\cdot\|dx\|_\infty,\] 

Note that on the unit ball in $c_0$ we obviously have 

\[
 \frac{1}{3}\|dx\|_\infty \le ds\le \|dx\|_\infty.
\]


If we have any continuous finite length curve $x(t)=(x_1(t),x_2(t),\dots)$ from $y$ to $z$ parametrized by the arclength, we can shorten it by replacing $x_m(t)$ by the maximum of the actual value of $x_m(t)$ and 
\[y_m+\tfrac td(z_m-y_m)+\tfrac 12 \min\{t,d-t\},\] 
where $d$ is the length of $x(t)$. 
This will work if $m$ is large enough since $\max_t\{|x_m(t)|\}\to 0$ as $m\to\infty$ and both functions change slower than the distance along the original curve.

The above solution was given by Fedor Nazarov, see \cite{nazarov}.

\parbf{Exercise~\ref{exercise from BH}.}
The following example is taken from \cite{BH}.

Consider the following subset of $\R^2$:

\[
\spc{X}
=
\bigl((0,1]\times\{0,1\}\bigr)
\cup
\bigl(\bigcup_{n\ge 1}\{1/n\}\times[0,1]\bigr)
\]
Consider the induced inner metric on $\spc{X}$. It's obviously locally compact and geodesic.
However, it's immediate to check that its metric completion \[\bar{\spc{X}}
=
\bigl([0,1]\times\{0,1\}\bigr)
\cup
\bigl(\bigcup_{n\ge 1}\{1/n\}\times[0,1]\bigr)\] 
is neither. \qeds 

\parbf{Exercise~\ref{ex:adjacent-angles}.}
Assume contrary; that is
\[\mangle\hinge pxz+\mangle\hinge pyz< \pi\]
By the  triangle inequality for angles (\ref{claim:angle-3angle-inq})
we have 
\[\mangle\hinge pxy<\pi\]
The latter contradicts the triangle inequality for the triangle $[\bar x p \bar y]$,
where the points
$\bar x\in \left]px\right]$ and $\bar y\in \left]py\right]$
are sufficiently close to $p$.


\parbf{Exercise~\ref{ex:geod-CBA}.}
Without loss of generality we can assume that $\spc{U}$ is a $\Cat{}{1}$.
Fix a sufficiently small $0<\eps<\pi$.

Recall that by Proposition~\ref{cor:loc-geod-are-min}, any local geodesic in  $\spc{U}$ of length $<\pi$ is a geodesic.

Consider a sequence of directions $\xi_n$ of geodesics $[pq_n]$.
Since the geodesics are extendable, by above 
we can assume that the distances $\dist{p}{q_n}{\spc{U}}$ are equal to $\eps$ for all $n$.

Since $\spc{U}$ is proper,
the sequence $q_n$ has a partial limit, say $q$.
It remains to note that the direction $\xi$ of $[pq]$ is the limit of directions $\xi_n$,
assuming the latter is defined.

\parbf{Exercise~\ref{ex:hausdorff-conv}.}
By the definition of the convergence
\[p\in A_\infty\quad\iff\quad\dist{A_n}{}{}(p)\to 0\quad\text{as}\quad n\to \infty.\] 
The latter equivalent to existence of a sequence $p_n\in A_n$ such that
$\dist{p_n}{p}{}\to0$ as $n\to \infty$;
or equivalently $p_n\to p$.
Hence the first statement follows.

To show that contrary does not hold,
consider the alternating sequence of two distinct closed set $A,B,A,B,\dots$;
note that it is not a converging sequence in the sence of Hausdorff.
On the other hand, the set of all limit points is well defined and it is formed by the intersection $A\cap B$.

\parit{Remark.} The set $\ushort{A}_\infty$ of all limits  of sequences $p_n\in A_n$ is called the \emph{lower closed limit}
and the set $\bar{A}_\infty$ of all partial limits of such sequences is called the \emph{upper closed limit}.

Clearly $\ushort{A}_\infty\subset \bar{A}_\infty$.
If $\ushort{A}_\infty=\bar{A}_\infty$ then it is called  the \emph{closed limit} of $A_n$.

For  subsets of a proper metric space closed limits coincide with limits in the sense of Hausdorff as we defined them.
All these notions were introduced by Felix Hausdorff in \cite{hausdorff}.



\parbf{Exercise~\ref{ex:compact-proper-GH}.}
Fix a countable dense set of points $\mathfrak{S}\subset\spc{X}_\infty$.
For each point $x\in \mathfrak{S}$, choose a sequence 
of points $x_n\in\spc{X}_n$ such that $x_n\xto{\rho} x$.

Applying a diagonal procedure, we can pass to a susequence of $ \spc{X}_n$ such that each of the constructed sequences $\rho'$-converge;
that is, $x_n\xto{\rho'} x'$ for some $x'\in\spc{X}_\infty'$.

This way we get a map $\mathfrak{S}\to\spc{X}_\infty'$ defined as $x\mapsto x'$.
Note that this map preserves the distances and therefore it can be extended to a distance preserving map $\spc{X}_\infty\to\spc{X}_\infty'$.

The same way we construct a distance preserving map $\spc{X}_\infty'\to\spc{X}_\infty$.

It remains to apply Exercise~\ref{ex:non-contracting-map}.

\parbf{Exercise~\ref{ex:cone+susp}.}
Given a point $x\in \Cone\spc{U}$, denote by $x'$ its projection to $\spc{U}$
and by $|x|$ the distance from $x$ to the tip of the cone;
if $x$ is the tip then $|x|=0$ and we can take any point of $\spc{U}$ as $x'$.

Let $p$, $q$, $x$, $y$
be a quadruple in $\Cone\spc{U}$.
Assume that the spherical model triangles $\trig{\~p'}{\~x'}{\~y'}=\modtrig(p'x'y')_{\SS^2}$ and $\trig{\~q'}{\~x'}{\~y'}=\modtrig(q'x'y')_{\SS^2}$ are defined.
Consider the following points in $\EE^3=\Cone\SS^2$: 
\begin{align*}
\~p&=|p|\cdot\~p',
&
\~q&=|q|\cdot\~q',
&
\~x&=|x|\cdot\~x',
&
\~y&=|y|\cdot\~y'.
\end{align*}

Note that
$\trig{\~p}{\~x}{\~y}\iso\modtrig(pxy)_{\EE^2}$
and
$\trig{\~q}{\~x}{\~y}\iso\modtrig(qxy)_{\EE^2}$.
Further note that if $\~z\in [\~x\~y]$ then
$\~z'=\~z/|\~z|$ lies on the geodesic $[\~x'\~y']$ in $\SS^2$.

The $\Cat{}{1}$ comparison for $\dist{p'}{q'}{}$ with $\~z'\in[\~x'\~y']$ implies the 
$\Cat{}{0}$ comparison $\dist{p}{q}{}$ with $\~z\in[\~x\~y]$.
The converse also holds if the $\Cat{}{0}$ comparison holds for all the quadruple of points $a\cdot p$, $b\cdot q$, $c\cdot x$, $d\cdot y$ for $a,b,c,d\ge 0$.

\parbf{Exercise~\ref{ex:product}.}
Fix a quadruple 
\begin{align*}
p&=(p_1,p_2),
&
q&=(q_1,q_2), 
&
x&=(x_1,x_2),
&
y&=(y_1,y_2)
\end{align*}
in $\spc{U}\times \spc{V}$.
For the quadruple $p_1,q_1,x_1,y_1$ in $\spc{U}$,
construct two model triangles $\trig{\~p_1}{\~x_1}{\~y_1}=\modtrig(p_1x_1y_1)_{\EE^2}$ 
and $\trig{\~q_1}{\~x_1}{\~y_1}=\modtrig(q_1x_1y_1)_{\EE^2}$.  
Similarly, for the quadruple $p_2,q_2,x_2,y_2$ in $\spc{V}$
construct two model triangles $\trig{\~p_2}{\~x_2}{\~y_2}$ and $\trig{\~q_2}{\~x_2}{\~y_2}$.

Consider four points in $\EE^4=\EE^2\times\EE^2$ 
\begin{align*}
\~p&=(\~p_1,\~p_2),
&
\~q&=(\~q_1,\~q_2),
&
\~x&=(\~x_1,\~x_2),
&
\~y&=(\~y_1,\~y_2).
\end{align*}
Note that the triangles $\trig{\~p}{\~x}{\~y}$ and $\trig{\~q}{\~x}{\~y}$ in $\EE^4$ are isometric to the model triangles 
$\modtrig(pxy)_{\EE^2}$ and $\modtrig(qxy)_{\EE^2}$.

If $\~z=(\~z_1,\~z_2)\in [\~x\~y]$ we have then $\~z_1\in [\~x_1\~y_1]$ and $\~z_2\in [\~x_2\~y_2]$ and
\begin{align*}
\dist[2]{\~z}{\~p}{\EE^4}&=\dist[2]{\~z_1}{\~p_1}{\EE^2}+\dist[2]{\~z_2}{\~p_2}{\EE^2},
\\
\dist[2]{\~z}{\~q}{\EE^4}&=\dist[2]{\~z_1}{\~q_1}{\EE^2}+\dist[2]{\~z_2}{\~q_2}{\EE^2},
\\
\dist[2]{p}{q}{\spc{U}\times\spc{V}}&=\dist[2]{p_1}{q_1}{\spc{U}}+\dist[2]{p_2}{q_2}{\spc{V}}.
\end{align*}
Therefore $\Cat{}{0}$ comparison for the quadruples $p_1,q_1,x_1,y_1$ in $\spc{U}$
and 
$p_2,q_2,x_2,y_2$ in $\spc{V}$ implies the 
$\Cat{}{0}$ comparison for the quadruples $p,q,x,y$ in $\spc{U}\times \spc{V}$.

\parbf{Exercise~\ref{ex:CAT-geodesic}.}
According to Lemma~\ref{lem:mid>geod}, it is sufficient to prove the existence of a midpoint for given two point $x$ and $y$.

For each $n$ choose a $\tfrac1n$-midpoint $z_n$;
that is a point such that
\[\dist{x}{z_n}{},\dist{y}{z_n}{}\le \tfrac12\cdot\dist{x}{y}{}+\tfrac1n.\]

From the $\Cat{}{0}$ comparison inequality for the quadruple $x,y,z_n,z_m$ we have that $\dist{z_m}{z_n}{}\to0$ as $n,m\to\infty$;
that is $z_n$ is Cauchy and hence converges.

Since the space is complete the sequence $z_n$ has a limit, say $z$, which is clearly a midpoint for the pair $x$ and $y$. 





\parbf{Exercise~\ref{ex:thin=cat}.} 
To prove the if part,
apply thinness of triangles $\trig pxy$ and $\trig qxy$, where $p$, $q$, $x$ and $y$ as in the definition of $\Cat{}{\kappa}$ comparison,
see page \pageref{page:CAT-comparison}.

\parit{Only-if part.} Fix a triangle $\trig pxy$ 
and two points $\bar x\in [px]$, $\bar y\in [py]$.
Apply the $\Cat{}{\kappa}$ comparison to the quadruples
$(\bar x, x, p, y)$ and 
and $(\bar x, \bar y, p, x)$ together with Alexandrov's lemma (\ref{lem:alex}).


\begin{wrapfigure}[7]{r}{39mm}
\begin{lpic}[t(-0mm),b(0mm),r(0mm),l(0mm)]{pics/resh(1)}
\lbl[t]{17.5,.5;$\~p$}
\lbl[b]{2,27.5;$\~x$}
\lbl[b]{30,27.5;$\~y$}
\lbl[b]{17,27.5;$\~z$}
\lbl[b]{12.5,18;$\~z_x$}
\lbl[b]{21,18;$\~z_y$}
\end{lpic}
\end{wrapfigure}
\parbf{Exercise~\ref{ex:short-map}.}
By Alexandrov's lemma (\ref{lem:alex}), 
there are nonoverlapping triangles 
$\trig{\~p}{\~x}{\~z_y}\iso\trig {\dot p}{\dot x}{\dot z}$ 
and 
$\trig{\~p}{\~y}{\~z_x}\iso\trig {\dot p}{\dot y}{\dot z}$
 inside the  triangle $\trig{\~p}{\~x}{\~y}$.

Connect  the points in each pair
$(\~z,\~z_x)$, 
$(\~z_x,\~z_y)$ 
and $(\~z_y,\~z)$ 
with arcs of circles centered at 
$\~y$, $\~p$, and $\~x$ respectively. 
Define $F$ as follows.
\begin{itemize}
\item Map  $\Conv\trig{\~p}{\~x}{\~z_y}$ isometrically onto  $\Conv\trig {\dot p}{\dot x}{\dot y}$;
similarly map $\Conv \trig{\~p}{\~y}{\~z_x}$ onto $\Conv \trig {\dot p}{\dot y}{\dot z}$.
\end{itemize}

\begin{itemize}
\item If $x$ is in one of the three circular sectors, say at distance $r$ from center of the circle, let $F(x)$ be the point on the corresponding segment 
$[p z]$, 
$[x z]$ 
or $[y z]$ whose distance from the lefthand endpoint of the segment is $r$.
\item Finally, if $x$ lies in the remaining curvilinear triangle $\~z \~z_x \~z_y$, 
set $F(x) = z$. 
\end{itemize}
By construction, $F$ satisfies the conditions. 

\parbf{Exercise~\ref{ex:convex-balls}.}
It is sufficient to show that 
\[\dist{p}{z}{}\le \max\{\dist{p}{x}{},\dist{p}{y}{}\}\]
for any $p$ and any $z\in[xy]$.

The statement follows since the triangle $\trig pxy$ is thin
and the above condition holds in the Euclidean plane.

In $\Cat{}{1}$ case, the proof is the same, but we need to assume in addition that 
$\max\{\dist{p}{x}{},\dist{p}{y}{}\}\le \tfrac\pi2$.

\parbf{Exercise~\ref{ex:locally-convex}.}
Denote by $K$ a closed connected locally convex set.
Note that $\dist{K}{}{}$ is convex in a neighborhood $\Omega\supset K$.

Since $K$ is locally convex,
it is locally path connected.
Since $K$ is connected and locally path connected it is path connected.

Fix two points $x,y\in K$. 
Let us connect $x$ to $y$ by a path $\alpha\:[0,1]\to K$.
By Theorem~\ref{thm:cat-unique}, the geodesic $[x\alpha(s)]$ 
uniquely defined and depends continuously on $s$.

If $[xy]=[x\alpha(1)]$ does not completely lie in $K$ then 
there is a value $s\in [0,1]$ such that $[x\alpha(s)]$ 
lies in $\Omega$,
but does not completely lie in $K$.
Therefore $f=\dist{K}{}{}$ is convex on
along $[x\alpha(s)]$.
Note that $f(x)=f(\alpha(s))=0$ and $f\ge 0$, 
therefore $f(x)= 0$ for any $x\in [x\alpha(s)]$;
that is, $x\in K$, a contradiction.


\parbf{Exercise~\ref{ex:closest-point}.}
Since $\spc{U}$ is proper, the set $K\cap \cBall[p,R]$ is compact for any $R<\infty$.
Hence the existence of $p^*$ follows.

Assume $p^*$ is not uniquely defined;
that is,  two distinct points in $K$, say $x$ and $y$, minimize the distance from $p$.
Sicne $K$ is convex the midpoint $z$ of $[xy]$ lies in $K$.

Note that $\dist{p}{z}{}<\dist{p}{x}{}=\dist{p}{y}{}$, a contradiction.

It remains to show that the map $p\mapsto p^*$ is short;
that is 
\[\dist{p}{q}{}\ge \dist{p^*}{q^*}{}.\eqlbl{eq:short-p-p}\]

Note that 
\[\mangle\hinge{p^*}{p}{q^*}\ge \tfrac\pi2
\quad
\text{and}
\quad
\mangle\hinge{q^*}{p}{p^*}\ge \tfrac\pi2,\] 
if the lefthand sides are defined. 


Construct the model triangles 
$\trig{\~p}{\~p^*}{\~q^*}$ and $\trig{\~p}{\~q}{\~q^*}$
of $\trig{p}{p^*}{q^*}$ and $\trig{p}{q}{q^*}$ so that 
the points $\~p^*$ and $\~q$ lie on the opposite sides from $[{\~p}{\~q^*}]$.

By comparison and triangle inequality for angles, we get \ref{claim:angle-3angle-inq}
\[\mangle\hinge{\~p^*}{\~p}{\~q^*}\ge \mangle\hinge{p^*}{p}{q^*}\ge \tfrac\pi2
\quad
\text{and}
\quad
\mangle\hinge{\~q^*}{\~q}{\~p^*}\ge \mangle\hinge{q^*}{q}{p^*}\ge \tfrac\pi2
\]
assuming that the left hand sides are defined. 
Hence 
\[\dist{\~p}{\~q}{}\ge \dist{\~p^*}{\~q^*}{}.\]
The latter is equivalent to \ref{eq:short-p-p}.

\parbf{Exercise~\ref{ex:supporting-planes}.}
By approximation, it is sufficient to consider the case when 
$A$ and $B$ have smooth boundary.
In the latter case the choice of $\dot A$ and $\dot B$ is unique;
and a variation argument shows that it satisfies the condition.

\begin{wrapfigure}{r}{44mm}
\begin{lpic}[t(0mm),b(-00mm),r(0mm),l(0mm)]{pics/compact-walls(1)}
\lbl[tr]{21,7.5;$0$}
\lbl[tr]{10,20;$A^i$}
\lbl[tr]{34,12;$A^j$}
\end{lpic}
\end{wrapfigure}

\parbf{Exercise~\ref{ex:compact-walls}.}
Fix $r>0$ and $R<\infty$ so that 
\[\oBall(0,r)\subset A^i\subset \oBall(0,R)\]
for each wall $A^i$.

It remains to note that all the intersections of  walls  have $\eps$-wide for
\[\eps=2\cdot \arcsin\tfrac rR.\]

\parbf{Exercise~\ref{ex:centrally-simmetric-walls}.}
Note that there is $R<\infty$
such that if $X$ is an intersection of arbitrary number of walls
and then there is an isometry of $X$ 
which moves any point $p\in X$ to a point in $\oBall(0,R)$.

The rest of the proof goes along the same lines as Exercise~\ref{ex:compact-walls}.

\parbf{Exercise~\ref{ex:null-homotopic}.}
Note that the existance of null-homotopy is equivalent to the following.
There are two one-parameter family of paths $\alpha_\tau$ and $\beta_\tau$, $\tau\in[0,1]$ 
such that 
\begin{itemize}
\item $\length\alpha_\tau$, $\length\beta_\tau<\pi$ for any $\tau$.
\item $\alpha_\tau(0)=\beta_\tau(0)$ and $\alpha_\tau(1)=\beta_\tau(1)$ for any $\tau$.
\item $\alpha_0(t)=\beta_0(t)$ for any $t$.
\item $\alpha_1(t)=\alpha(t)$ and $\beta_1(t)=\beta(t)$ for any $t$.
\end{itemize}

By Corollary~\ref{cor:discrete-paths},
the construction in Corollary~\ref{cor:path-geod} produces the same result for $\alpha_\tau$ and $\beta_\tau$.
Hence the result follows.

\parbf{Exercise~\ref{ex:geod-circle}.}
The following proof works for any compact locally simply connected metric space;
by uniqueness of geodesics (\ref{thm:cat-unique}) 
this class of spaces includes compact locally $\Cat{}{\kappa}$ length spaces.

Consider a shortest noncontractible closed curve $\gamma$ in the space.
(Note that it exists.)

Assume that $\gamma$ is not a geodesic circle,
that is  there are two points $p$ and $q$ on $\gamma$ such that the distance $\dist{p}{q}{}$ 
is shorter then the length of the arcs, say $\alpha_1$ and $\alpha_2$, of $\gamma$ from $p$ to $q$.
Consider the joints, say $\gamma_1$ and $\gamma_2$
of $[qp]$ with $\alpha_1$ and $\alpha_2$.
Note that
\begin{itemize}
 \item  $\gamma_1$ or $\gamma_2$ is noncontractible,
 \item $\length\gamma_1, \length\gamma_2<\length \gamma$,
\end{itemize}
a contradiction.


\parbf{Exercise~\ref{ex:unique-geod=CAT}.}
Assume $\spc{P}$ is not $\Cat{}{0}$ then
by Theorem~\ref{thm:PL-CAT} there a link $\Sigma$ of some simplex contains a closed geodesic $\alpha$ with length $4\cdot\ell<2\cdot\pi$.
Divide $\alpha$ into two equal arcs $\alpha_1$ and $\alpha_2$
parametrized by $[-\ell,\ell]$

Fix small $\delta>0$ and 
consider two curves in $\Cone\Sigma$ written in the polar coordinates as 
\[\gamma_i(t)=(\alpha_i(\tan \tfrac t\delta),\sqrt{\delta^2+t^2}).\]
Note that both form geodesics in $\Cone\Sigma$ and they have common ends.

Finally note that small neighborhood of the tip of $\Cone\Sigma$ admits an isometric embedding into $\spc{P}$.
Hence the statement follows.




\parbf{Exercise~\ref{ex:baricenric-flag}.}
Checking the flag condition is straightforward once you know the following description of the barycentric subdevision.

Each vertexes $v$ of the barycentric subdevision 
corresponds to a simplex $\triangle_v$ of the original triangulation.
A set vertices form a simplex in the subdevision, 
if it can be ordered, say as $v_1,\dots,v_\kay$,
such that the corresponding simplices form a nested sequence
\[\triangle_{v_1}\subset\dots\subset\triangle_{v_\kay}.\]




\parbf{Exercise~\ref{ex:flag>=pi/2}.}
The proof goes along the same lines as the proof of Flag condition~\ref{thm:flag}.
The only difference that geodesic spends time at least $\pi$ the star of each vertex.

\parbf{Exercise~\ref{ex:short-retracts}.}
Start with a copy of $\spc{U}$ for each vertex of $\Gamma$
and glue them recuresevely using the following operation and stages.

At the stage $n$ glue only spaces along $A^n$.

In one step of the stage, choose two connected components of partially glued copies and glue them along the corresponding $A^n$'s.
Apply Reshetnyak gluing theorem to show that each connected component after the stage is $\Cat{}{0}$ assuming that all connected component were $\Cat{}{0}$ before the step.


\parbf{Exercise~\ref{ex:tree}.}
The space $\spc{T}_n$ has natural cone structure with the vertex formed by completely degenerate tree --- all its edges have zero length.
Note that the space $\Sigma$, 
over which the cone is taken comes naturally with triangulation 
with all-right spherical simplicies.

It remains to check that the complex is flag 
and apply Exercise~\ref{ex:cone+susp}.

\parbf{Exercise~\ref{ex:flag-aspherical}.}
If the complex $\mathcal{S}$ is flag then its cubical analog $\square_{\mathcal{S}}$ is locally $\Cat{}{0}$ and therefore aspherical.

Assume now that the complex $\mathcal{S}$ is not flag. 
Extend it to a flag complex $\mathcal{T}$ by gluing a simplex in every complete subgraph of its 1-skeliton.

Note that the cubical analog $\square_{\mathcal{S}}$ is a proper subcomplex in $\square_{\mathcal{T}}$.
Since $\mathcal{T}$ is flag,
$\tilde\square_{\mathcal{T}}$,
the universal cover of $\square_{\mathcal{T}}$, is $\Cat{}{0}$.


Choose a minimal cube $Q$ in $\tilde\square_{\mathcal{T}}$ which is not presented in $\tilde\square_{\mathcal{S}}$.
By Exercise~\ref{ex:locally-convex}, $Q$ is convex set in $\tilde\square_{\mathcal{T}}$.
The closest point projection $\tilde\square_{\mathcal{T}}\to Q$ is a retraction.
It follows that the boundary $\partial Q$ is not contractable in $\tilde\square_{\mathcal{T}}\backslash\Int Q$ and therefore it is also not contractable in $\tilde\square_{\mathcal{S}}$.


\parbf{Exercise~\ref{ex:example-pi_infty-new}.}
Solution goes along the same lines as the proof of Lemma~\ref{lem:example-pi_infty}.
The only difference, the set $G$ formed by squares with the vertexes at the  midpoints of some edges of the cubical analog. 

\parbf{Exercise~\ref{ex:funny-S}.}
In the proof we apply the following lemma from \cite{edwards}; 
it follows from the disjoint discs property.

\medskip

\parbf{Lemma.}
\textit{Let $\spc{S}$ be a simplicial complex which 
is an $m$-dimensional homology manifold for some $m\ge 5$.
Assume all the vertices of
$\spc{S}$ have simply connected links.
Then $\spc{S}$ is a topological manifold.}

\medskip


It is sufficient to construct a simplicial complex $\spc{S}$
such that 
\begin{itemize}
\item $\spc{S}$ is a closed $(m-1)$-dimensional homology manifold;
\item $\pi_1(\spc{S}\backslash\{v\})\ne0$ for some vertex $v$ in $\spc{S}$;
\item $\spc{S}\sim \mathbb{S}^{m-1}$; that is, $\spc{S}$ is homotopy equivalent to $\mathbb{S}^{m-1}$.
\end{itemize}

Indeed, assume such $\spc{S}$ is constructed.
Then the suspension
$\spc{R}\z=\Susp\spc{S}$
is an $m$-dimensional homology manifold with a natural triangulation coming from $\spc{S}$.
By Lemma~\ref{lem:homomanifold-characterization},
$\spc{R}$ is a topological manifold.
According to generalized Poincar\'{e} conjecture,
$\spc{R}\simeq\mathbb{S}^m$;
that is
$\spc{R}$ is homeomorphic to $\mathbb{S}^m$.
Since $\Cone \spc{S}\simeq \spc{R}\backslash\{s\}$ where $s$ denotes a south pole of the suspension 
and $\EE^m\simeq \mathbb{S}^m\backslash\{p\}$
for any point $p\in \mathbb{S}^m$
we get 
\[\Cone \spc{S}\simeq\EE^m.\]

Let us construct $\spc{S}$.
Fix an $(m-2)$-dimensional homology sphere $\Sigma$ with a triangulation such that $\pi_1\Sigma\ne0$.
An example of that type exists for any $m\ge 5$, a proof is given in \cite{kervaire}.

Remove from $\Sigma$ one $(m-2)$-simplex.
Denote the obtained complex by $\Sigma'$.
Since $m\ge 5$, we have $\pi_1\Sigma=\pi_1\Sigma'$.

Consider the product $\Sigma'\times [0,1]$. 
Attach to it the cone over its boundary $\partial (\Sigma'\times [0,1])$.
Denote by $\spc{S}$ the obtained simplicial complex
and by $v$ the tip of the attached cone.

Note that $\spc{S}$ is homotopy equivalent to the spherical suspension over $\Sigma$ which is a simply connected homology sphere and hence is homotopy equivalent to $\mathbb{S}^{m-1}$.
  Hence  $\spc{S}\sim\mathbb{S}^{m-1}$.

The complement $\spc{S}\backslash\{v\}$ is homotopy equivalent to $\Sigma'$.
Therefore 
\[
\pi_1(\spc{S}\backslash\{v\})
=\pi_1\Sigma'
=\pi_1\Sigma\ne 0.
\]
That is, $\spc{S}$ satisfies the conditions above.

\parbf{Exercise~\ref{ex:concave-triangle}.}
By approximation,
it is sufficient to consider the case of poygonalal sides.

The latter case can be done by induction on number of sides.
The base case of triangle is evident.

To prove the step, apply Alexndrov's lemma (\ref{lem:alex}) 
together with the costruction in Exercise~\ref{ex:short-map}.

\parbf{Exercise~\ref{ex:bishop-sphere}.}
Unlike in the plane, sphere contains loons and triangles with concave sides.
To adapt the proof of Theorem~\ref{thm:bishop-plane}
to the spherical case, we need  the following claim; 
its proof is left to the reader.

\medskip

\parbf{Claim.}
\textit{Any closed curve shorter than $2\cdot\pi$ in the unit sphere lies in an open hemisphere.}

\medskip

From the claim it follows that such if a concave loon or triangle has perimeter smaller then $2\cdot\pi$ then it contains a closed hemisphere in its interior.

By the assumption, $\Theta$ does not contain a closed hemisphere. 
Therefore the argument in the proof of Theorem~\ref{thm:bishop-plane} shows that any triangle in $\tilde\Theta$ with perimeter less than $2\cdot\pi$ is spherically thin. 



\parbf{Exercise~\ref{ex:two-planes}.}
The space $K$ forms a cone over branched covering $\Sigma$ of $\mathbb{S}^3$ infinitely branching along two big circles.

If the planes are not orthogonal then the minimal distance between the circles is less than $\tfrac\pi2$.
Assume that this distance is realized by a geodesic $[\xi\zeta]$.
broken line made by four liftings of $[\xi\zeta]$ forms a closed geodesic in $\Sigma$. 
By Corollary~\ref{cor:closed-geod-cat}, $\Sigma$ is not $\Cat{}{1}$.
Therefore by Exercise~\ref{ex:cone+susp}, $K$ is not $\Cat{}{0}$.

If the planes are orthogonal then the corresponding big circles in $\SS^3$ are formes by subcomplexes of a flag triangulation of $\SS^3$ with all-right simplicies.
The branching cover is also flag.
It remains to apply the flag condition \ref{thm:flag}.

\parit{Comments.}
In \cite{charney-davis-93}, Ruth Charney and Michael Davis
gave a complete answer to the analogous question for three planes.
In particular they show that if a covering space of $\EE^4$
branching at three planes through one origin is $\Cat{}{0}$ then these all are complex planes for some complex structure on $\EE^4$.

\parbf{Exercise~\ref{ex:hemisphere}.}
Let $\alpha$ be a closed curve in $\SS^2$ of length $2\cdot\ell$.

Assume $\ell<\pi$.
Let $\check\alpha$ be a subarc of $\alpha$ of length $\ell$, with endpoints $p$ and $q$. 
Since $\dist{p}{q}{}\le\ell<\pi$, there is a unique geodesic $[pq]$ in $\SS^2$.  
Let $z$ be the midpoint of  $[pq]$.  
We claim that $\alpha$ lies in the open hemisphere centered at $z$.  
If not, $\alpha$ intersects the boundary  great circle in a point say $r$.
Without loss of generality we may assume that $r\in\check\alpha$. 
The arc $\check\alpha$ together with its reflection in $z$ form a closed curve of length $2\cdot \ell$ which passes through $r$ and its antipodal point $r'$.
Thus $\ell=\length \check\alpha\ge \dist{r}{r'}{}=\pi$, a contradiction.

\parit{A solution via Crofton formula.}
Let $\alpha$ be a closed curve in  $\SS^2$ of length $\le 2\cdot\pi$.  We wish to prove that it's contained in a hemisphere in $\SS^2$.
By approximation it's clearly enough to prove this for  smooth curves of length $< 2\cdot\pi$ with transverse self-intersections. 

Given $v\in \SS^2$, denote by $v^\perp$ the equator in $\SS^2$ with the pole at $v$.
Further, $\# X$ will denote the number of points in the set $X$.

Obviously,  if $\#(\alpha\cap v^\perp) =0$ then $\alpha$ is contained in one of the hemispheres determined by $v^\perp$. 
Note that $\#(\alpha\cap v^\perp)$ is even for almost all $v$.

Therefore, if $\alpha$ does not lie in a hemisphere then
$\#(\alpha\cap v^\perp) \ge 2$ for almost all $v\in\SS^2$.  

By Crofton's formula we have that
\begin{align*}
\length(\alpha)
&=\frac 1 4\cdot \int\limits_{\SS^2}\#(\alpha\cap v^\perp)\cdot \d_v\area\ge
\\
&\ge2\cdot\pi.
\end{align*}


\parbf{Exercise~\ref{ex:convex+saddle+broken=>PL}.}
Consider the surface $S$ 
formed by closure of the remaining part of the boundary.
Note that the boundary $\partial S$ of $S$ is formed by some number of broken lines.

Assume $S$ is not a piecewise linear.
Show that there is a line segment $[pq]$ in $\EE^3$ tangent to $S$ at $p$ which has no common points with $S$ except $p$.

Since $S$ is locally concave,
there is a local inner supporting plane $\Pi$ at $p$ which contains the segment $[pq]$.

Note that $\Pi\cap S$ contains a segment $[xy]$ passing through $p$ with the ends at $\partial S$.

Show that there is a line segmet of $\partial S$ starting at $x$ or $y$ which lies in $\Pi$.

From the latter statement and local convexity of $S$, 
all points of $[pq]$ sufficiently close to $p$ lie in $S$,
a contradiction.



\parbf{Exercise~\ref{ex:CAT=>two-convex}.}
Show that if $K$ is not two-convex then there is a plane triangle $\triangle$ which sides lying completely in $K$, 
but its interior contains the points of the complement $\EE^m\backslash K$.

It remains to note that $\triangle$ is not thin in $K$.

\parbf{Exercise~\ref{ex:two-convex-not-a-CAT}.}
Clearly the set $W$ is two-convex.
Therefore $K$ is two-convex as an interesection of two convex sets.

The set $K$ on big scale looks like the complement of the half-spaces of two 3-dimnsional spaces in $\EE^4$.
More precisely the limit, say $K'$ of $\tfrac1n\cdot K$ is 
the completion of the complement of the half-spaces of two 3-dimensional spaces in $\EE^4$ equipped with intrinsic metric.

In particular $K'$ is a cone over space $\Sigma$ which 
can be obtained as the completion of sphere $\SS^3$ with removed two 2-dimensional hemispheres, say $H_1$ and $H_2$.
The intersection of these hemispheres is tipically formed by a geodesic segment, say $[\xi\zeta]$.

Consider the hemispheres $H_1$ and $H_2$ so that $[\xi\zeta]$ is orthogonal to the boundary spheres and its length less than $\tfrac\pi2$.
Then the projection of a closed geodesic in $\Sigma$ to $\SS^3$
is formed by a joint of 4 copies of $[\xi\zeta]$.
In particular there is a closed geodesic in $\Sigma$ shorter than $2\cdot\pi$.
Hence $\Sigma$ is not $\Cat{}{1}$ 
and therefore $K'$ and consequently $K$ are not $\Cat{}{0}$.


\parbf{Exercise~\ref{ex:branching-cover}.}
Consider a $\eps$-neighborhood $A$ of the geodesic.
Note that $A_\eps$ forms a convex set.
By Reshetnyak gluing theorem the doubling $\spc{W}_\eps$ of $\spc{U}$ along $A_\eps$ is a $\Cat{}{0}$ space.

Consider the other space $\spc{W}'_\eps$ obtained by double covering of $\spc{U}\backslash A_\eps$ and gluing back $A_\eps$.

Note that $\spc{W}'_\eps$ is locally isometric to $\spc{W}_\eps$. 
That is for any point $p'\in\spc{W}'_\eps$ there is a point $p\in\spc{W}_\eps$ such that $\delta$-neighborhood of $p$ is isometric to $\delta$-neighborhood of $p$ for all small $\delta>0$.

Further note that $\spc{W}'_\eps$ is simply connected.
By globalization theorem, $\spc{W}'_\eps$ s $\Cat{}{0}$.

It remains to note that $\~{\spc{U}}$ can be obtained as a limit of $\spc{W}'_\eps$ as $\eps\to 0$ and apply Proposition~\ref{prop:cat-limit}.


