\chapter*{Semisolutions}
\parbf{Exercise~\ref{ex:two-components-of-M4}.}
Let $\spc{X}$ be a 4-point metric space.

Fix a tetrahedron $\triangle$ in $\RR^3$.
The verices of tetrahedron, 
say $x_0$, $x_1$, $x_2$, $x_3$ can be identified with the points of $\spc{X}$.

Note that there is unique qadratic form $W$ on $\RR^3$
such that 
\[W(x_i-x_j)=\dist[2]{x_i}{x_j}{\spc{X}}\]
for all $i$ and $j$;
here $\dist{x_i}{x_j}{\spc{X}}$ denotes the distance beween $x_i$ and $x_j$ in $\spc{X}$.

By triangle inequality $W(v)\ge 0$ 
for any vector $v$ parallel to one of the faces of $\triangle$.

Note that $\spc{X}$ is isometric to a 4-point subset in the Euclidean space
if and ony if $W(v)\ge 0$ for any vector $v$ in $\RR^3$.

\begin{wrapfigure}{r}{52mm}
\begin{lpic}[t(-5mm),b(3mm),r(0mm),l(0mm)]{pics/quad(1)}
\end{lpic}
\end{wrapfigure}

Therefore if $\spc{X}$ is not of type $\mathcal{E}_4$ then $W(v)<0$ for some vector $v$.
From above, the vector $v$ must be transveral to each of 4 faces of $\triangle$.
Therefore if we project $\triangle$ along $v$ to a plane transveral to $v$ we see one of the folwing two pictures.

It is easy to see that the combinatorics of the picture does not depend on the choice of $v$. 
Hence $\mathcal{M}_4\backslash\mathcal{E}_4$ not connected. 

It remains to show that if the combinatorics of the pictures for two spaces is the same then one can continuousely deform one space into the other.
This can be easily done deforming $W$ and apply permutation of $x_0$, $x_1$, $x_2$, $x_3$ if necessay.





\parbf{Exercise~\ref{ex:convex-set}.}
The simplest proof we know require proof tangent cones for Alexandrov spaces.


\parbf{Exercise~\ref{ex:product-cone}.}
\parbf{Exercise~\ref{ex:no-geod}.}
Given a metric graph $\Gamma$ define $P_k(\Gamma)$ as follows. Let $\Gamma^b$ be the barycentric subdivision of $\Gamma$ with the natural metric. For any two adjacent vertices $p,q\in\Gamma^b$ substitute the edge $[pq]$ by  a countable collection of intervals $\{I_i\}_{i\ge 1}$ of length $\dist{p}{q}{}+\frac{\dist{p}{q}{}}{2^ki}$ where one end of each $I_i$ is glued to $p$ and the other to $q$. Note that the resulting space $P_k(\Gamma)$ is again a metric graph  with an inner metric. 

Let $\spc{X}_0=[0,1]$ and define $\spc{X}_k$ for $k\ge 1$ inductively as $\spc{X}_k=P_k(\spc{X}_{k-1})$.

Let $\spc{Y}_k$ be the set of vertices of $\spc{X}_k$ with the induced metric. By construction the inclusion $\spc{Y}_k\subset \spc{Y}_{k+1}$ is distance preserving.

Let $\spc{Y}_\infty=\cup_{k\ge 1}\spc{Y}_k$ with the obvious metric and let $\spc{Y}=\bar {\spc{Y}}_\infty$ be its metric completion. Then $\spc{Y}$ is a length space since it satisfies the almost midpoint property. But it is not hard to see that no two distinct points in $\spc{Y}$ can be connected by a shortest geodesic. \qeds

AN OTHER SOLUTION from here http://mathoverflow.net/q/15720
if we want it, I can ask fedya.???

Well, the unit ball in $c_0$ is almost what you want (there is no unique shortest curve between points). 
All we need now is to enhance "bypasses" and to give disadvantage to "straight lines". 
This can easily be done by taking the distance element to be \[(2+\sum_n 2^{-n}x_n)^{-1}\|dx\|_\infty,\] 
which is never less than the usual distance element in $c_0$ and never greater than 3 times it in the unit ball. Now, if we have any continuous finite length curve $x(t)$ from $y$ to $z$ parametrized by the arclength, we can easily shorten it by replacing the $m$-th position by the maximum of the actual value of $x_m(t)$ and 
\[y_m+t(z_m-y_m)/d+\frac 12 \min(t,d-t),\] 
where $d$ is the length of $x(t)$, which will work if $m$ is large enough since $\max_t|x_m(t)|\to 0$ as $m\to\infty$ and both functions change slower than the distance along the original curve.

\parbf{Exercise~\ref{exercise from BH}.}
The following example is taken from the book of Bridson and Haefliger \cite{BH}.

Consider the following subset of $\R^2$:

\[
\spc{X}
=
\bigl((0,1]\times\{0,1\}\bigr)
\cup
\bigl(\bigcup_{n\ge 1}\{1/n\}\times[0,1]\bigr)
\]
Consider the induced inner metric on $\spc{X}$. It's obviously locally compact and geodesic.
However, it's immediate to check that its metric completion \[\bar{\spc{X}}
=
\bigl([0,1]\times\{0,1\}\bigr)
\cup
\bigl(\bigcup_{n\ge 1}\{1/n\}\times[0,1]\bigr)\] 
is neither. \qeds 

\parbf{Exercise~\ref{ex:adjacent-angles}.}
Assume contrary; that is
\[\mangle\hinge pxz+\mangle\hinge pyz< \pi\]
By triangle inequality for angles (\ref{claim:angle-3angle-inq})
we have 
\[\mangle\hinge pxy<\pi\]
The latter cotradicts the triangle inequality for the triangle $[\bar x p \bar y]$,
where the points
$\bar x\in ]px]$ and $\bar y\in ]py]$
are sufficiently close to $p$.

%\parbf{Exercise~\ref{ex:geod-CBB}.}
\parbf{Exercise~\ref{ex:geod-CBA}.}
Fix sufficiently small $\eps>0$.

Consider a sequence of directions $\xi_n$ of geodesics $[pq_n]$.
Since the geodesics are extandable,
we can assume that the distances $\dist{p}{q_n}{\spc{U}}$ are equal.

Since $\spc{U}$ is proper,
the sequence $q_n$ has partial limit, say $q$.
It remains to note that the direction $\xi$ of $[pq]$ is the limit of directions $\xi_n$,
assuming the latter is defined.


\parbf{Exercise~\ref{exr-crofton}.}
Let $\alpha$ be a closed curve in  $\SS^2$ of length $\le 2\cdot\pi$.  We wish to prove that it's contained in a hemisphere in $\SS^2$.
By approximation it's clearly enough to prove this for  smooth curves of length $< 2\cdot\pi$ with transverse self-intersections. 

Given $v\in \SS^2$, denote by $v^\perp$ the equator in $\SS^2$ with the pole at $v$.
Further, $\# X$ will denote the number of points in the set $X$.

Obviously,  if $\#(\alpha\cap v^\perp) =0$ then $\alpha$ is contained in one of the hemispheres determined by $v^\perp$. 
Note that $\#(\alpha\cap v^\perp)$ is even for almost all $v$.

Therefore, if $\alpha$ does not lie in a hemisphere then
$\#(\alpha\cap v^\perp) \ge 2$ for almost all $v\in\SS^2$.  

By Crofton's formula we have that
\begin{align*}
\length(\alpha)
&=\frac 1 4\cdot \int\limits_{\SS^2}\#(\alpha\cap v^\perp)\cdot \d_v\area\ge
\\
&\ge2\cdot\pi.
\end{align*}


\parbf{Exercise~\ref{ex:cone+susp}.}
\parbf{Exercise~\ref{ex:product}.}

\begin{wrapfigure}{r}{27mm}
\begin{lpic}[%draft,
t(-5mm),b(0mm),r(0mm),l(0mm)]{sbapics/resh(0.3)}
\lbl[t]{47,9;$\~p$}
\lbl[l]{82,88;$\~x$}
\lbl[r]{2,74;$\~y$}
\lbl[b]{37,82;$\~z$}
\lbl[tr]{33,62;$\~z_x$}
\lbl[tl]{46,63;$\~z_y$}
\end{lpic}
\end{wrapfigure}
\parbf{Exercise~\ref{ex:short-map}.}
By Alexandrov's lemma (\ref{lem:alex}), 
there are nonoverlapping triangles 
$\trig{\~p}{\~x}{\~z_y}\iso\trig {\dot p}{\dot x}{\dot z}$ 
and 
$\trig{\~p}{\~y}{\~z_x}\iso\trig {\dot p}{\dot y}{\dot z}$
 inside triangle $\trig{\~p}{\~x}{\~y}$.

Connect points in each pair
$(\~z,\~z_x)$, 
$(\~z_x,\~z_y)$ 
and $(\~z_y,\~z)$ 
with arcs of circles centered at 
$\~y$, $\~p$, and $\~x$ respectively. 
Define $F$ as follows.
\begin{itemize}
\item Map  $\Conv\trig{\~p}{\~x}{\~z_y}$ isometrically onto  $\Conv\trig {\dot p}{\dot x}{\dot y}$;
similarly map $\Conv \trig{\~p}{\~y}{\~z_x}$ onto $\Conv \trig {\dot p}{\dot y}{\dot z}$.
\end{itemize}

\begin{itemize}
\item If $x$ is in one of the three circular sectors, say at distance $r$ from center of the circle, let $F(x)$ be the point on the corresponding segment 
$[p z]$, 
$[x z]$ 
or $[y z]$ whose distance from the lefthand endpoint of the segment is $r$.
\item Finally, if $x$ lies in the remaining curvilinear triangle $\~z \~z_x \~z_y$, 
set $F(x) = z$. 
\end{itemize}
By construction, $F$ satisfies the conditions. 

\parbf{Exercise~\ref{ex:convex-balls}.}
It is sufficient to show that 
\[\dist{p}{z}{}\le \max\{\dist{p}{x}{},\dist{p}{y}{}\}\]
for any $p$ and any $z\in[xy]$.

The statement follows since the triangle $\trig pxy$ is thin
and the above condition holds in the Euclidean plane.

In $\Cat{}{1}$ case, the proof is the same, but we need to assume in addition that 
$\max\{\dist{p}{x}{},\dist{p}{y}{}\}\le \tfrac\pi2$

\parbf{Exercise~\ref{ex:closest-point}.}
Since $\spc{U}$ is proper, the set $K\cap \cBall[p,R]$ is compact for any $R<\infty$.
Hence the existance of $p^*$ follows.

Assume $p^*$ is not uniquely defined;
that is,  two distinct points in $K$, say $x$ and $y$, minimize the distance from $p$.
Sicne $K$ is convex the midpoint $z$ of $[xy]$ lies in $K$.

Note that $\dist{p}{z}{}<\dist{p}{x}{}=\dist{p}{y}{}$, a contradiction.

It remains to show that the map $p\mapsto p^*$ is short;
that is 
\[\dist{p}{q}{}\ge \dist{p^*}{q^*}{}.\eqlbl{eq:short-p-p}\]

Note that 
\[\mangle\hinge{p^*}{p}{q^*}\ge \tfrac\pi2
\quad
\text{and}
\quad
\mangle\hinge{q^*}{p}{p^*}\ge \tfrac\pi2,\] 
if the lefthand sides are defined. 


Construct the model triangles 
$\trig{\~p}{\~p^*}{\~q^*}$ and $\trig{\~p}{\~q}{\~q^*}$
of $\trig{p}{p^*}{q^*}$ and $\trig{p}{q}{q^*}$ so that 
the points $\~p^*$ and $\~q$ lie on the opposite sides from $[{\~p}{\~q^*}]$.

By comparison and triangle inequality for angles, we get \ref{claim:angle-3angle-inq}
\[\mangle\hinge{\~p^*}{\~p}{\~q^*}\ge \mangle\hinge{p^*}{p}{q^*}\ge \tfrac\pi2
\quad
\text{and}
\quad
\mangle\hinge{\~q^*}{\~q}{\~p^*}\ge \mangle\hinge{q^*}{q}{p^*}\ge \tfrac\pi2
\]
assuming that the left hand sides are defined. 
Hence 
\[\dist{\~p}{\~q}{}\ge \dist{\~p^*}{\~q^*}{}.\]
The latter is equivalent to \ref{eq:short-p-p}.

\parbf{Exercise~\ref{ex:two-rays}.}
Consider the angle $A$ in the plane of measure $\pi-\alpha$.
Note that $A$ is $\Cat{}{0}$.
Therefore by Reshetnyak gluing theorem \ref{thm:gluing},
by gluing a side of $A$ to $\gamma_1$ in $\spc{U}$ we obtain a $\Cat{}{0}$ space, say $\spc{U}'$.

Note that $\gamma_2$ together with the other side of $A$ forms a both sides infinite geodesic, say $\gamma$ in $\spc{U}'$.
In particular, $\gamma$ is a convex set isometric to $\RR$.

Glue a half-plane along its boundary to $\gamma$.
By Reshetnyak gluing theorem \ref{thm:gluing} the obtained space is $\Cat{}{0}$.

It remains to note that this space can be obtained directly by gluing $\spc{U}$ to with $Q$ along $\gamma_1$ and $\gamma_2$.


\parbf{Exercise~\ref{ex:branching-cover}.}
Consider a $\eps$-neighborhood $A$ of the geodesic.
Note that $A_\eps$ forms a convex set.
By Reshetnyak gluing theorem the doubling $\spc{W}_\eps$ of $\spc{U}$ in $A_\eps$ is a $\Cat{}{0}$ space.

Consider the other space $\spc{W}'_\eps$ obtained by double covering of $\spc{U}\backslash A_\eps$ and gluing back $A_\eps$.

Note that $\spc{W}'_\eps$ is locally isometric to $\spc{W}_\eps$. 
That is for any point $p'\in\spc{W}'_\eps$ there is a point $p\in\spc{W}_\eps$ such that $\delta$-neighborhood of $p$ is isometric to $\delta$-neighborhood of $p$ for all small $\delta>0$.

Further note that $\spc{W}'_\eps$ is simply connected.
By globalization theorem, $\spc{W}'_\eps$ s $\Cat{}{0}$.

It remains to note that $\~{\spc{U}}$ can be obtained as a limit of $\spc{W}'_\eps$ as $\eps\to 0$.

%%%???HERE WE USE GLOBALIZATION AND LIMIT!!!


\parbf{Exercise~\ref{ex:supporting-planes}.}
By approximation, it is sufficient to consider the case when 
$A$ and $B$ have smooth boundary.
In the latter case the choice of $\dot A$ and $\dot B$ is unique;
and a variation argument shows that it satisfies the condition.

\begin{wrapfigure}{r}{44mm}
\begin{lpic}[t(0mm),b(-00mm),r(0mm),l(0mm)]{pics/compact-walls(1)}
\lbl[tr]{21,7.5;$0$}
\lbl[tr]{10,20;$A^i$}
\lbl[tr]{34,12;$A^j$}
\end{lpic}
\end{wrapfigure}

\parbf{Exercise~\ref{ex:compact-walls}.}
Fix $r>0$ and $R<\infty$ so that 
\[\oBall(0,r)\subset A^i\subset \oBall(0,R)\]
for each wall $A^i$.

It remains to note that all the intersections of  walls  have $\eps$-wide for
\[\eps=2\cdot \arcsin\tfrac rR.\]

\parbf{Exercise~\ref{ex:centrally-simmetric-walls}.}
Note that there is $R<\infty$
such that if $X$ is an intersection of arbitrary number of walls
and then there is an isometry of $X$ 
which moves any point $p\in X$ to a point in $\oBall(0,R)$.

The rest of the proof goes along the same lines as Exercise~\ref{ex:compact-walls}.

\parbf{Exercise~\ref{ex:null-homotopic}.}
Note that the existance of null-homotopy is equivalent to the following.
There are two one-parameter family of paths $\alpha_\tau$ and $\beta_\tau$, $\tau\in[0,1]$ 
such that 
\begin{itemize}
\item $\length\alpha_\tau$, $\length\beta_\tau<\pi$ for any $\tau$.
\item $\alpha_\tau(0)=\beta_\tau(0)$ and $\alpha_\tau(1)=\beta_\tau(1)$ for any $\tau$.
\item $\alpha_0(t)=\beta_0(t)$ for any $t$.
\item $\alpha_1(t)=\alpha(t)$ and $\beta_1(t)=\beta(t)$ for any $t$.
\end{itemize}

By Corollary~\ref{cor:discrete-paths},
the construction in Corollary~\ref{cor:path-geod} produces the same result for $\alpha_\tau$ and $\beta_\tau$.
Hence the result follows.


\parbf{``If'' part of Theorem~\ref{thm:PL-CAT}.}
\parbf{Exercise~\ref{ex:unique-geod=CAT}.}

\parbf{Exercise~\ref{ex:baricenric-flag}.}
Checking the flag condition is straightforward once you know the following description of the barycentric subdevision.

Each vertexes $v$ of the barycentric subdevision 
corresponds to a simplex $\triangle_v$ of the original triangulation.
A set vertices form a simplex in the subdevision, 
if it can be ordered, say as $v_1,\dots,v_\kay$,
such that the correspoding simplexes form a nested sequence
\[\triangle_{v_1}\subset\dots\subset\triangle_{v_\kay}.\]




\parbf{Exercise~\ref{ex:flag>=pi/2}.}
The proof goes along the same lines as the proof of Flag condition~\ref{thm:flag}.
The only difference that geodesic spends time at least $\pi$ the star of each vertex.

\parbf{Exercise~\ref{ex:short-retracts}.}
Start with a copy of $\spc{U}$ for each vertex of $\Gamma$
and glue them recuresevely using the following operation and stages.

At the stage $n$ glue only spaces along $A^n$.

In one step of the stage, choose two connected components of partially glued copies and glue them along the corresponding $A^n$'s.
Apply Reshetnyak gluing theorem to show that each connected component after the stage is $\Cat{}{0}$ assuming that all connected component were $\Cat{}{0}$ before the step.


\parbf{Exercise~\ref{ex:tree}.}
The space $\spc{T}_n$ has natural cone structure with the vertex formed by completely degenerate tree --- all its edges have zero length.
Note that the space $\Sigma$, 
over which the cone is taken comes naturally with triangulation 
with all-right spherical simplicies.

It remains to check that the complex is flag 
and apply Exercise~\ref{ex:cone+susp}.

\parbf{Exercise~\ref{ex:flag-aspherical}.}
\parbf{Exercise~\ref{ex:example-pi_infty-new}.}
Solution goes along the same lines as the proof of Lemma~\ref{lem:example-pi_infty}.
The only difference, the set $G$ formed by squares with the vertexes at the  midpoints of some edges of the cubical analog. 

\parbf{Exercise~\ref{ex:funny-S}.}
In the proof we apply the following lemma from \cite{edwards}; 
it follows from the disjoint discs property.


\begin{thm}{Lemma}\label{lem:homomanifold-characterization}
Let $\spc{S}$ be a simplicial complex which 
is an $m$-dimensional homology manifold for some $m\ge 5$.
Assume all the vertices of
$\spc{S}$ have simply connected links.
Then $\spc{S}$ is a topological manifold.
\end{thm}


It is sufficient to construct a simplicial complex $\spc{S}$
such that 
\begin{itemize}
\item $\spc{S}$ is a closed $(m-1)$-dimensional homology manifold;
\item $\pi_1(\spc{S}\backslash\{v\})\ne0$ for some vertex $v$ in $\spc{S}$;
\item $\spc{S}\sim \mathbb{S}^{m-1}$; that is, $\spc{S}$ is homotopy equivalent to $\mathbb{S}^{m-1}$.
\end{itemize}

Indeed, assume such $\spc{S}$ is constructed.
Then the suspension
$\spc{R}\z=\Susp\spc{S}$
is an $m$-dimensional homology manifold with a natural triangulation coming from $\spc{S}$.
By Lemma~\ref{lem:homomanifold-characterization},
$\spc{R}$ is a topological manifold.
According to generalized Poincar\'{e} conjecture,
$\spc{R}\simeq\mathbb{S}^m$;
that is
$\spc{R}$ is homeomorphic to $\mathbb{S}^m$.
Since $\Cone \spc{S}\simeq \spc{R}\backslash\{s\}$ where $s$ denotes a south pole of the suspension 
and $\EE^m\simeq \mathbb{S}^m\backslash\{p\}$
for any point $p\in \mathbb{S}^m$
we get 
\[\Cone \spc{S}\simeq\EE^m.\]

Let us construct $\spc{S}$.
Fix an $(m-2)$-dimensional homology sphere $\Sigma$ with a triangulation such that $\pi_1\Sigma\ne0$.
According to \cite{kervaire} %it is a good readable paper, but I am sure the existance follows from sometheng written before
an example of that type exists for any $m\ge 5$.

Remove from $\Sigma$ one $(m-2)$-simplex.
Denote the obtained complex by $\Sigma'$.
Since $m\ge 5$, we have $\pi_1\Sigma=\pi_1\Sigma'$.

Consider the product $\Sigma'\times [0,1]$. 
Attach to it the cone over its boundary $\partial (\Sigma'\times [0,1])$.
Denote by $\spc{S}$ the obtained simplicial complex
and by $v$ the tip of the attached cone.

Note that $\spc{S}$ is homotopy equivalent to the spherical suspension over $\Sigma$ which is a simply connected homology sphere and hence is homotopy equivalent to $\mathbb{S}^{m-1}$.
  Hence  $\spc{S}\sim\mathbb{S}^{m-1}$.

The complement $\spc{S}\backslash\{v\}$ is homotopy equivalent to $\Sigma'$.
Therefore 
\[
\pi_1(\spc{S}\backslash\{v\})
=\pi_1\Sigma'
=\pi_1\Sigma\ne 0.
\]
That is, $\spc{S}$ satisfies the conditions above.

\parbf{Exercise~\ref{ex:concave-triangle}.}
By approximation,
it is sufficient to consider the case of poygonalal sides.

The latter case can be done by induction on number of sides.
The base case of triangle is evident.

To prove the step, apply Alexndrov's lemma (\ref{lem:alex}) 
together with the costruction in Exercise~\ref{ex:short-map}.

\parbf{Exercise~\ref{ex:bishop-sphere}.}


%%%???SHOULD WE DO FINITE COVER TO AVOID NONPROPER SPACES???
\parbf{Exercise~\ref{ex:two-planes}.}
The space $K$ forms a cone over branched covering of $\mathbb{S}^3$ infinitely branching along two big circles.

If the planes are not orthogonal then the minimal distance between the circles is less than $\tfrac\pi2$.
Assume that this distance is realized by a geodesic $[\xi\zeta]$.
broken line made by four liftings of $[\xi\zeta]$ forms a closed geodesic in $K$. 
By ???, $K$ is not $\Cat{}{0}$.

If the planes are orthogonal then the corresponding big circles in $\SS^3$ are formes by subcomplexes of a flag triangulation of $\SS^3$ with all-right simplicies.
The branching cover is also flag.
It remains to apply the flag condition \ref{thm:flag}.

\parit{Comments.}
In \cite{charney-davis-93},  Charney and Davis
gave a complete answer to the analogous question for three planes.
In particular they show that if a covering space of $\EE^4$
branching at three planes through one origin is $\Cat{}{0}$ then these all are complex planes for some complex structure on $\EE^4$.



\parbf{Exercise~\ref{ex:convex+saddle+broken=>PL}.}
Consider the surface $S$ 
formed by closure of the remaining part of the boundary.
Note that the boundary $\partial S$ of $S$ is formed by some number of broken lines.

Assume $S$ is not a piecewise linear.
Show that there is a line segment $[pq]$ in $\EE^3$ tangent to $S$ at $p$ which has no common points with $S$ except $p$.

Since $S$ is locally concave,
there is a local inner supporting plane $\Pi$ at $p$ which contains the segment $[pq]$.

Note that $\Pi\cap S$ contains a segment $[xy]$ passing through $p$ with the ends at $\partial S$.

Show that there is a line segmet of $\partial S$ starting at $x$ or $y$ which lies in $\Pi$.

From the latter statement and local convexity of $S$, 
all points of $[pq]$ sufficiently close to $p$ lie in $S$,
a contradiction.



\parbf{Exercise~\ref{ex:CAT=>two-convex}.}
Show that if $K$ is not two-convex then there is a plane triangle $\triangle$ which sides lying completely in $K$, 
but its interior contains the points of the complement $\EE^m\backslash K$.

It remains to note that $\triangle$ is not thin in $K$.

\parbf{Exercise~\ref{ex:two-convex-not-a-CAT}.}
Clearly the set $W$ is two-convex.
Therefore $K$ is two-convex as an interesection of two convex sets.

The set $K$ on big scale looks like the complement of the half-spaces of two 3-dimnsional spaces in $\EE^4$.
More precisely the limit, say $K'$ of $\tfrac1n\cdot K$ is 
the completion of the complement of the half-spaces of two 3-dimensional spaces in $\EE^4$ equipped with intrinsic metric.

In particular $K'$ is a cone over space $\Sigma$ which 
can be obtained as the completion of sphere $\SS^3$ with removed two 2-dimensional hemispheres, say $H_1$ and $H_2$.
The intersection of these hemispheres is tipically formed by a geodesic segment, say $[\xi\zeta]$.

Consider the hemispheres $H_1$ and $H_2$ so that $[\xi\zeta]$ is orthogonal to the boundary spheres and its length less than $\tfrac\pi2$.
Then the projection of a closed geodesic in $\Sigma$ to $\SS^3$
is formed by a joint of 4 copies of $[\xi\zeta]$.
In particular there is a closed geodesic in $\Sigma$ shorter than $2\cdot\pi$.
Hence $\Sigma$ is not $\Cat{}{1}$ 
and therefore $K'$ and consequently $K$ are not $\Cat{}{0}$.





