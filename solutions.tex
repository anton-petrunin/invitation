\chapter*{Semisolutions}
\parbf{Exercise~\ref{ex:two-components-of-M4}.}
\parbf{Exercise~\ref{ex:convex-set}.}
\parbf{Exercise~\ref{ex:product-cone}.}
\parbf{Exercise~\ref{ex:no-geod}.}
Given a metric graph $\Gamma$ define $P_k(\Gamma)$ as follows. Let $\Gamma^b$ be the barycentric subdivision of $\Gamma$ with the natural metric. For any two adjacent vertices $p,q\in\Gamma^b$ substitute the edge $[pq]$ by  a countable collection of intervals $\{I_i\}_{i\ge 1}$ of length $\dist{p}{q}{}+\frac{\dist{p}{q}{}}{2^ki}$ where one end of each $I_i$ is glued to $p$ and the other to $q$. Note that the resulting space $P_k(\Gamma)$ is again a metric graph  with an inner metric. 

Let $\spc{X}_0=[0,1]$ and define $\spc{X}_k$ for $k\ge 1$ inductively as $\spc{X}_k=P_k(\spc{X}_{k-1})$.

Let $\spc{Y}_k$ be the set of vertices of $\spc{X}_k$ with the induced metric. By construction the inclusion $\spc{Y}_k\subset \spc{Y}_{k+1}$ is distance preserving.

Let $\spc{Y}_\infty=\cup_{k\ge 1}\spc{Y}_k$ with the obvious metric and let $\spc{Y}=\bar {\spc{Y}}_\infty$ be its metric completion. Then $\spc{Y}$ is a length space since it satisfies the almost midpoint property. But it is not hard to see that no two distinct points in $\spc{Y}$ can be connected by a shortest geodesic. \qeds

AN OTHER SOLUTION from here http://mathoverflow.net/q/15720
if we want it, I can ask fedya.???

Well, the unit ball in $c_0$ is almost what you want (there is no unique shortest curve between points). 
All we need now is to enhance "bypasses" and to give disadvantage to "straight lines". 
This can easily be done by taking the distance element to be \[(2+\sum_n 2^{-n}x_n)^{-1}\|dx\|_\infty,\] 
which is never less than the usual distance element in $c_0$ and never greater than 3 times it in the unit ball. Now, if we have any continuous finite length curve $x(t)$ from $y$ to $z$ parametrized by the arclength, we can easily shorten it by replacing the $m$-th position by the maximum of the actual value of $x_m(t)$ and 
\[y_m+t(z_m-y_m)/d+\frac 12 \min(t,d-t),\] 
where $d$ is the length of $x(t)$, which will work if $m$ is large enough since $\max_t|x_m(t)|\to 0$ as $m\to\infty$ and both functions change slower than the distance along the original curve.

\parbf{Exercise~\ref{exercise from BH}.}The following example is from~\cite{BH}.

Consider the following subset of $\R^2$:

\[
\spc{X}=(0,1]\times\{0\}\cup (0,1]\times\{1\}\cup_{n\ge 1}\{1/n\}\times[0,1]
\]
Consider the induced inner metric on $\spc{X}$. It's obviously locally compact and geodesic.
However, it's immediate to check that its metric completion $\bar{\spc{X}}=[0,1]\times\{0\}\cup [0,1]\times\{1\}\cup_{n\ge 1}\{1/n\}\times[0,1]$ is neither. \qeds 

\parbf{Exercise~\ref{ex:adjacent-angles}.}
\parbf{Exercise~\ref{ex:geod-CBB}.}
\parbf{Exercise~\ref{ex:geod-CBA}.}
\parbf{Exercise~\ref{exr-crofton}.}
Let $\alpha$ be a closed curve in  $\SS^2$ of length $\le 2\pi$.  We wish to prove that it's contained in a hemisphere in $\SS^2$.
By approximation it's clearly enough to prove this for  smooth curves of length $< 2\pi$ with transverse self-intersections. Furthermore, by changing such  a curve near its self-intersection points  it can be approximated with respect to the Hausdorff distance by simple closed curves. 
Thus, without loss of generality we can assume that $\alpha$ is a simple closed curve of length $<2\pi$.
By Crofton's formula we have that
\[
L(\alpha)=\frac 1 4 \int _{\SS^2}\#(\alpha\cap v^\perp) \d_v\vol_2
\]

Obviously,  if $\#(\alpha\cap v^\perp) =0$ then $\alpha$ is contained in one of the hemispheres determined by $v^\perp$. By the intermediate value theorem the same holds true if $\#(\alpha\cap v^\perp) =1$.
Suppose  $\#(\alpha\cap v^\perp) \ge 2$ for all $v\in\SS^2$. Then Crofton's formula implies that
$L(\alpha)\ge \frac 1 4 \int_{\SS^2}2=2\pi$. \qeds

\parbf{Exercise~\ref{ex:cone+susp}.}
\parbf{Exercise~\ref{ex:product}.}
\parbf{Exercise~\ref{ex:short-map}.}
\parbf{Exercise~\ref{ex:ruled-surface}.}
\parbf{Exercise~\ref{ex:convex-balls}.}
\parbf{Exercise~\ref{ex:closest-point}.}
\parbf{Exercise~\ref{ex:two-rays}.}
\parbf{Exercise~\ref{ex:branching-cover}.}
\parbf{Exercise~\ref{ex:supporting-planes}.}
\parbf{Exercise~\ref{ex:compact-walls}.}
\parbf{Exercise~\ref{ex:centrally-simmetric-walls}.}
\parbf{Exercise~\ref{ex:null-homotopic}.}
\parbf{``If'' part of Theorem~\ref{thm:PL-CAT}.}
\parbf{Exercise~\ref{ex:unique-geod=CAT}.}
\parbf{Exercise~\ref{ex:baricenric-flag}.}
\parbf{Exercise~\ref{ex:flag>=pi/2}.}
\parbf{Exercise~\ref{ex:short-retracts}.}
\parbf{Exercise~\ref{ex:tree}.}
\parbf{Exercise~\ref{ex:flag-aspherical}.}
\parbf{Exercise~\ref{ex:example-pi_infty-new}.}
\parbf{Exercise~\ref{ex:funny-S}.}
In the proof we apply the following lemma from \cite{edwards}; 
it follows from the disjoint discs property.


\begin{thm}{Lemma}\label{lem:homomanifold-characterization}
Let $\spc{S}$ be a simplicial complex which 
is an $m$-dimensional homology manifold for some $m\ge 5$.
Assume all the vertices of
$\spc{S}$ have simply connected links.
Then $\spc{S}$ is a topological manifold.
\end{thm}


It is sufficient to construct a simplicial complex $\spc{S}$
such that 
\begin{itemize}
\item $\spc{S}$ is a closed $(m-1)$-dimensional homology manifold;
\item $\pi_1(\spc{S}\backslash\{v\})\ne0$ for some vertex $v$ in $\spc{S}$;
\item $\spc{S}\sim \mathbb{S}^{m-1}$; that is, $\spc{S}$ is homotopy equivalent to $\mathbb{S}^{m-1}$.
\end{itemize}

Indeed, assume such $\spc{S}$ is constructed.
Then the suspension
$\spc{R}\z=\Susp\spc{S}$
is an $m$-dimensional homology manifold with a natural triangulation coming from $\spc{S}$.
By Lemma~\ref{lem:homomanifold-characterization},
$\spc{R}$ is a topological manifold.
According to generalized Poincar\'{e} conjecture,
$\spc{R}\simeq\mathbb{S}^m$;
that is
$\spc{R}$ is homeomorphic to $\mathbb{S}^m$.
Since $\Cone \spc{S}\simeq \spc{R}\backslash\{s\}$ where $s$ denotes a south pole of the suspension 
and $\EE^m\simeq \mathbb{S}^m\backslash\{p\}$
for any point $p\in \mathbb{S}^m$
we get 
\[\Cone \spc{S}\simeq\EE^m.\]

Let us construct $\spc{S}$.
Fix an $(m-2)$-dimensional homology sphere $\Sigma$ with a triangulation such that $\pi_1\Sigma\ne0$.
According to \cite{kervaire} %it is a good readable paper, but I am sure the existance follows from sometheng written before
an example of that type exists for any $m\ge 5$.

Remove from $\Sigma$ one $(m-2)$-simplex.
Denote the obtained complex by $\Sigma'$.
Since $m\ge 5$, we have $\pi_1\Sigma=\pi_1\Sigma'$.

Consider the product $\Sigma'\times [0,1]$. 
Attach to it the cone over its boundary $\partial (\Sigma'\times [0,1])$.
Denote by $\spc{S}$ the obtained simplicial complex
and by $v$ the tip of the attached cone.

Note that $\spc{S}$ is homotopy equivalent to the spherical suspension over $\Sigma$ which is a simply connected homology sphere and hence is homotopy equivalent to $\mathbb{S}^{m-1}$.
  Hence  $\spc{S}\sim\mathbb{S}^{m-1}$.

The complement $\spc{S}\backslash\{v\}$ is homotopy equivalent to $\Sigma'$.
Therefore 
\[
\pi_1(\spc{S}\backslash\{v\})
=\pi_1\Sigma'
=\pi_1\Sigma\ne 0.
\]
That is, $\spc{S}$ satisfies the conditions above.
\parbf{Exercise~\ref{ex:concave-triangle}.}
\parbf{Exercise~\ref{ex:bishop-sphere}.}
\parbf{Exercise~\ref{ex:two-planes}.}
\parbf{Exercise~\ref{ex:CAT=>two-convex}.}
\parbf{Exercise~\ref{ex:two-convex-not-a-CAT}.}