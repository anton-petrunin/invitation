%%!TEX root = invitation.tex
\chapter{Globalization and asphericity}\label{chapter:globalization}

In this chapter we introduce locally $\Cat{}{0}$ spaces and prove the globalization theorem which provides a sufficient condition for locally $\Cat{}{0}$ spaces to be globally $\Cat{}{0}$.

The theorem implies in particular, that the universal metric cover of a locally $\Cat{}{0}$ proper length space is a $\Cat{}{0}$ proper length space. 
It follows that any locally  $\Cat{}{0}$ proper length space is aspherical; 
that is, its universal cover is contractible.

This globalization theorem leads to a {}\emph{construction toy set}, described by the flag condition (\ref{thm:flag}).
Playing with this toy set, we produce examples of exotic aspherical spaces.

\section{Locally CAT spaces}

We say that a space $\spc{U}$ is \index{locally $\Cat{}{\kappa}$ space}\emph{locally $\Cat{}{0}$} (or \emph{locally $\Cat{}{1}$}) if
a small closed ball centered at any point $p$ in $\spc{U}$ is $\Cat{}{0}$ (or $\Cat{}{1}$, respectively).

For example, $\SS^1$ is locally isometric to $\RR$, and so $\SS^1$ is locally $\Cat{}{0}$.
On the other hand, $\SS^1$ is not $\Cat{}{0}$, since closed local geodesics in $\SS^1$ are not geodesics, so $\SS^1$ does not satisfy Proposition~\ref{cor:loc-geod-are-min}.

If $\spc{U}$ is a proper length space then it is locally $\Cat{}{0}$ (or locally $\Cat{}{1}$) 
if and only if 
each point $p\in \spc{U}$ admits an open neighborhood $\Omega$ which is geodesic and any triangle in $\Omega$ is thin (or spherically thin, respectively).
The proof goes along the same lines as in Exercise~\ref{ex:convex-balls}.

\section{Space of local geodesic paths}\label{sec:geod-space}

In this section we will study behavior of local geodesics in locally $\Cat{}{\kappa}$  spaces.  
The results will be used in the proof of the globalization theorem (\ref{thm:hadamard-cartan}).

Recall that \index{path}\emph{path} is a curve parametrized by $[0,1]$.
The space of paths in metric space $\spc{U}$ comes with the natural metric
\[\dist{\alpha}{\beta}{}
=
\sup\set{\dist{\alpha(t)}{\beta(t)}{\spc{U}}}{t\in[0,1]}.
\eqlbl{eq:dist-between-paths}
\]

\begin{thm}{Proposition}\label{prop:geo-complete}
Let $\spc{U}$ be a proper locally $\Cat{}{\kappa}$ length space.

Assume $\gamma_n\:[0,1]\to\spc{U}$ is a sequence of local geodesic paths converging to a path $\gamma_\infty\:[0,1]\to\spc{U}$.
Then $\gamma_\infty$ is a local geodesic path.
Moreover 
\[\length\gamma_n\to\length\gamma_\infty\]
as $n\to\infty$.
\end{thm}

\parit{Proof; $\Cat{}{0}$ case.} 
Fix $t\in[0,1]$.  
Let $R>0$ be the value such that $\cBall[\gamma_\infty(t),R]$ 
forms a $\Cat{}{0}$ proper length space.

Assume a local geodesic $\sigma$  is shorter than $R/2$ and intersects the ball $\oBall(\gamma_\infty(t),R/2)$.
Then $\sigma$ cannot leave the ball $\cBall[\gamma_\infty(t),R]$.
Hence, by Proposition~\ref{cor:loc-geod-are-min}, $\sigma$ is a geodesic.  
In particular, for all sufficiently large $n$, any arc of $\gamma_n$ of length $R/2$ or less is a geodesic.

Since $\spc{B}=\cBall[\gamma_\infty(t),R]$ is a $\Cat{}{0}$ proper length space, by Theorem~\ref{thm:cat-unique},
geodesic segments in $\spc{B}$ depend uniquely and continuously on their endpoint pairs.  
Thus there is a subinterval $\II$ of $[0,1]$,
which contains a neighborhood of $t$ in $[0,1]$
and such that the arc $\gamma_n|_\II$ is minimizing for all large~$n$.
It follows that $\gamma_\infty|_\II$ is a geodesic,
and therefore $\gamma_\infty$ is a local geodesic.

The $\Cat{}{1}$ case is done the same way, but one has to assume in addition that $R<\pi$.
\qeds

The following lemma and its proof were suggested to us by Alexander Lytchak.  
This lemma allows  a local geodesic path 
to be moved continuously so that its endpoints follow given trajectories.
This statement was originally proved by the first author and Richard Bishop using a different method; see~\cite{a-b:h-c}.

\begin{thm}{Patchwork along a curve}
\label{lem:patch}
Let $\spc{U}$ be a locally $\Cat{}{0}$  proper length space, 
and $\gamma\:[0,1]\to\spc{U}$ be a 
 path.

Then there is a $\Cat{}{0}$ proper length space $\spc{N}$,
an open set $\hat\Omega\subset \spc{N}$,
and a  
 path $\hat\gamma\:[0,1]\to\hat\Omega$,
such that there is an open locally isometric immersion 
$\map\:\hat\Omega\looparrowright\spc{U}$ satisfying
$\map\circ\hat\gamma=\gamma$.

If $\length\gamma<\pi$,
then the same holds in $\Cat{}{1}$ case.
Namely we assume that $\spc{U}$ is a $\Cat{}{1}$ proper length space and construct a $\Cat{}{1}$ proper length space $\spc{N}$ with the same property as above.
\end{thm}

\parit{Proof.} 
Fix $r>0$ so that for each $t\in[0,1]$,
the closed ball
$\cBall[\gamma(t),r]$ forms a $\Cat{}{\kappa}$ proper length space.

\begin{center}
\begin{lpic}[t(0mm),b(0mm),r(0mm),l(0mm)]{pics/patching-balls(1)}
\lbl{7,17;$\spc{B}^0$}
\lbl{25,4;{$\spc{B}^1$}}
\lbl{47,2;$\dots$}
\lbl{58,2.5,4;$\dots$}
\lbl{86,17;$\spc{B}^n$}
\end{lpic}
\end{center}

Choose a partition $0\z=t^0<t^1<\dots<t^n\z=1$ so that 
\[\oBall(\gamma(t^i),r)\supset \gamma([t^{i-1},t^i])\] for all $n>i>0$.
Set $\spc{B}^i=\cBall[\gamma(t^i),r]$.

Consider the disjoint union $\bigsqcup_i\spc{B}^i=\set{(i,x)}{x\in\spc{B}^i}$ with the minimal equivalence relation $\sim$ such that $(i,x)\sim(i-1,x)$ for all~$i$.
Let  $\spc{N}$ be the space obtained by gluing the $\spc{B}^i$ along~$\sim$.

Note that $A^i=\spc{B}^i\cap\spc{B}^{i-1}$ is convex in $\spc{B}^i$ and in $\spc{B}^{i-1}$.
Applying the Reshetnyak gluing theorem (\ref{thm:gluing}) $n$ times, 
we conclude that $\spc{N}$ is a $\Cat{}{0}$ proper length space.

For $t\in[t^{i-1},t^i]$, define $\hat\gamma(t)$ as the equivalence class of $(i,\gamma(t))$ in~$\spc{N}$.
Let $\hat\Omega$ be the $\eps$-neighborhood of $\hat\gamma$ in $\spc{N}$, where $\eps>0$ is chosen so that $\oBall(\gamma(t),\eps)\subset\spc{B}^i$ for all $t\in[t^{i-1},t^i]$.

Define $\map\:\hat\Omega\to\spc{U}$
by sending the equivalence class of $(i,x)$ to~$x$.
It is straightforward to check that $\map$, 
$\hat\gamma$ and $\hat\Omega\subset\spc{N}$ satisfy the conclusion of  the lemma.

The $\Cat{}{1}$ case is proved the same way.
\qeds

The following two corollaries follow from:
(1) the patchwork (\ref{lem:patch});
(2) Proposition \ref{cor:loc-geod-are-min}, which states that local geodesics are geodesics in any $\Cat{}{}$ space; 
and (3) Theorem \ref{thm:cat-unique} on uniqueness of geodesics.

\begin{thm}{Corollary}\label{cor:discrete-paths}
If $\spc{U}$ is a locally $\Cat{}{0}$ proper length space, then for any pair of points $p,q\in\spc{U}$, the space of all local geodesic paths from $p$ to $q$ is discrete;
that is, for any local geodesic path $\gamma$ connecting $p$ to $q$, there is $\eps>0$ such that for any other local geodesic path $\delta$ from $p$ to $q$ we have
$\dist{\gamma(t)}{\delta(t)}{\spc{U}}>\eps$ for some $t\in[0,1]$.

Analogously, if $\spc{U}$ is a locally $\Cat{}{1}$ proper length space, then for any pair of points $p,q\in\spc{U}$,  the space of all local geodesic paths shorter than $\pi$ from $p$ to $q$ is discrete.
\end{thm}

\begin{thm}{Corollary}\label{cor:path-geod}
If $\spc{U}$ is a locally $\Cat{}{0}$ proper length space, then 
for any path $\alpha$ there is a choice of  a local geodesic path $\gamma_\alpha$  connecting the ends of $\alpha$ such that the map $\alpha\mapsto\gamma_\alpha$ is continuous, and if $\alpha$ is a local geodesic path then $\gamma_\alpha=\alpha$. 

Analogously, if $\spc{U}$ is a locally $\Cat{}{1}$ proper length space, then 
for any path $\alpha$ shorter than $\pi$,  
there is a choice of a local geodesic path $\gamma_\alpha$ shorter than $\pi$ connecting the ends of $\alpha$ such that the map $\alpha\mapsto\gamma_\alpha$ is continuous, and if $\alpha$ is a local geodesic path then $\gamma_\alpha=\alpha$.
\end{thm}

\parit{Proof of \ref{cor:path-geod}.} 
We do the $\Cat{}{0}$ case;
the $\Cat{}{1}$ case is analogous.

Consider the maximal interval $\II\subset[0,1]$ containing $0$,
such that there is a continuous one-parameter family of 
local geodesic paths $\gamma_t$ for $t\in \II$ connecting $\alpha(0)$ to $\alpha(t)$ with $\gamma_t(0)=\gamma_0(t)=\alpha(0)$ for any~$t$. 

By Proposition~\ref{prop:geo-complete}, $\II$ is closed,
so we can assume $\II=[0,s]$ for some $s\in [0,1]$.

Applying the patchwork (\ref{lem:patch}) to  $\gamma_{s}$, 
we get that $\II$ is also open in $[0,1]$. 
Hence $\II=[0,1]$.
Set $\gamma_\alpha=\gamma_1$.

By construction,  if $\alpha$ is a local geodesic path, then $\gamma_\alpha=\alpha$. 

Moreover, from Corollary \ref{cor:discrete-paths},
the construction $\alpha\mapsto \gamma_\alpha$ produces close results for sufficiently close paths in the metric defined by \ref{eq:dist-between-paths};
that is, the map  $\alpha\mapsto \gamma_\alpha$ is continuous.
\qeds

Given a path $\alpha\:[0,1]\to\spc{U}$,
we denote by $\bar\alpha$ the same path traveled in the opposite direction;
that is,
\[\bar\alpha(t)=\alpha(1-t).\]
The \index{product of paths}\emph{product} of two paths  will be denoted with ``$*$'';
if two paths $\alpha$ and $\beta$ connect the same pair of points then the product $\bar\alpha*\beta$ is a closed curve.

\begin{thm}{Exercise}\label{ex:null-homotopic}
Assume $\spc{U}$ is a locally $\Cat{}{1}$ proper length space. 
Consider the construction $\alpha\mapsto\gamma_\alpha$ provided by Corollary~\ref{cor:path-geod}.

Assume $\alpha$ and $\beta$ are two paths connecting the same pair of points in $\spc{U}$, where 
each is shorter than $\pi$ 
and the product  
$\bar\alpha*\beta$ is null-homotopic in the class of closed curves shorter than $2\cdot\pi$.
Then $\gamma_\alpha=\gamma_\beta$.
\end{thm}

\section{Globalization}\label{sec:Hadamard--Cartan}

The original formulation of the 
\index{globalization theorem}\emph{globalization theorem}, or 
\index{Hadamard--Cartan theorem}\emph{Hadamard--Cartan theorem}, states that if $M$ is a complete Riemannian manifold with sectional curvature $\le 0$,  
then the exponential map at any point $p\in M$ is a covering;
in particular it implies that the universal cover of $M$ is diffeomorphic to the Euclidean space of the same dimension.

In this generality, this theorem appeared in the lectures of Elie Cartan, see~\cite{cartan}.
This theorem was proved for surfaces in Euclidean $3$-space 
by Hans von Mangoldt \cite{mangoldt},  
and a few years later independently for two-dimensional Riemannian manifolds by Jacques Hadamard \cite{hadamard}.

Formulations for metric spaces of different generality were proved by 
Herbert Busemann in \cite{busemann-CBA},
Willi Rinow in \cite{rinow},
Mikhael Gromov in \cite[p.119]{gromov:hyp-groups}. 
A detailed proof of Gromov's statement was given by Werner Ballmann in \cite{ballmann:cartan-hadamard} when $\spc{U}$ is proper,
and by the first author and Richard Bishop in \cite{a-b:h-c} in more generality;  also see references in our book~\cite{AKP}.

For  $\Cat{}{1}$ proper spaces, the globalization theorem was proved by Brian Bowditch in~\cite{bowditch}.

\begin{thm}{Globalization theorem}
\label{thm:hadamard-cartan}
Any simply connected proper locally $\Cat{}{0}$ length space 
is $\Cat{}{0}$.

Analogously, assume $\spc{U}$ is a locally $\Cat{}{1}$ proper length space
such that any closed curve $\gamma\:\SS^1\to \spc{U}$ shorter than $2\cdot\pi$
is null-homotopic in the class of closed curves shorter than $2\cdot\pi$.
Then $\spc{U}$ is $\Cat{}{1}$.
\end{thm}

\begin{wrapfigure}{r}{39mm}
\begin{lpic}[t(-3mm),b(-3mm),r(0mm),l(0mm)]{pics/cat-1(1)}
\end{lpic}
\end{wrapfigure}

The surface on the diagram 
is an example of a simply connected space that  is locally $\Cat{}{1}$ but not $\Cat{}{1}$.
To contract the marked curve one has to increase its length to $2\cdot\pi$ or more;
in particular the surface does not satisfy the assumption of the globalization theorem.


The proof of the globalization theorem relies on the following theorem, 
which essentially is \cite[Satz 9]{alexandrov:devel}.  

\begin{thm}{Patchwork globalization theorem}\label{thm:alex-patch}
A locally $\Cat{}{0}$ proper length space $\spc{U}$ is $\Cat{}{0}$
if and only if all pairs of points in $\spc{U}$  are joined by unique geodesics, and these geodesics depend continuously on their endpoint pairs.

Analogously, a locally $\Cat{}{1}$ proper length space $\spc{U}$ is $\Cat{}{1}$ 
if and only if all pairs of points in $\spc{U}$ at distance less than $\pi$ are joined by unique geodesics, and these geodesics depend continuously on their endpoint pairs.
\end{thm}

The proof uses a thin-triangle decomposition with the inheritance lemma (\ref{lem:inherit-angle}) and the following construction:

\begin{thm}{Line-of-sight map} \label{def:sight}
Let  $p$ be a point and $\alpha$ be a curve of finite length in  a length space~$\spc{X}$. 
Let $\mathring\alpha:[0,1]\to\spc{U}$ be the constant-speed parametrization of~$\alpha$.  
If   $\gamma_t\:[0,1]\to\spc{U}$ is a geodesic path from $p$ to $\mathring\alpha(t)$, we say 
\[
[0,1]\times[0,1]\to\spc{U}\:(t,s)\mapsto\gamma_t(s)%\eqlbl{eq:line-of-sight}
\]
is a \index{line-of-sight map}\emph{line-of-sight map from $p$ to $\alpha$}.  
\end{thm}

\parit{Proof of the patchwork globalization theorem (\ref{thm:alex-patch}).} 
Note that the implication ``only if'' is already proved in  Theorem~\ref{thm:cat-unique}; it only remains to prove ``if'' part.

Fix a triangle $\trig p x y$  in $\spc{U}$. 
We need to show that $\trig p x y$ is thin.

By the assumptions, the line-of-sight map  $(t,s)\mapsto\gamma_t(s)$ from $p$ to   $[x y]$ is uniquely defined and continuous.    

\begin{center}
\begin{lpic}[t(3mm),b(3mm),r(0mm),l(0mm)]{pics/cba-globalization-net(1)}
\lbl[lb]{29,46;$p=x^{0,0}=\dots=x^{N,0}$}
\lbl[rb]{23,37;$x^{0,1}$}
\lbl[lb]{39,37;$x^{N,1}$}
\lbl{13,23,56;$\dots$}
\lbl{52,23,-46;$\dots$}
\lbl[rt]{2,0;$x^{0,N}=x$}
\lbl[tl]{14,0;$x^{1,N}$}
\lbl{44,-1;$\dots$}
\lbl[lt]{69,0;$y=x^{N,N}$}
\end{lpic}
\end{center}

Fix  a partition \[0\z=t^0\z<t^1\z<\z\dots\z<t^N=1,\] 
and set $x^{i,j}=\gamma_{t^i}(t^j)$. 
Since the line-of-sight map is continuous and $\spc{U}$ is locally $\Cat{}{0}$, we may assume that the triangles 
\[\trig{x^{i,j}}{x^{i,j+1}}{x^{i+1,j+1}}\quad\text{and}\quad\trig{x^{i,j}}{x^{i+1,j}}{x^{i+1,j+1}}\] 
are thin for each pair $i$,~$j$.

Now we show that the thin property propagates to $\trig p x y$ by repeated application of the inheritance lemma (\ref{lem:inherit-angle}):
\begin{itemize}
\item 
For fixed $i$, 
sequentially applying the lemma shows that the triangles 
$\trig{p}{x^{i,1}}{x^{i+1,2}}$, 
$\trig{p}{x^{i,2}}{x^{i+1,2}}$, 
$\trig{p}{x^{i,2}}{x^{i+1,3}}$,
and so on are thin. 
\end{itemize}
In particular, for each $i$, the long triangle $\trig{p}{x^{i,N}}{x^{i+1,N}}$ is thin.
\begin{itemize} 
\item 
Applying the same lemma again shows that the  triangles $\trig{p}{x^{0,N}}{x^{2,N}}$, $\trig{p}{x^{0,N}}{x^{3,N}}$, and so on are thin. 
\end{itemize}
In particular, $\trig p x y=\trig{p}{x^{0,N}}{x^{N,N}}$ is thin.
\qeds

\parit{Proof of the globalization theorem; $\Cat{}{0}$ case.}
Let $\spc{U}$ be a simply connected, locally $\Cat{}{0}$ proper length space.
Given a path $\alpha$ in $\spc{U}$, 
denote by $\gamma_\alpha$ the local geodesic path provided by Corollary \ref{cor:path-geod}.
Since the map $\alpha\mapsto\gamma_\alpha$ is continuous, by Corollary~\ref{cor:discrete-paths}
we have $\gamma_\alpha=\gamma_\beta$ for any pair of  paths $\alpha$ and $\beta$  homotopic relative to the ends.

Since $\spc{U}$ is simply connected, any pair of paths with common ends are homotopic.  In particular, if $\alpha$ and $\beta$ are local geodesics from $p$ to $q$, then $\alpha =\gamma_\alpha=\gamma_\beta=\beta$ by Corollary \ref{cor:path-geod}.
It follows that any two points $p,q\in\spc{U}$ are joined by a unique local geodesic that depends continuously on $(p,q)$.

Since $\spc{U}$ is geodesic, it remains to apply the patchwork globalization theorem (\ref{thm:alex-patch}).

\parit{$\Cat{}{1}$ case.}
The proof goes along the same lines, 
but one needs to use Exercise~\ref{ex:null-homotopic}. \qeds

\begin{thm}{Corollary}\label{cor:closed-geod-cat} 
Any compact locally $\Cat{}{0}$ space that contains no closed local geodesics is $\Cat{}{0}$.
 
Analogously, any compact locally $\Cat{}{1}$ space that  contains no closed local geodesics shorter than $2\cdot\pi$ is $\Cat{}{1}$.
\end{thm}

\parit{Proof.}
By the globalization theorem (\ref{thm:hadamard-cartan}), we need to show that the space is simply connected.
Assume the contrary. 
Fix a nontrivial homotopy class of closed curves.

Denote by $\ell$ the exact lower bound for the lengths of curves in the class.
Note that $\ell>0$;
otherwise there would be a closed noncontractible curve in a $\Cat{}{0}$ spherical neighborhood of some point, contradicting Corollary \ref{cor:contractible-cat}.

Since the space is compact, the class contains a length-minimizing curve, 
which must be a closed local geodesic. 

The $\Cat{}{1}$ case is analogous, one only has to consider a homotopy class of closed curves shorter than $2\cdot\pi$.
\qeds

\begin{thm}{Exercise}\label{ex:geod-circle}
Prove that any compact locally $\Cat{}{0}$ space that is not $\Cat{}{0}$ contains a \index{geodesic circle}\emph{geodesic circle}, 
that is, a simple closed curve $\gamma$ such that 
for any two points $p,q\in\gamma$, one of the arcs of $\gamma$ with endpoints $p$ and $q$ is a  geodesic.

Formulate and prove the analogous statement for $\Cat{}{1}$ spaces.
\end{thm}

\begin{thm}{Exercise}\label{ex:branching-cover} 
Let $\spc{U}$ be a $\Cat{}{0}$ proper length space.
Assume $\tilde{\spc{U}}\to \spc{U}$ is a metric  double covering branching along a geodesic.
Show that $\tilde{\spc{U}}$ is $\Cat{}{0}$.
\end{thm}

\section{Polyhedral spaces}

\begin{thm}{Definition}\label{def:poly}
A length space $\spc{P}$ is called  
a \index{polyhedral space}\emph{(spherical) polyhedral space} 
if it admits a finite triangulation $\tau$ 
such that every simplex in $\tau$ is isometric to a simplex in a Euclidean space (or correspondingly a unit sphere) of appropriate dimension.

By a 
\index{triangulation of a polyhedral space}\emph{triangulation of a polyhedral space} 
we will always understand the triangulation as above. 
\end{thm}

Note that according to the above definition,
all polyhedral spaces are compact.
However, 
most of the statements below admit straightforward generalizations 
to \index{polyhedral spaces!locally polyhedral spaces}\emph{locally polyhedral spaces};
that is, complete length spaces,  
any point of which admits a closed neighborhood isometric to a polyhedral space.
The latter class of spaces includes in particular infinite covers of polyhedral spaces.

The \index{dimension of a polyhedral space}\emph{dimension} of a polyhedral space $\spc{P}$
is defined as the maximal dimension of the simplices 
in one (and therefore any) triangulation of~$\spc{P}$.

\parbf{Links.}
Let $\spc{P}$ be a polyhedral space
and $\sigma$ be a simplex in a triangulation $\tau$ of~$\spc{P}$.

The simplices which contain $\sigma$
form an abstract simplicial complex called the \index{link}\emph{link} of $\sigma$, 
denoted by $\Link_\sigma$.
If $m$ is  the dimension of $\sigma$
then the set of vertices of $\Link_\sigma$
is formed by the $(m+1)$-simplices which contain $\sigma$;
the set of its edges are formed by the $(m+2)$-simplices 
which contain $\sigma$; and so on.

The link $\Link_\sigma$
can be identified with the subcomplex of $\tau$ 
formed by all the simplices $\sigma'$ 
such that $\sigma\cap\sigma'=\emptyset$ 
but both $\sigma$ and $\sigma'$ are faces of a simplex of~$\tau$.

The points in $\Link_\sigma$ can be identified with the normal directions to $\sigma$ at a point in its interior.
The angle metric between directions makes  $\Link_\sigma$ into a spherical polyhedral space.
We will always consider the link with this metric.

\parbf{Tangent space and space of directions.}
Let $\spc{P}$ be a polyhedral space (Euclidean or spherical) and  $\tau$ be its triangulation.
If a point $p\in \spc{P}$ 
lies in the interior of a $\kay$-simplex $\sigma$ of $\tau$ 
then the tangent space $\T_p=\T_p\spc{P}$
is  naturally isometric to
\[\EE^\kay\times(\Cone\Link_\sigma).\]
Equivalently, the space of directions $\Sigma_p=\Sigma_p\spc{P}$
can be isometrically identified with the 
$\kay$-times-iterated suspension over $\Link_\sigma$;
that is, 
\[\Sigma_p\iso\Susp^{\kay}(\Link_\sigma).\]

If $\spc{P}$ is an $m$-dimensional polyhedral space,
then for any $p\in \spc{P}$
the space of directions $\Sigma_p$ is a spherical polyhedral space
of dimension at most $m-1$. 

In particular, 
for any point $p$ in $\sigma$,
the isometry class of $\Link_\sigma$ together with $\kay=\dim\sigma$
determine the isometry class of $\Sigma_p$, 
 and the other way around --- $\Sigma_p$ and $k$ determine the isometry class of $\Link_\sigma$.

A small neighborhood of $p$ is isometric to a neighborhood of the tip of $\Cone\Sigma_p$. 
(If $\spc{P}$ is a spherical polyhedral space, then a small neighborhood of $p$ is isometric to a neighborhood of the north pole in $\Susp\Sigma_p$.)
In fact, if this property holds at any point of a compact length space $\spc{P}$,
then  $\spc{P}$ is a polyhedral space, 
see \cite{lebedeva-petrunin-poly} by Nina Lebedeva and the third author.

The following theorem provides a combinatorial description of polyhedral spaces with curvature bounded above.


\begin{thm}{Theorem}\label{thm:PL-CAT}
Let $\spc{P}$ be a polyhedral space and $\tau$ be its triangulation. 
Then $\spc{P}$ is locally $\Cat{}{0}$ if and only if the link of each simplex in $\tau$ has no closed local geodesic shorter than $2\cdot\pi$.

Analogously, let $\spc{P}$ be a spherical polyhedral space and $\tau$ be its triangulation. 
Then $\spc{P}$ is $\Cat{}{1}$ if and only if neither $\spc{P}$ nor  the  link of any simplex in $\tau$ has a closed local geodesic shorter than $2\cdot\pi$.
\end{thm}





\parit{Proof of \ref{thm:PL-CAT}.}
The ``only if'' part follows from
Proposition~\ref{cor:loc-geod-are-min} 
and 
Exercise~\ref{ex:cone+susp}.

To prove ``if'' part,
we apply induction on $\dim\spc{P}$.
The base case $\dim\spc{P}=0$ is evident.
Let us start with the $\Cat{}{1}$ case.

\parit{Step.}
Assume that the theorem is proved in the case $\dim\spc{P}<m$. Suppose  $\dim\spc{P}=m$.


Fix a point $p\in\spc{P}$.
A neighborhood of $p$ 
is isometric to a neighborhood of the north pole in the suspension over the space of directions~$\Sigma_p$.

Note that $\Sigma_p$ is a spherical polyhedral space 
and its  links are isometric to  links of~$\spc{P}$. 
By the  induction hypothesis, $\Sigma_p$ is $\Cat{}{1}$.
Thus, by the second part of Exercise~\ref{ex:cone+susp}, $\spc{P}$ is locally  $\Cat{}{1}$.


Applying the second part of Corollary~\ref{cor:closed-geod-cat},
we get the statement.

The $\Cat{}{0}$ case is done exactly the same way except we need to use the first part of Exercise~\ref{ex:cone+susp} and  the first part of Corollary~\ref{cor:closed-geod-cat} on the last step.
\qeds

\begin{thm}{Exercise}\label{ex:unique-geod=CAT}
Let $\spc{P}$
be a polyhedral space such that any two points can be connected by a unique geodesic.
Show that $\spc{P}$ is $\Cat{}{0}$.
\end{thm}

\begin{thm}{Advanced exercise}\label{ex:S3}
Construct a Euclidean polyhedral metric on $\mathbb{S}^3$
such that the total angle around each edge in its triangulation is at least $2\cdot \pi$.
\end{thm}


\section{Flag complexes}


\begin{thm}{Definition}\label{def:flag}
A simplicial complex $\mathcal{S}$ 
is called \index{flag complex}\emph{flag} if whenever $\{v^0,\z\dots,v^\kay\}$
is a set of distinct vertices of $\mathcal{S}$
which are pairwise joined by edges, then the vertices $v^0,\dots,v^\kay$
span a $\kay$-simplex in~$\mathcal{S}$.

If the above condition is satisfied for $\kay=2$, 
then we say $\mathcal{S}$ satisfies 
the \index{no-triangle condition}\emph{no-triangle condition}.
\end{thm}

Note that every flag complex is determined by its one-skeleton.
Moreover, for any graph, its \index{clique}\emph{cliques} (that is, complete subgraphs) define a flag complex.
By that reason flag complex also called \index{clique complex}\emph{clique complex}.

\begin{thm}{Exercise}\label{ex:baricenric-flag}
Show that the barycentric subdivision of any simplicial complex is a flag complex.

Use the flag condition (see \ref{thm:flag} below)
to conclude that any finite simplicial complex is homeomorphic to a $\Cat{}{1}$ proper length space.

\end{thm}


\begin{thm}{Proposition}\label{prop:no-trig}
A simplicial complex $\mathcal{S}$ is flag if and only if 
$\mathcal{S}$ as well as all the links of all its simplices
satisfy the no-triangle condition.
\end{thm}

From the definition of flag complex 
we get the following.

\begin{thm}{Observation}\label{obs:link-of-flag}
Any link of any simplex in a flag complex is flag.
\end{thm}


\parit{Proof of \ref{prop:no-trig}.}
By Observation~\ref{obs:link-of-flag}, the no-triangle condition holds 
for any flag complex and the  links of all its simplices.

Now assume a complex $\spc{S}$ and all its links satisfy 
the no-triangle condition.
It follows that $\spc{S}$ includes a 2-simplex for each triangle.
Applying the same observation for each edge we get that $\spc{S}$ 
includes a 3-simplex for any complete graph with 4 vertices.
Repeating this observation 
for triangles, 
4-simplices,
5-simplices
and so on, we get that $\spc{S}$ is flag.
\qeds


\parbf{All-right triangulation.}
A triangulation of a spherical polyhedral space 
is called an  \index{all-right triangulation}\emph{all-right triangulation} 
if each simplex of the triangulation is isometric 
to a spherical simplex all of whose angles are right.
Similarly, we say that a simplicial complex 
is equipped with an  \index{all-right spherical metric}\emph{all-right spherical metric}
if it is a length metric and each simplex is isometric 
to a spherical simplex all of whose angles are right.

Spherical polyhedral $\Cat{}{1}$ spaces glued from of right-angled simplices
admit the following characterization 
discovered by Mikhael Gromov \cite[p. 122]{gromov:hyp-groups}.

\begin{thm}{Flag condition}\label{thm:flag}
Assume that a spherical polyhedral space $\spc{P}$
admits an all-right triangulation~$\tau$.
Then $\spc{P}$ is $\Cat{}{1}$
if and only if $\tau$ is flag.
\end{thm}

\parit{Proof; ``only-if'' part.} 
Assume there are three vertices $v^1,v^2$ and $v^3$ of $\tau$
that are pairwise joined by edges 
but do not span a triangle.
Note that in this case 
\[
\mangle\hinge{v^1}{v^2}{v^3}=
\mangle\hinge{v^2}{v^3}{v^1}=
\mangle\hinge{v^3}{v^1}{v^2}=
\pi.
\]
Equivalently,
\begin{clm}{}\label{clm:3pi/2}
The product
of the geodesics $[v^1v^2]$, $[v^2v^3]$ and $[v^3v^1]$
forms a locally geodesic loop in~$\spc{P}$ of length $\tfrac32\cdot\pi$.
\end{clm}

Now assume that $\spc{P}$ is $\Cat{}{1}$.
Then by Theorem~\ref{thm:PL-CAT},
$\Link_\sigma\spc{P}$ is $\Cat{}{1}$ for every simplex $\sigma$ 
in~$\tau$. 

Each of these links is an all-right spherical complex and by Theorem \ref{thm:PL-CAT}, none of these links can contain a geodesic circle shorter than $2\cdot\pi$. 

Therefore Proposition~\ref{prop:no-trig} and \ref{clm:3pi/2} 
imply the ``only-if'' part.

\parit{``If'' part.} 
By Observation~\ref{obs:link-of-flag} and Theorem~\ref{thm:PL-CAT},
it is sufficient to show that any closed local geodesic $\gamma$ 
in a flag complex $\spc{S}$ with all-right metric has length at least $2\cdot\pi$.

Recall that the  \index{star of vertex}\emph{closed star} of a vertex $v$ (briefly $\overline \Star_v$)
is formed by all the simplices containing~$v$. 
Similarly, $\Star_v$, the open star of $v$, is the union of all simplices containing $v$ with faces opposite $v$ removed.

Choose a simplex $\sigma$ which contains a point $\gamma(t_0)$.
Let $v$ be a vertex of~$\sigma$.
Consider the maximal arc $\gamma_v$ of $\gamma$ 
which contains the point $\gamma(t_0)$
and runs in $\Star_v$.
Note that the distance $\dist{v}{\gamma_v(t)}{\spc{P}}$ behaves exactly the same way 
as the distance from north pole to a geodesic in the north hemisphere;
that is, there is a geodesic $\tilde\gamma_v$ in the north hemisphere of $\mathbb{S}^2$ such that for any $t$ we have
\[\dist{v}{\gamma_v(t)}{\spc{P}}
=
\dist{n}{\tilde\gamma_v(t)}{\mathbb{S}^2},\]
where $n$ denotes the north pole of~$\mathbb{S}^2$.
In particular, 
\[\length\gamma_v=\pi;\]
that is, $\gamma$ spends time $\pi$ on every visit to $\Star_v$.

\begin{wrapfigure}{r}{44mm}
\begin{lpic}[t(-1mm),b(0mm),r(0mm),l(0mm)]{pics/two-stars(1)}
\lbl[br]{10,11;$v$}
\lbl[bl]{33,11;$v'$}
\lbl{11,5;$\Star_v$}
\lbl{32,19;$\Star_{v'}$}
\lbl[b]{15,18;$\gamma_v$}
\end{lpic}
\end{wrapfigure}

After leaving $\Star_v$,
the local geodesic $\gamma$ has to enter another simplex, 
say~$\sigma'$.
Since $\tau$ is flag, the simplex $\sigma'$
has a vertex $v'$ not joined to $v$ by an edge;
that is, 
\[\Star_v\cap\Star_{v'}=\emptyset\]

The same argument as above shows that $\gamma$ spends time $\pi$ on every visit to $\Star_{v'}$.
Therefore the total length of $\gamma$ is at least~$2\cdot\pi$.
\qeds

\begin{thm}{Exercise}\label{ex:flag>=pi/2}
Assume that a spherical polyhedral space $\spc{P}$
admits a triangulation $\tau$ such that all sides of all simplices are at least~$\tfrac\pi2$.
Show that $\spc{P}$ is $\Cat{}{1}$
if $\tau$ is flag.
\end{thm}

\parbf{The space of trees.}
The following construction is given by
Louis Billera, Susan Holmes and  Karen Vogtmann
 in~\cite{BHV}.

Let $\spc{T}_n$ be the set of all metric trees with 
$n$ end vertices
labeled by $a^1,\dots,a^n$.
To describe one tree in $\spc{T}_n$ we may fix a topological  trivalent tree $t$ with end vertices $a^1,\dots,a^n$ 
and all  other vertices of degree 3; 
and prescribe the lengths of $2\cdot n-3$ edges.
If the length of an edge vanish, we assume that this edge degenerates;
such a tree can be also described using a different topological tree~$t'$.
The subset of $\spc{T}_n$ corresponding to the given topological tree $t$ can be identified with the octant
\[\set{(x_1,\dots,x_{2\cdot n-3})\in\mathbb{R}^{2\cdot n-3}}{x_i\ge 0}.\]
Equip each such subset with the metric induced from $\mathbb{R}^{2\cdot n-3}$ and consider the length metric on $\spc{T}_n$ induced by these metrics.

\begin{thm}{Exercise}\label{ex:tree}
Show that $\spc{T}_n$ with the described metric is $\Cat{}0$.
\end{thm}



\section{Cubical complexes}

The definition of a cubical complex
mostly repeats the definition of a simplicial complex, 
with simplices replaced by cubes.

Formally, a \index{cubical complex}\emph{cubical complex} is defined as a subcomplex 
of the unit cube in the Euclidean space of large dimension;
that is, a collection of faces of the cube
such that together with each face it contains all its sub-faces.
Each cube face in this collection 
will be called a \index{cube}\emph{cube} of the cubical complex.

Note that according to this definition, 
any cubical complex is finite.

The union of all the cubes in a cubical complex $\spc{Q}$ will be called its \index{underlying space}\emph{underlying space}.
A homeomorphism from the underlying space of $\spc{Q}$ to a topological space $\spc{X}$ is called a \index{cubulation}\emph{cubulation of}~$\spc{X}$.

The underlying space of a cubical complex $\spc{Q}$ will be always considered with the length metric
induced from~$\RR^N$.
In particular, with this metric, 
each cube of $\spc{Q}$ is isometric to the unit cube of the same dimension.

It is straightforward to construct a triangulation 
of the underlying space of $\spc{Q}$ 
such that each simplex is isometric to a Euclidean simplex.
In particular the underlying space of $\spc{Q}$ is a Euclidean polyhedral space.

The link of each cube in a cubical complex admits a natural 
all-right triangulation --- each simplex corresponds to an adjusted cube.

\parbf{Cubical analog of a simplicial complex.}
Let $\spc{S}$ be a finite simplicial complex and $\{v_1,\dots,v_N\}$ be the set of its vertices.

Consider $\RR^N$ with the standard basis $\{e_1,\dots,e_N\}$.
Denote by $\square^N$ the standard unit cube in $\RR^N$;
that is 
\[\square^N=\set{(x_1,\dots,x_N)\in \RR^N}{0\le x_i\le 1\ \text{for each}\ i}.\]

Given a $\kay$-dimensional simplex $\<v_{i_0},\dots,v_{i_\kay}\>$ in $\spc{S}$, 
mark the $(\kay\z+1)$-dimensional faces in $\square^N$ (there are  $2^{N-\kay}$ of them)
which are parallel to the coordinate $(k+1)$-plane 
spanned by $e_{i_0},\dots,e_{i_\kay}$.


Note that the set of all marked faces of $\square^{N}$
forms a cubical complex;
it will be called 
the \index{cubical analog}\emph{cubical analog} of $\spc{S}$
and will be denoted as $\square_\spc{S}$.

\begin{thm}{Proposition}\label{prob:cubical-analog}
Let $\spc{S}$ be a finite connected simplicial complex
and $\spc{Q}=\square_{\spc{S}}$ be its cubical analog.
Then the underlying space of $\spc{Q}$ is connected 
and the link of any vertex of $\spc{Q}$
is isometric to  ${\spc{S}}$
equipped with the spherical right-angled metric.

In particular, if $\spc{S}$ is a flag complex 
then $\spc{Q}$ is a locally $\Cat{}{0}$
and therefore its universal cover $\tilde{\spc{Q}}$ is $\Cat{}{0}$.
\end{thm}

\parit{Proof.}
The first part of the proposition follows 
from the construction of $\square_{\spc{S}}$.

If ${\spc{S}}$ is flag, 
then by the flag condition (\ref{thm:flag}) 
the link of any cube in $\spc{Q}$ is $\Cat{}{1}$.
Therefore, by the cone construction (Exercise \ref{ex:cone+susp})
$\spc{Q}$
is locally $\Cat{}{0}$.
It remains to apply the globalization theorem 
(\ref{thm:hadamard-cartan}).
\qeds

From Proposition \ref{prob:cubical-analog}, 
it follows that the cubical analog
of any flag complex is aspherical.
The following exercise states that the  converse also holds, see \cite[5.4]{davis-survey}.

\begin{thm}{Exercise}\label{ex:flag-aspherical}
Show that a finite simplicial complex is flag 
if and only if its cubical analog is aspherical.
\end{thm}

\section{Exotic aspherical manifolds}


By the globalization theorem (\ref{thm:hadamard-cartan}),
any $\Cat{}{0}$ proper length space is contractible.
Therefore all complete, locally $\Cat{}{0}$ proper length spaces 
are \index{aspherical}\emph{aspherical};
that is, they have contractible universal covers.
This observation can be used to construct examples of aspherical spaces. 

Let $\spc X$ be a proper topological space.
Recall that $\spc X$ is called 
\index{simply connected space at infinity}\emph{simply connected at infinity} 
if for any compact set $K\subset\spc X$
there is a bigger compact set $K'\supset K$
such that  $\spc X\backslash K'$ is path connected 
and any loop which lies in $\spc X\backslash K'$
is null-homotopic in  $\spc X\backslash K$.

Recall that path connected spaces are not empty by definition.
Therefore compact spaces are not simply connected at infinity.

The following example was constructed by Michael 
Davis in \cite{davis-noneuclidean}.

\begin{thm}{Proposition}\label{prop:aspherical}
For any  $m\ge 4$ there is a closed aspherical 
$m$-dimensional piecewise linear manifold
whose universal cover is not simply connected at infinity.

In particular, the universal cover of this manifold 
is not homeomorphic to the $m$-dimensional Euclidean space.
\end{thm}

The proof requires the following lemma.

\begin{thm}{Lemma}\label{lem:example-pi_infty}
Let $\spc{S}$ be a finite flag complex,
$\spc{Q}=\square_{\spc{S}}$ be its cubical analog
and $\tilde{\spc{Q}}$ be the universal cover of~$\spc{Q}$.

Assume  $\tilde{\spc{Q}}$ is simply connected at infinity.
Then $\spc{S}$ is simply connected.
\end{thm}

\parit{Proof.}
Assume $\spc{S}$ is not simply connected. Equip $\spc{S}$ with an all-right spherical metric.
Choose a shortest noncontractible circle $\gamma\:\mathbb{S}^1\to\spc{S}$ formed by the edges of~$\spc{S}$.

Note that $\gamma$ forms a one-dimensional subcomplex of $\spc{S}$ which is a closed local geodesic.
Denote by $G$ the subcomplex of $\spc{Q}$ which corresponds to~$\gamma$.

Fix a vertex $v\in G$;
let $G_v$ be the connected component of $v$ in~$G$.
Let $\tilde G$ be a connected component of the inverse image of $G_v$ in $\tilde{\spc{Q}}$
for the universal cover $\tilde{\spc{Q}}\to \spc{Q}$.
Fix a point $\tilde v\in\tilde G$ in the inverse image of~$v$.

 
Note that 
\begin{clm}{}\label{tilde-G-convex}
$\tilde G$ is a convex set in~$\tilde{\spc{Q}}$.
\end{clm}

\begin{wrapfigure}[4]{r}{30mm}
\begin{lpic}[t(-14mm),b(0mm),r(0mm),l(0mm)]{pics/gamma-in-S(1)}
\lbl[t]{15,5;$\xi$}
\lbl[b]{15,17;$\zeta$}
\lbl[r]{14.5,10.5;$e$}
\end{lpic}
\end{wrapfigure}

Indeed, according to Proposition \ref{prob:cubical-analog},
$\tilde{\spc{Q}}$ is $\Cat{}{0}$.
By Exercise \ref{ex:locally-convex},
it is sufficient to show that $\tilde G$ is locally convex in $\tilde{\spc{Q}}$,
or equivalently, $G$ is locally convex in~$\spc{Q}$.

Note that the latter can only fail if $\gamma$ contains two vertices, say $\xi$ to $\zeta$ in $\spc{S}$,
which are joined by an edge not in $\gamma$; 
denote this edge by~$e$.

Each edge of $\spc{S}$ has length~$\tfrac\pi2$.
Therefore each of two circles formed by $e$ and an arc of $\gamma$
from $\xi$ to $\zeta$ is shorter that~$\gamma$.
Moreover,
at least one of them is noncontractible 
since $\gamma$ is not.
That is, 
$\gamma$ is not a shortest noncontractible circle 
--- a contradiction.
\claimqeds

Further, note that 
$\tilde G$ is homeomorphic to the plane, 
since $\tilde G$ is 
a two-dimensional manifold without boundary which 
by the above is $\Cat{}{0}$ and hence is contractible.

Denote by $C_R$ the circle of radius $R$ in $\tilde G$ centered at~$\tilde v$.
All $C_R$ are homotopic to each other in $\tilde G\backslash\{\tilde v\}$ and therefore in $\tilde{\spc{Q}}\backslash \{\tilde v\}$.

Note that the map $\tilde{\spc{Q}}\backslash \{\tilde v\}\to \spc{S}$
which returns the direction of $[{\tilde v}{x}]$  for any $x\ne \tilde v$, maps $C_R$ to a circle homotopic to~$\gamma$.
Therefore $C_R$ is not contractible in $\tilde{\spc{Q}}\backslash \{\tilde v\}$.

In particular, 
$C_R$ is not contactable in $\tilde{\spc{Q}}\backslash K$
if $K\supset \tilde v$.
If $R$ is large, 
the circle $C_R$  
lies outside of any compact set $K'$ in~$\tilde{\spc{Q}}$.
It follows that $\tilde{\spc{Q}}$ is not simply connected at infinity, a contradiction.
\qeds

The proof of the following exercise is analogous.
It will be used latter in the proof of Proposition~\ref{prop:loc-CAT-mnfld} --- a more geometric version of Proposition~\ref{prop:aspherical}.

\begin{thm}{Exercise}\label{ex:example-pi_infty-new}
Under the assumptions of Lemma~\ref{lem:example-pi_infty}, 
for any vertex $v$ in $\spc{S}$
the complement $\spc{S}\backslash\{v\}$ is simply connected.
\end{thm}

\parit{Proof of \ref{prop:aspherical}.}
Let $\Sigma^{m-1}$ be an $(m-1)$-dimensional smooth homology sphere which is not simply connected and bounds a contractible smooth compact $m$-dimensional manifold~$\spc{W}$. 

For $m\ge 5$ the existence of such $(\spc{W}, \Sigma)$ follows from \cite{kervaire}. 
For $m=4$ it follows from the construction in \cite{mazur}.

Pick any triangulation $\tau$ of $W$ and let $\spc{S}$ be the resulting subcomplex that triangulates~$\Sigma$.


We can assume that $\spc{S}$ is flag; 
otherwise pass to the barycentric subdivision 
of $\tau$ and apply Exercise~\ref{ex:baricenric-flag}.


Let $\spc{Q}=\square_{\spc{S}}$ be the cubical analog of~$\spc{S}$.

By Proposition~\ref{prob:cubical-analog},
$\spc{Q}$ is a homology manifold.
It follows that $\spc{Q}$ is a piecewise linear manifold 
with a finite number of singularities at its vertices.


Removing a small contractible neighborhood $V_v$ of each vertex $v$ in $\spc{Q}$,
we can obtain a piecewise linear manifold $\spc{N}$
whose boundary is formed by several copies of~$\Sigma$.

Let us glue a copy of  $\spc{W}$ along its boundary to each copy of $\Sigma$ in the boundary of~$\spc{N}$.
This results in a  closed piecewise linear manifold 
$\spc{M}$ which is homotopically equivalent to~$\spc{Q}$.

Indeed, since both $V_v$ and $W$ are contractible, the identity map of  their common boundary $\Sigma$ can be extended to a homotopy equivalence relative to the boundary between $V_v$ and $W$.
Therefore the identity map on $\spc{N}$ extends to homotopy equivalences 
$f\: \spc Q\to \spc M$ and $g\:\spc M\to \spc Q$.

Finally, by the Lemma~\ref{lem:example-pi_infty},  
the universal cover $\tilde{\spc{Q}}$ of $\spc{Q}$
is not simply connected at infinity.

The same holds for 
the universal cover $\tilde{\spc{M}}$ of $\spc{M}$.
The latter follows since the constructed homotopy equivalences 
$f\: \spc Q\to \spc M$ and $g\:\spc M\to \spc Q$ 
lift to {}\emph{proper maps} 
$\tilde f \: \tilde{\spc{Q}}\to \tilde{\spc{M}}$
and $\tilde g \: \tilde{\spc{M}}\to \tilde{\spc{Q}}$;
that is, for any compact sets $A\subset \tilde{\spc{Q}}$ and $B\subset\tilde{\spc{M}}$, the inverse images $\tilde g^{-1}(A)$ and $\tilde f^{-1}(B)$ are compact.
\qeds


The following Proposition was proved by Fredric Ancel, 
Michael Davis and Craig Guilbault in \cite{ADG};
it could be considered as a more geometric version of Proposition~\ref{prop:aspherical}.


\begin{thm}{Proposition}\label{prop:loc-CAT-mnfld}
Given $m\ge 5$, there is a Euclidean polyhedral space $\spc{P}$ such that:
\begin{subthm}{}
$\spc{P}$ is homeomorphic to a closed $m$-dimensional manifold.
\end{subthm}

\begin{subthm}{}
$\spc{P}$ is locally $\Cat{}{0}$.
\end{subthm}

\begin{subthm}{}
The universal cover of $\spc{P}$ is not simply connected at infinity.
\end{subthm}
\end{thm}

There are no three-dimensional examples of that type, see \cite{rolfsen} by Dale Rolfsen.
In \cite{thurston}, Paul Thurston conjectured that the same holds in the four-dimensional case.
\parit{Proof.}
Apply Exercise~\ref{ex:example-pi_infty-new} to the barycentric subdivision of the simplicial complex $\spc{S}$ provided by Exercise~\ref{ex:funny-S}.
\qeds

\begin{thm}{Exercise}\label{ex:funny-S}
Given an integer $m\ge 5$,
construct a finite $(m-1)$-dimensional simplicial complex $\spc{S}$ such that $\Cone\spc{S}$ is homeomorphic to $\EE^m$
and $\pi_1(\spc{S}\backslash\{v\})\ne0$ for some vertex $v$ in~$\spc{S}$.
\end{thm} 


\section{Comments}

In the globalization theorem (\ref{thm:hadamard-cartan}) properness can be weakened to completeness, see our book \cite{AKP} and the references therein.

The condition on polyhedral $\Cat{}{\kappa}$ spaces given in Theorem~\ref{thm:PL-CAT} might look easy to use, 
but in fact, it is hard to check even in the very simple cases.
For example the description of those coverings of $\SS^3$ branching at three 
great circles which are $\Cat{}{1}$ requires quite a bit of work;
see \cite{charney-davis-93} --- try to guess the answer before reading.

Another example is the space $\spc{B}_4$ which is the universal cover of $\CC^4$ infinitely branching in six complex planes $z_i=z_j$ with the induced length metric.
So far it is not known if $\spc{B}_4$ is $\Cat{}{0}$.
Understanding this space could be helpful to study the braid group on 4 strings;
read \cite{panov-petrunin:ramification} 
by Dmitri Panov and the third author for more on it.
This circle of questions is closely related to the generalization of the flag condition (\ref{thm:flag}) to the spherical simplices with few acute dihedral angles.


The construction used in the proof of  Proposition~\ref{prop:aspherical} admits a number of interesting modifications,  
number of which are discussed in the survey \cite{davis-survey} by Michael Davis.

A similar argument was used by Michael Davis, 
Tadeusz Januszkiewicz 
and 
Jean-Fran\c{c}ois Lafont in \cite{davis-januszkiewicz-lafont}.
They constructed a closed smooth four-dimensional manifold $M$ with universal cover $\tilde M$ diffeomorphic to $\RR^4$ such that $M$ admits a polyhedral metric which is locally $\Cat{}{0}$, but does not admit a Riemannian metric with nonpositive sectional curvature.
Another example of that type was constructed by Stephan Stadler, see~\cite{stadler}.

It is noteworthy that any complete, simply connected Riemannian manifold with nonpositive curvature is homeomorphic to the Euclidean space of the same dimension.
In fact, by the globalization theorem
(\ref{thm:hadamard-cartan}) 
the exponential map at a point of such a manifold is a homeomorphism.
In particular, there is no Riemannian analog of Proposition~\ref{prop:loc-CAT-mnfld}.
Moreover, according to Stone's theorem, see \cite{stone, davis-januszkiewicz}, there is no piecewise linear analog of the proposition; 
that is the homeomorphism to a manifold in Proposition~\ref{prop:loc-CAT-mnfld} 
can not be made piecewise linear.

The flag condition also leads to the so-called {}\emph{hyperbolization} procedure, a flexible tool to construct  aspherical spaces;
a good survey on the subject is given by Ruth Charney and Michael Davis in \cite{charney-davis}.

All the topics discussed in this chapter link Alexandrov geometry with the fundamental group.
The theory of {}\emph{hyperbolic groups}, 
a branch of \emph{geometric group theory}, 
introduced by 
Mikhael Gromov \cite{gromov:hyp-groups},
could be considered as a further step in this direction.




