\chapter{Globalization and asphericity}

In this chapter we introduce locally $\Cat{}{0}$ spaces and show that universal cover of locally $\Cat{}{0}$ proper length space forms a $\Cat{}{0}$ proper length space. 
The latter implies that any locally $\Cat[0]$ proper length space is aspherical; 
that is its universal cover is contractible.

The globalization theorem leads to a \emph{construction toy set}, described by Flag condition~\ref{thm:flag}.
Playing with this toy set we produce examples of exotic aspherical spaces.



\section{Locally CAT spaces}

We say that space $\spc{U}$ is \emph{locally $\Cat{}{0}$} (or \emph{locally $\Cat{}{1}$}) if
a small closed ball centered at any point $p$ in $\spc{U}$ forms a $\Cat{}{0}$ (or correspondingly $\Cat{}{1}$ space).

For example, $\SS^1$ is locally isometric to $\RR$, and so $\SS^1$ is locally $\Cat{}{0}$.
On the other hand, $\SS^1$ is not $\Cat{}{0}$ since the closed geodesic in $\SS^1$ is not minimizing, so $\SS^1$ does not satisfy Theorem~\ref{thm:cat-unique}.

This definition is equivalent to saying that each point $p\in \spc{U}$ admits an open neighborhood $\Omega$
such that any triangle in $\Omega$ is thin, or correspondingly spherically thin.
The proof goes along the same lines as Exercise \ref{ex:convex-balls}.

%%%%%%%%%%%%%%%%%%%%%%%%%%%%%%%%%%%%%%%%%%%%%%%%%%%%%%%%%%%%%%%%%%%%%%

\section{Space of local geodesics}\label{sec:geod-space}

In this section we will study behavior of local geodesics in the locally $\Cat{}{}$  spaces.  
The proved statements will be used in the proof of Globalization theorem (\ref{thm:hadamard-cartan}).

\begin{thm}{Proposition}\label{prop:geo-complete}
Let $\spc{U}$ be a locally $\Cat{}{0}$ or locally $\Cat{}{1}$ proper length space.

Assume $\gamma_n\:[0,1]\to\spc{U}$ be a sequence of local geodesic paths converging to a path $\gamma_\infty\:[0,1]\to\spc{U}$.
Then $\gamma_\infty$ is a local geodesic path.
Moreover 
\[\length\gamma_n\to\length\gamma_\infty\]
as $n\to\infty$.
\end{thm}

\parit{Proof.} 
Fix $t\in[0,1]$.  
Let $R>0$ be the value such that $\cBall[\gamma_\infty(t),R]$ 
forms a $\Cat{}{0}$ proper length space.

A local geodesic segment  with length $<R/2$, and intersecting $\oBall(\gamma_\infty(t),R/2)$, cannot leave $\cBall[\gamma_\infty(t),R]$ and hence  is  minimizing by Corollary~\ref{cor:loc-geod-are-min}.  
In particular, for all sufficiently large $n$, any subsegment  of $\gamma_n$ through $\gamma_n(t)$  with length $<R/2$ is a geodesic.


Since $\cBall[\gamma_\infty(t),R]$ is a $\Cat{}{0}$ proper length space, by Theorem~\ref{thm:cat-unique},
geodesic segments in $\spc{B}$ depend uniquely and continuously on their endpoint pairs.  
Thus there is a subinterval $\II$ of $[0,1]$,
which contains a neighborhood of $t$ in $[0,1]$
and such that $\gamma_n|\II$ is minimizing for all large $n$.
It follows that $\gamma_\infty|\II$ is a geodesic,
and therefore $\gamma_\infty$ is a local geodesic.
\qeds


The following lemma and its proof were suggested to us by Alexander Lytchak.  
This lemma will allow us to move a local geodesic path 
so that its endpoints follow given trajectories.

\begin{thm}{Patchwork along a curve}
\label{lem:patch}
Let $\spc{U}$ be a locally $\Cat{}{\kappa}$  proper length space, 
and $\gamma\:[0,1]\to\spc{U}$ be a locally geodesic path.

Then there is a $\Cat{}{\kappa}$ proper length space $\spc{N}$,
an open set $\hat\Omega\subset \spc{N}$,
and a geodesic path $\hat\gamma\:[0,1]\to\hat\Omega$,
such that there is an open locally isometric immersion 
$\map\:\hat\Omega\looparrowright\spc{U}$ such that
$\map\circ\hat\gamma=\gamma$.
\end{thm}

\parit{Proof.} 
Fix $r>0$ so that for each $t\in[0,1]$,
the closed ball
$\cBall[\gamma(t),r]$ forms a $\Cat{}{\kappa}$ proper length space.

Choose a partition $0\z=t^0<t^1<\dots<t^n\z=1$ so that 
\[\oBall(\gamma(t^i),r)\supset \gamma([t^{i-1},t^i])\] for all $n>i>0$.
Set $\spc{B}^i=\cBall[\gamma(t^i),r]$.

%???PIC

Consider the disjoint union $\bigsqcup_i\spc{B}^i=\set{(i,x)}{x\in\spc{B}^i}$ with the minimal equivalence relation $\sim$ such that $(i,x)\sim(i-1,x)$ for all $i$.
Let  $\spc{N}$ be the space obtained by gluing the $\spc{B}^i$ along $\sim$.

Note that $A^i=\spc{B}^i\cap\spc{B}^{i-1}$ is convex in $\spc{B}^i$ and in $\spc{B}^{i-1}$.
Applying the Reshetnyak gluing theorem (\ref{thm:gluing}) $n$ times, 
we conclude that $\spc{N}$ is a $\Cat{}{\kappa}$ proper length space.

For $t\in[t^{i-1},t^i]$, define $\hat\gamma(t)$ as the equivalence class of $(i,\gamma(t))$ in $\spc{N}$.
Let $\hat\Omega$ be the $\eps$-neighborhood of $\hat\gamma$ in $\spc{N}$, where $\eps>0$ is chosen so that $\oBall(\gamma(t),\eps)\subset\spc{B}^i$ for all $t\in[t^{i-1},t^i]$.

Define $\map\:\hat\Omega\to\spc{U}$
by sending the equivalence class of $(i,x)$ to $x$.
It is straightforward to check that $\map$, 
$\hat\gamma$ and $\hat\Omega\subset\spc{N}$ satisfy the conclusion of the main part of the lemma.
\qeds

The following two corollaries follow from Patchwork along a curve
(\ref{lem:patch}) 
and Theorem on uniqueness of geodesics~\ref{thm:cat-unique}.

\begin{thm}{Corollary}\label{cor:discrete-paths}
If $\spc{U}$ is a locally $\Cat{}{0}$ proper length space than for any pair of points $p,q\in\spc{U}$  the space of all locally geodesic paths from $p$ to $q$ is discrete;
that is for any local geodesic path $\gamma$ connecting $p$ to $q$ there is $\eps>0$ such that for any other geodesic path $\delta$ from $p$ to $q$ we have
$\dist{\gamma(t)}{\delta(t)}{\spc{U}}>\eps$ for some $t\in[0,1]$.

Analogously, if $\spc{U}$ is a locally $\Cat{}{1}$ proper length space than for any pair of points $p,q\in\spc{U}$  the space of all locally geodesic paths from $p$ to $q$ with length less than $\pi$ is discrete.
\end{thm}

Recall that \emph{path} is a curve parametrized by $[0,1]$.
The spaces of paths in $\spc{U}$ comes with the natural metric
\[\dist{\alpha}{\beta}{}
=
\sup\set{\dist{\alpha(t)}{\beta(t)}{\spc{U}}}{t\in[0,1]}.\]

\begin{thm}{Corollary}\label{cor:path-geod}
Let $\spc{U}$ is a locally $\Cat{}{0}$ proper length space than 
for any path $\alpha$ there is a choice of local geodesic path $\gamma$ connecting the ends of $\alpha$ such that the map $\alpha\mapsto\gamma$ is continuous and if $\alpha$ is a local geodesic path then $\gamma=\alpha$. 

Analogously, if $\spc{U}$ is a locally $\Cat{}{1}$ proper length space than 
for any path $\alpha$ with length less than $\pi$ 
there is a choice of local geodesic path $\gamma$ with length less than $\pi$ connecting the ends of $\alpha$ such that the map $\alpha\mapsto\gamma$ is continuous and if $\alpha$ is a local geodesic path then $\gamma=\alpha$.
\end{thm}

Given a path $\alpha\:[0,1]\to\spc{U}$,
we denote by $\bar\alpha$ the same path traveled in the opposite direction;
that is,
\[\bar\alpha(t)=\alpha(1-t).\]
Joint of two paths will be denoted with ``$*$'';
if two paths $\alpha$ and $\beta$ connect the same pair of points then the joint $\bar\alpha*\beta$ forms a closed curve.

\begin{thm}{Exercise}\label{ex:null-homotopic}
Assume $\spc{U}$ is a locally $\Cat{}{1}$ proper length space and the construction $\alpha\mapsto\gamma_\alpha$ provided by Corollary \ref{cor:path-geod}.

Assume $\alpha$ and $\beta$ are two paths connecting the same pair of points in $\spc{U}$,
each has lengths less than $\pi$ 
and the joint 
$\bar\alpha*\beta$ is null-homotopic in the class of closed curves with length smaller than $2\cdot\pi$ then $\gamma_\alpha=\gamma_\beta$.
\end{thm}


%%%%%%%%%%%%%%%%%%%%%%%%%%%%%%%%%%%%%%%%%%%%%%%%%%%%%%%%%%%%%%%%%%

\section{Globalization}\label{sec:Hadamard--Cartan}

The original formulation of \emph{Globalization theorem}, or \emph{Hadamard--Cartan theorem} states that if $M$ is a complete Riemannian manifold with sectional curvature $\le 0$ 
then
exponential map at any point $p\in M$ is a covering;
in particular it implies that universal cover of $M$ is diffeomorphic to the Euclidean space of the same dimension.

In this generality, theorem appeared in the lectures of Cartan, see \cite{cartan}.
For surfaces in the Euclidean plane, 
this theroem was proved by
Hans von Mangoldt see \cite{mangoldt} 
and few years later independently by Hadamard \cite{hadamard}.

Formulations for metric spaces of different generality were proved by 
Busemann see \cite{busemann-CBA},
Rinow see \cite{rinow},
Gromov  \cite[p.119]{gromov:hyp-groups}. 
A detailed proof of Gromov's statement when $\spc{U}$ is proper  was given by Alexander and Bishop in \cite{ballmann:cartan-hadamard}
and generalized further \cite{a-b:h-c}.  

For $\Cat{}{1}$ spaces, the globalization theorem was proved by Bowditch in \cite{bowditch}.

\begin{thm}{Globalization theorem}
\label{thm:hadamard-cartan}
Any simply connected locally $\Cat{}{0}$ proper length space 
is $\Cat{}{0}$.

Analogously, assume $\spc{U}$ is a locally $\Cat{}{1}$ proper length space
such that any closed curve $\gamma\:\SS^1\to \spc{U}$ with length smaller than $2\cdot\pi$
is null-homotopic in the class of closed curves length smaller than $2\cdot\pi$.
Then $\spc{U}$ is a $\Cat{}{1}$ space.
\end{thm}

From the surface of revolution showed on the diagram, give an example of simply connected space which is locally $\Cat{}{1}$ but not $\Cat{}{1}$.
To contract the marked curve one has to increase its length to $2\cdot\pi$ or more. 

%%%???+PIC

The proof of globalization theorem relies on the following theorem, 
which essentially is \cite[Satz 9]{alexandrov:devel}.  
The proof use a thin-triangle decompositions, 
and the inheritance lemma (\ref{lem:inherit-angle}). 

\begin{thm}{Patchwork globalization theorem}\label{thm:alex-patch}
A locally $\Cat{}{0}$ proper length space $\spc{U}$ is $\Cat{}{0}$
if and only if the pairs of points in $\spc{U}$  are joined by unique geodesics, and these geodesics depend continuously on their endpoint pairs.

Analogously, a locally $\Cat{}{1}$ proper length space $\spc{U}$ is $\Cat{}{1}$ 
if and only if the pairs of points in $\spc{U}$ at distance less $\pi$ are joined by unique geodesics, and these geodesics depend continuously on their endpoint pairs.
\end{thm}

The proof of Patchwork globalization uses the following construction:

\begin{thm}{Line-of-sight map} \label{def:sight}
Let  $p$ be a point and $\alpha$ be a curve of finite length in  a length space $\spc{X}$. 
Let $\bar\alpha:[0,1]\to\spc{U}$ be the constant-speed parameterization of $\alpha$.  If   $\gamma_t\:[0,1]\to\spc{U}$ is a geodesic from $p$ to $\bar\alpha(t)$, we say 
\[[
0,1]\times[0,1]\to\spc{U}\:(t,s)\mapsto\gamma_t(s)%\eqlbl{eq:line-of-sight}
\]
is a \emph{line-of-sight map from $p$ to $\alpha$} .  
\end{thm}

\parit{Proof of Patchwork globalization theorem (\ref{thm:alex-patch}).} 
Note that the implication the ``only if'' part is already proved in  Theorem~\ref{thm:cat-unique}; it only remains to prove ``if'' part.

\begin{wrapfigure}{r}{58mm}
\begin{lpic}[t(0mm),b(0mm),r(0mm),l(0mm)]{pics/cba-globalization-net(0.4)}
\lbl[t]{6,0;$x$}
\lbl[lt]{21,0;$x^{0,1}$}
\lbl[lt]{65,0;$\dots$}
\lbl[l]{102,6;$x^{0,N}=y$}
\lbl[ll]{91,24;$x^{1,N}$}
\lbl[bl]{67,62,-50;$\dots$}
\lbl[l]{44,88;$x^{N,N}=p$}
\end{lpic}
\end{wrapfigure}

Fix a triangle $[p x y]$  in $\spc{U}$. 
We need to show that $\trig p x y$ is thin.

By the assumptions the line-of-sight map (\ref{def:sight}) 
\[(t,\tau)\mapsto\gamma_t(\tau)\]
for  $[x y]$ from $p$ is uniquely defied and continuous.    

Fix a fine a partition \[0\z=t^0\z<t^1\z<\z\dots\z<t^N=1,\] 
set $x^{i,j}=\gamma_{t^i}(t^j)$. 
Since the line-of-sight map is continuous and $\spc{U}$ is locally $\Cat{}{0}$, we may assume that each triangle 
\[\trig{x^{i,j}}{x^{i,j+1}}{x^{i+1,j+1}}\quad\text{and}\quad\trig{x^{i,j}}{x^{i+1,j}}{x^{i+1,j+1}}\] 
are thin.

Now we show that the thin property propagates to $\trig p x y$ by repeated application of the inheritance lemma (\ref{lem:inherit-angle}):
\begin{itemize}
\item 
First, for fixed $i$, 
sequentially applying the lemma shows  that the triangles 
$\trig{x}{x^{i,1}}{x^{i+1,2}}$, 
$\trig{x}{x^{i,2}}{x^{i+1,2}}$, 
$\trig{x}{x^{i,2}}{x^{i+1,3}}$,
and so on are thin. 
\end{itemize}
In particular, for each $i$, the long triangle $\trig{x}{x^{i,N}}{x^{i+1,N}}$ is thin.
\begin{itemize} 
\item 
Applying the lemma again shows that the  triangles $\trig{x}{x^{0,N}}{x^{2,N}}$, $\trig{x}{x^{0,N}}{x^{3,N}}$, and so on are thin. 
\end{itemize}
In particular, $\trig p x y=\trig{p}{x^{0,N}}{x^{N,N}}$ is thin.
\qeds

\parit{Proof of Globalization theorem; $\Cat{}{0}$ case.}
Given a path $\alpha\:[0,1]\to\spc{U}$ denote by $\gamma_\alpha$ the local geodesic provided by Corollary \ref{cor:path-geod}.
Since the map $\alpha\mapsto\gamma_\alpha$ is continuous,
we have $\gamma_\alpha=\gamma_\beta$ for any pair of  paths $\alpha$ and $\beta$  homotopic rel. to the ends.

Since $\spc{U}$ is simply connected any pair of paths with common ends are homotopic.
It follows that for any two points $p,q\in\spc{U}$ are joint by unique local geodesic.

It remains to apply the patchwork globalization theorem (\ref{thm:alex-patch}).

\parit{$\Cat{}{1}$ case.}
The proof goes along the same lines, 
but one one need to use Exercise~\ref{ex:null-homotopic}. \qeds


\section{Polyhedral spaces}

\begin{thm}{Definition}\label{def:poly}
A length space $\spc{P}$ is called  
\emph{polyhedral space} 
if it admits a finite triangulation $\tau$ 
such that an every simplex in $\tau$ is isometric to a simplex in a Euclidean space of appropriate dimension.

Analogousely, a length space $\spc{P}$ is called  
\emph{spherical polyhedral space} 
if it admits a finite triangulation $\tau$ 
such that an every simplex in $\tau$ is isometric to a simplex in a unit sphere of appropriate dimension.

By a 
\index{triangulation of a polyhedral space space}\emph{triangulation of a polyhedral space} 
we will always understand the triangulation as above. 
\end{thm}

Note that according to the above definition,
all polyhedral spaces are compact.
However, 
most of the statements below admit straightforward generalizations 
to \index{polyhedral spaces!locally polyhedral spaces}\emph{locally polyhedral spaces};
that is, complete length spaces,  
any point of which admits a closed neighborhood isometric to a polyhedral space.
The latter class of spaces includes in particular the infinite covers of polyhedral spaces.

The dimension of a polyhedral space $\spc{P}$
is defined as the maximal dimension of the simplex 
in one (and therefore any) triangulation of $\spc{P}$.

\parbf{Links.}
Let $\spc{P}$ be a polyhedral space
and $\sigma$ be a simplex in a triangulation $\tau$ of $\spc{P}$.

The simplices which contain $\sigma$
form an abstract simplicial complex called the \emph{link} of $\sigma$, 
denoted by $\Link_\sigma$.
If $m$ is  the dimension of $\sigma$
then the set of vertices of $\Link_\sigma$
is formed by the $(m+1)$-simplices which contain $\sigma$;
the set of its edges are formed by the $(m+2)$-simplices 
which contain $\sigma$, and so on.

The link $\Link_\sigma$
can be identified with the subcomplex of $\tau$ 
formed by all the simplices $\sigma'$ 
such that $\sigma\cap\sigma'=\emptyset$ 
but both $\sigma$ and $\sigma'$ are faces of the same simplex.

The points in $\Link_\sigma$ can be identified with the normal directions to $\sigma$ at a point in its interior.
The angle metric between directions makes  $\Link_\sigma$ into a spherical polyhedral space.
We will always consider the link with this metric.

\parbf{Tangent space and space of directions.}
Let $\spc{P}$ be a polyhedral space and let  $\tau$ be a triangulation of $\spc{P}$.
If a point $p\in \spc{P}$ 
lies in the interior of a $\kay$-simplex $\sigma$ of $\tau$ 
then the tangent space $\T_p\spc{P}$
is  naturally isometric to
\[\EE^\kay\times(\Cone\Link_\sigma).\]
Equivalently, the space of directions $\Sigma_p$
can be isometrically identified with the 
$\kay$-th spherical suspension over $\Link_\sigma$;
that is, 
\[\Sigma_p\iso\Susp^{\kay}(\Link_\sigma).\]

If $\spc{P}$ is an $m$-dimensional polyhedral space,
then for any $p\in \spc{P}$
the space of directions $\Sigma_p$ is a spherical polyhedral space
of dimension at most $m-1$. 

In particular, 
for any point $p$ in $\sigma$,
the isometry class of $\Link_\sigma$ together with $\kay=\dim\sigma$
determines the isometry class of $\Sigma_p$ 
 and the other way around.

A small neighborhood of $p$ is isometric to a neighborhood of the tip of the $\kappa$-cone over $\Sigma_p$.
In fact, if this property holds at any point of a compact length space $\spc{P}$
then  $\spc{P}$ is a polyhedral space
see \cite{lebedeva-petrunin-poly}.

The following theorem provides a combinatorial description of polyhedral spaces with curvature bounded above.


\begin{thm}{Theorem}\label{thm:PL-CAT}
Let $\spc{P}$ be a polyhedral space and $\tau$ be its triangulation. 
Then $\spc{P}$ is locally $\Cat{}{0}$ if and only if the link of any simplex in $\tau$ has no closed local geodesic shorter than $2\cdot\pi$.

Analogously, if $\spc{P}$ be a spherical polyhedral space and $\tau$ be a its triangulation. 
Then $\spc{P}$ is $\Cat{}{1}$ if and only if $\spc{P}$ and every link of any simplex in $\tau$ has no closed local geodesic shorter than $2\cdot\pi$.
\end{thm}

In the proof we will use the following corollary of Globalization theorem \ref{thm:hadamard-cartan}).

\begin{thm}{Corollary} Any locally $\Cat{}{0}$ compact space which contains no closed geodesics is $\Cat{}{0}$.
 
Analogously, any locally $\Cat{}{1}$ compact space which contains no closed geodesics of length smaller than $2\cdot\pi$ is $\Cat{}{1}$.
\end{thm}

\parit{Proof.}
By Globalization theorem (\ref{thm:hadamard-cartan}), we need to show that the space is simply connected.
Assume contrary. 
Fix a nontrivial homotopy class of closed curves.

Denote by $\ell$ the exact lower bound for the lengths of curves in the class.
Note that $\ell>0$. %%%???MORE???

Since the space is compact the class contains a length minimizing curve
which has to be a closed geodesic. 

The $\Cat{}{1}$ is analogous, one only has to consider the homotopy class of closed curves shorter than $2\cdot\pi$.
\qeds


\parit{Proof of Theorem~\ref{thm:PL-CAT}.}
We prove ``only if'' part and leave ``if'' part as an exercise.

We apply induction on $\dim\spc{P}$.
The base case $\dim\spc{P}=0$ is evident.
Let us start with $\Cat{}{1}$ case.

\parit{Step.}
Assume that the theorem is proved in the case $\dim\spc{P}<m$. Suppose  $\dim\spc{P}=m$.


Fix a point $p\in\spc{P}$.
A neighborhood of $p$ 
is isometric to the neighborhood of the north pole in the suspension over 
the $\Sigma_p$.
By the second part of Exercise~\ref{ex:cone+susp} 
it is sufficient to show that 
\[\Sigma_p\in \Cat{}1.
\eqlbl{eq:Sigma-in-CAT(1)}\]

Note that $\Sigma_p$ is a spherical polyhedral space 
and its  links are isometric to  links of $\spc{P}$. 
By the  induction hypothesis, $\Sigma_p$ is $\Cat{}{1}$.
Applying Grolbalization (\ref{thm:hadamard-cartan}),
we get the statement.

The $\Cat{}{0}$ case is different only in the last step, where we need to use the first part of Exercise~\ref{ex:cone+susp}.
\qeds

\begin{thm}{Exercise}\label{ex:unique-geod=CAT}
Show that if in a polyhedral space $\spc{P}$
any two points can be connected by a unique geodesic 
then $\spc{P}$ is a $\Cat{}{0}$ space.
\end{thm}

\section{Flag complexes}


\begin{thm}{Definition}
A simplicial complex $\mathcal{S}$ 
is called \index{flag complex}\emph{flag} if whenever $\{v^0,\z\dots,v^\kay\}$
is a set of distinct vertices of $\mathcal{S}$
which are pairwise joined by edges, then the vertexes $v^0,\dots,v^\kay$
span a $\kay$-simplex in $\mathcal{S}$.

If the above condition is satisfied only for $\kay=2$, 
then we say $\mathcal{S}$ satisfies 
the \emph{no-triangle condition}\index{no-triangle condition}.
\end{thm}

Note that every flag complex is determined by its 1-skeleton.

\begin{thm}{Exercise}\label{ex:baricenric-flag}
Show that the barycentric subdivision of any simplicial complex is a flag complex.

Conclude that any finite simplicial complex is homeomorphic to a $\Cat{}{1}$ proper length space.
\end{thm}


\begin{thm}{Proposition}\label{prop:no-trig}
A simplicial complex $\mathcal{S}$ is flag if and only if 
$\mathcal{S}$ as well as all the links of all its simplices
satisfies the no-triangle condition.
\end{thm}

From the definition of flag complex 
we get the following.

\begin{thm}{Lemma}\label{lem:link-of-flag}
Any link of a flag complex is flag.
\end{thm}


\parit{Proof of Proposition~\ref{prop:no-trig}.}
By Lemma~\ref{lem:link-of-flag}, the no-triangle condition holds 
for any flag complex and all its links.

Now assume a complex $\spc{S}$ and all its links satisfy 
the no-triangle condition.
It follows that $\spc{S}$ includes a 2-simplex for each triangle.
Applying the same observation for each edge we get that $\spc{S}$ 
includes a 3-simplex for any complete graph with 4 vertices.
Repeating this observation 
for triangles, 
4-simplexes,
5-simplices
and so on we get that $\spc{S}$ is flag.
\qeds


\parbf{All-right triangulation.} %???right-angled triangulation???
A triangulation of a spherical polyhedral space 
is called an  \emph{all-right triangulation} 
if each simplex of the triangulation is isometric 
to a spherical simplex all of whose angles are right.
Similarly, we say that a simplicial complex 
is equipped with an  \emph{all-right spherical metric}
if it is a length metric and each simplex is isometric 
to a spherical simplex all of whose angles are right.

Spherical polyhedral $\Cat{}{1}$ spaces glued from of right-angled simplices
admit the following characterization 
discovered by Gromov \cite[p. 122]{gromov:hyp-groups}.

\begin{thm}{Flag condition}\label{thm:flag}
Assume that a spherical polyhedral space $\spc{P}$
admits an all-right triangulation $\tau$.
Then $\spc{P}$ is a $\Cat{}{1}$ space
if and only if $\tau$ is flag.
\end{thm}

\parit{Proof; ``only-if'' part.} 
Assume there are three vertices $v^1,v^2$ and $v^3$ of $\tau$
that are pairwise joined by edges 
but do not span a simplex.
Note that in this case 
$$\mangle\hinge{v^1}{v^2}{v^3}=\mangle\hinge{v^2}{v^3}{v^1}=\mangle\hinge{v^3}{v^1}{v^1}=\pi.$$
Equivalently,
\begin{clm}{}\label{clm:3pi/2}
The join of the geodesics $[v^1v^2]$, $[v^2v^3]$ and $[v^3v^1]$
forms a locally geodesic loop in $\spc{P}$. 
\end{clm}

Now assume that $\spc{P}$ is a $\Cat{}{1}$ space.
Then by Theorem~\ref{thm:PL-CAT},
$\Link_\sigma\spc{P}$ is a $\Cat{}{1}$ space for every simplex $\sigma$ 
in $\tau$. 

Each of these links is an all-right spherical complex
and
by Theorem \ref{thm:PL-CAT}, 
none
of these links can contain a geodesic circle of length less than $2\cdot\pi$. 

Therefore Proposition~\ref{prop:no-trig} and \ref{clm:3pi/2} 
imply the ``only-if'' part.

\parit{``If'' part.} 
By Lemma~\ref{lem:link-of-flag} and Theorem~\ref{thm:PL-CAT},
it is sufficient to show that any closed local geodesic $\gamma$ 
in a flag complex $\spc{S}$ with all-right metric has length at least $2\cdot\pi$.

%???+PIC

Fix a flag complex $\spc{S}$.
Recall that the  \index{star of vertex}\emph{star} of a vertex $v$ (briefly $\overline \Star_v$)
is formed by all the simplices  containing $v$. Similarly, $\Star_v$,   the open star of $v$, is the union of all simplices containing $v$ with faces opposite $v$ removed.

Choose a simplex $\sigma$ which contains a point of $\gamma$.
Let $v$ be a vertex of $\sigma$.
Set $f(t)=\cos\dist{v}{\gamma(t)}{}$.
Note that 
\[f''(t)+f(t)=0\] if $f(t)>0$.  
Since the zeroes of $f$ are  $\pi$ apart,
$\gamma$ 
spends time $\pi$ on every visit to $\Star_v$.

After leaving $\Star_v$,
the local geodesic $\gamma$ has to enter another simplex, 
say $\sigma'$, 
which has a vertex $v'$ not joined to $v$ by an edge.

Since $\tau$ is flag, we have that the stars $\Star_v$ and $\Star_{v'}$
do not overlap.
The same argument as above shows that $\gamma$ spends time $\pi$ on every visit to $\Star_{v'}$.
Therefore the total length of $\gamma$ is at least $2\cdot\pi$.
\qeds

\begin{thm}{Exercise}\label{ex:flag>=pi/2}
Assume that a spherical polyhedral space $\spc{P}$
admits a triangulation $\tau$ such that all dihedral angles of all simplexes are at least $\tfrac\pi2$.
Show that $\spc{P}$ is a $\Cat{}{1}$ space
if $\tau$ is flag.
\end{thm}

\begin{thm}{Exercise}\label{ex:short-retracts}
Let $\spc{U}\in \Cat{}{0}$
and $\phi^1,\phi^2,\dots,\phi^k\:\spc{U}\to \spc{U}$ be commuting short retractions; 
that is 
\begin{itemize}
\item $\phi^i\circ\phi^i=\phi^i$ for each $i$;
\item $\phi^i\circ\phi^j=\phi^j\circ\phi^i$ for any $i$ and $j$;
\item $\dist{\phi^i(x)}{\phi^i(y)}{\spc{U}}\le \dist{x}{y}{\spc{U}}$ for each $i$ and any $x,y\in\spc{U}$.
\end{itemize}
Set $A^i=\Im \phi^i$ for all $i$;
note that each $A^i$ is a weakly convex set.

Assume $\Gamma$ is a finite graph 
(without loops and multiple edges) 
with edges labeled by $1,2,\dots, n$.
Denote by $\spc{U}^\Gamma$ the space obtained by taking 
a copy of $\spc{U}$ for each vertex of $\Gamma$ and 
gluing two such copies along $A^i$ if the corresponding vertices are joint by an edge labeled by $i$.

Show that $\spc{U}^\Gamma$ is a $\Cat{}{0}$ space
\end{thm}

\parbf{The space of trees.}
The following construction is given by Billera, Holmes and Vogtmann in \cite{BHV}.

Let $\spc{T}_n$ be the set of all metric trees with $n$ end-vertices
labeled by $a^1,\dots,a^n$.
To describe one tree in $\spc{T}_n$ we may fix a topological tree $\tau$ with end vertices $a^1,\dots,a^n$ and all the other verices of degree 3 
and prescribe the lengths of $2\cdot n-3$ edges.
If the lengh of an edge is $0$, we assume that edge degenerates;
such a tree can be also decribed using a different topological tree $\tau'$.
The subset of $\spc{T}_n$ corresponding to the given topological tree $\tau$ can be identified with a convex closed cone in  $\mathbb{R}^{2\cdot n-3}$.
Equip each such subset with the metric induced from $\mathbb{R}^{2\cdot n-3}$ and consider the legth metric on $\spc{T}_n$ induced by these metrics.

\begin{thm}{Exercise}\label{ex:tree}
Show that $\spc{T}_n$ with the described metric is a $\Cat{}0$ space.
\end{thm}



\section{Cubical complexes}

The definition of a cubical complex
mostly repeats the definition of a simplicial complex, 
with simplices replaced by cubes.

Formally, a \index{cubical complex}\emph{cubical complex} is defined as a subcomplex 
of the unit cube in the Euclidean space of large dimension;
that is, a collection of faces of cube
such that together with each face it contains all its sub-faces.
Each cube face in this collection 
will be called a \emph{cube} of the cubical complex.

Note that according to this definition, 
any cubical complex is finite,
that is, contains a finite number of cubes.

The union of all the cubes in a cubical complex $\spc{Q}$ will be called its \emph{underlying space};
it will be denoted by $\spc{Q}$ or by $\ushort{\spc{Q}}$ 
if we need to emphasize that we are talking about a set, 
not a complex.
A homeomorphism from $\ushort{\spc{Q}}$ to a topological space $\spc{X}$ is called a \index{cubulation}\emph{cubulation of} $\spc{X}$.

The underlying space of a cubical complex $\spc{Q}$ will be always considered with the length metric
induced from $\RR^N$.
In particular, with this metric, 
each cube of $\spc{Q}$ is isometric to the unit cube of the same dimension.

It is straightforward to construct a triangulation 
of $\ushort{\spc{Q}}$ 
such that each simplex is isometric to a Euclidean simplex.
In particular $\ushort{\spc{Q}}$ is a Euclidean polyhedral space.

The link of each cube in a cubical complex admits a natural 
all-right triangulation; 
each simplex corresponds to an adjusted cube.

\parbf{Cubical analog of a simplicial complex.}
Let $\spc{S}$ be a simplicial complex and $\{v_1,\dots,v_N\}$ be the set of its vertexes.

Consider $\RR^N$ with the standard basis $\{e_1,\dots,e_N\}$.
Denote by $\square^N$ the standard unit cube in $\RR^N$;
that is 
\[\square^N=\set{(x_1,\dots,x_N)\in \RR^N}{0\le x_i\le 1\ \text{for each}\ i}.\]

Given a $\kay$-dimensional simplex $\<v_{i_0},\dots,v_{i_\kay}\>$ in $\spc{S}$, 
mark the $(\kay\z+1)$-dimensional faces in $\square^N$ (there are  $2^{N-\kay}$ of them)
which are parallel to the coordinate $(k+1)$-plane 
spanned by $e_{i_0},\dots,e_{i_\kay}$.


Note that the set of all marked faces of $\square^{N}$
forms a cubical complex;
it will be called 
the \index{cubical analog}\emph{cubical analog} of $\spc{S}$
and will be denoted as $\square_\spc{S}$.

Note that if a simplicial complex is connected then so is its cubical analog.

\begin{thm}{Proposition}\label{prob:cubical-analog}
Let $\spc{S}$ be a connected simplicial complex
and $\spc{Q}=\square_{\spc{S}}$ be its cubical analog.
Then $\ushort{\spc{Q}}$ is connected 
and the link of any vertex of $\spc{Q}$
is isometric to  ${\spc{S}}$
equipped with the spherical right-angled metric.

In particular, if $\spc{S}$ is a flag complex 
then $\spc{Q}$ is a locally $\Cat{}{0}$
and therefore its universal cover $\tilde{\spc{Q}}$ is a $\Cat{}{0}$ space.
\end{thm}

\parit{Proof.}
The first part of the proposition follows 
from the construction above.

If ${\spc{S}}$ is flag, 
then by Flag condition (\ref{thm:flag}) 
the link of any cube in $\spc{Q}$ is a $\Cat{}{1}$ space.
Therefore, by cone construction (Exercise \ref{ex:cone+susp})
$\spc{Q}$
is locally $\Cat{}{0}$ space.
It remains to apply Globalization theorem 
\ref{thm:hadamard-cartan}.
\qeds

From Proposition \ref{prob:cubical-analog}, 
it follows that the cubical analog
of any flag complex is aspherical.
The following exercise states that the  converse also holds, see \cite[5.4]{davis-survey}.

\begin{thm}{Exercise}\label{ex:flag-aspherical}
A simplicial complex is flag 
if and only if its cubical analog is aspherical.
\end{thm}

\section{Exotic aspherical manifolds}


By Globalization theorem (\ref{thm:hadamard-cartan})
any $\Cat{}{0}$ proper length space is contractible.
Therefore any complete locally $\Cat{}{0}$ proper length space 
is \emph{aspherical};
that is, they have contractible universal covers.
This observation can be used to construct examples of aspherical spaces. 

Let $\spc X$ be a proper topological space.
Recall that $\spc X$ is called 
\index{simply connected space at infinity}\emph{simply connected at infinity} 
if for any compact set $K\subset\spc X$
there is a bigger compact set $K'\supset K$
such that  $\spc X\backslash K'$ is path connected 
and any loop which lies in $\spc X\backslash K'$
is null-homotopic in  $\spc X\backslash K$.

Recall that path connected spaces are not empty by definition.
Therefore compact spaces are not simply connected at infinity.

The following statement was proved by Michael Davis in \cite{davis-noneuclidean}.

\begin{thm}{Proposition}\label{prop:aspherical}
For any  $m\ge 4$ there is a closed aspherical 
$m$-dimensional piecewise linear manifold
whose universal cover is not simply connected at infinity.

In particular, the universal cover of this manifold 
is not homeomorphic to the $m$-dimensional Euclidean space.
\end{thm}

The proof requires the following lemma.

\begin{thm}{Lemma}\label{lem:example-pi_infty}
Let $\spc{S}$ be a flag complex,
$\spc{Q}=\square_{\spc{S}}$ be its cubical analog
and $\tilde{\spc{Q}}$ be the universal cover of $\spc{Q}$.

Assume  $\tilde{\spc{Q}}$ is simply connected at infinity.
Then $\spc{S}$ is simply connected.
\end{thm}

\parit{Proof of Lemma~\ref{lem:example-pi_infty}.}
Assume $\spc{S}$ is not simply connected.
Choose a shortest noncontractible circle $\gamma\:\mathbb{S}^1\to\spc{S}$ formed by the edges of $\spc{S}$.

Note that $\gamma$ forms a 1-dimensional subcompelex of $\spc{S}$ which is a closed local geodesic.
Denote by $G$ the subcomplex of $\spc{Q}$ which corresponds to $\gamma$.

Fix a vertex $v\in G$;
let $G_v$ be the connected component of $G$ containing $v$.
Let $\tilde G$ be 
the inverse image 
of $G_v$ in $\tilde{\spc{Q}}$
for the universal cover $\tilde{\spc{Q}}\to \spc{Q}$.
Fix a point $\tilde v\in\tilde G$ in the inverse image of $v$.
 
Note that 
\begin{clm}{}\label{tilde-G-convex}
$\tilde G$ forms a convex set in $\tilde{\spc{Q}}$.
\end{clm}


Indeed, according to Proposition \ref{prob:cubical-analog},
$\tilde{\spc{Q}}$ is $\Cat{}{0}$.
By Exercise \ref{ex:locally-convex}???,
it is sufficient to show that $\tilde G$ is locally convex in $\tilde{\spc{Q}}$,
or equivalently $G$ is locally convex in $\spc{Q}$.

\begin{wrapfigure}[5]{r}{30mm}
\begin{lpic}[t(-7mm),b(0mm),r(0mm),l(0mm)]{pics/gamma-in-S(1)}
\lbl[t]{15,5;$\xi$}
\lbl[b]{15,17;$\zeta$}
\lbl[r]{14.5,10.5;$e$}
\end{lpic}
\end{wrapfigure}

The latter can only fail if $\gamma$ passes through two vertices, say $\xi$ to $\zeta$ in $\spc{S}$,
which are joined by an edge not in $\gamma$; 
denote this edge by $e$.

Each edge of $\spc{S}$ has length $\tfrac\pi2$.
Therefore each of two circles formed by $e$ and an arc of $\gamma$
from $\xi$ to $\zeta$ is shorter that $\gamma$.
Moreover,
at least one of them is noncontractable 
since $\gamma$ is not.
That is, 
$\gamma$ is not a shortest noncontractible circle 
--- a contradiction.
\claimqeds

Further, note that 
$\tilde G$ is homeomorphic to the plane, 
since $\tilde G$ is 
a 2-dimensional manifold without boundary which 
by the above is $\Cat{}{0}$ and hence is contractible.

Denote by $C_R$ the circle of radius $R$ in $\tilde G$ centered at $\tilde v$.
All $C_R$ are homotopic to each other in $\tilde G\backslash\{\tilde v\}$ and therefore in $\tilde{\spc{Q}}\backslash \{\tilde v\}$.

Note that the map $\tilde{\spc{Q}}\backslash \{\tilde v\}\to \spc{S}$
defined by $x\mapsto\dir{\tilde v}{x}$ maps $C_R$ to a circle homotopic to $\gamma$.
Therefore $C_R$ is not contractible in $\tilde{\spc{Q}}\backslash \{\tilde v\}$.

In particular, 
$C_R$ is not contactable in $\tilde{\spc{Q}}\backslash K$
if $K\supset \tilde v$.
If $R$ is large, 
the circle $C_R$  
lies outside of any compact set $K'$ in $\tilde{\spc{Q}}$.
It follows that $\tilde{\spc{Q}}$ is not simply connected at infinity, a contradiction.
\qeds

\begin{thm}{Exercise}\label{ex:example-pi_infty-new}
Under the assumptions of the Lemma~\ref{lem:example-pi_infty}, 
for any vertex $v$ in $\spc{S}$
the complement $\spc{S}\backslash\{v\}$ is simply connected.
\end{thm}

\parit{Proof of Proposition~\ref{prop:aspherical}.}
Let $\Sigma^{m-1}$ be an $(m-1)$-dmensional smooth homology sphere which is not simply connected and bounds a contractible smooth compact $m$-dimensional manifold $\spc{W}$. 

For $m\ge 5$ the existence of such $(\spc{W}, \Sigma)$ follows from \cite{kervaire}. 
For $m=4$ it follows from the construction in \cite{mazur}.

Pick any smooth triangulation $\tau$ of $W$ and let $\spc{S}$ be the resulting subcomplex which triangulates $\Sigma$.


We can assume that $\spc{S}$ is flag; 
otherwise pass to the barycentric subdivision 
of $\tau$ and apply Exercise~\ref{ex:baricenric-flag}.


Let $\spc{Q}=\square_{\spc{S}}$ be the cubical analog of $\spc{S}$.

By Proposition~\ref{prob:cubical-analog},
$\spc{Q}$ is a homology manifold.
It follows that $\spc{Q}$ is a piecewise linear manifold 
with a finite number of singularities at its vertices.


Removing a small neighborhood of each vertex in $\spc{Q}$,
we can obtain a piecewise linear manifold 
which boundary is formed by several copies of $\Sigma$.
To each copy of $\Sigma$, 
glue a copy of  $\spc{W}$ along its boundary.
This results in a  closed piecewise linear manifold 
$\spc{M}$ which is homotopically equivalent to $\spc{Q}$.

Finally, by the Lemma~\ref{lem:example-pi_infty},  
the universal cover $\tilde{\spc{Q}}$ of $\spc{Q}$
is not simply connected at infinity.

The same holds for 
the universal cover $\tilde{\spc{M}}$ of $\spc{M}$.
The later follows since homotopy equivalences 
$f\: \spc Q\to \spc M$ and $g\:\spc M\to \spc Q$ 
lift to proper maps 
$\tilde f \: \tilde{\spc{Q}}\to \tilde{\spc{M}}$
and $\tilde g \: \tilde{\spc{M}}\to \tilde{\spc{Q}}$. 
\qeds


\section{Comments}

The construction used in the proof of  Proposition~\ref{prop:aspherical} does many other interesting thins; 
some of them discussed in the survey \cite{davis-survey} by Michael Davis.

The following Proposition was proved by Fredric Ancel, Michael Davis and Craig Guilbault in \cite{ADG}. 


\begin{thm}{Proposition}\label{prop:loc-CAT-mnfld}
Given $m\ge 5$ there is a Euclidean polyhedral space $\spc{P}$ such that
\begin{subthm}{}
$\spc{P}$ is homeomorphic to a closed $m$-dimensional manifold.
\end{subthm}

\begin{subthm}{}
$\curv\spc{P}\le 0$
\end{subthm}

\begin{subthm}{}
The universal cover of $\spc{P}$ is not simply connected at infinity.
\end{subthm}
\end{thm}

\parit{Proof.}
Apply Exercise~\ref{ex:example-pi_infty-new} to the barycentric subdivision of the simplicial complex $\spc{S}$ provided by Exercise~\ref{ex:funny-S}.
\qeds

\begin{thm}{Exercise}\label{ex:funny-S}
Given a positive integer $m\ge 5$
construct an $(m-1)$-dimensional simplicial complex $\spc{S}$ such that $\Cone\spc{S}$ is homeomorphic to $\EE^m$
and $\pi_1(\spc{S}\backslash\{v\})\ne0$ for some vertex $v$ in $\spc{S}$.
\end{thm}


It worth to note that any complete simply connected Riemannian manifold with nonpositive curvature is homeomorphic to the Euclidean space of the same dimension.
In fact by Globalization theorem
(\ref{thm:hadamard-cartan}), 
the exponential map at one point of such manifold is a homeomorphism.
In particular there is no Riemannian analog of the proposition.
Moreover, according to a theorem of Stone, see \cite{stone, davis-januszkiewicz}, there is no piecewise linear analog of the proposition; 
that is the homeomorphism to a manifold in Proposition~\ref{prop:loc-CAT-mnfld} 
can not be made piecewise linear. 

A similar construction was used by Michael Davis, Tadeusz Januszkiewicz and Jean-Fran\c{c}ois Lafont \cite{davis-januszkiewicz-lafont}
to construct a closed smooth 4-dimensional manifold $M$ with universal cover $\tilde M$ diffeomorphic to $\RR^4$ such that (1) $M$ admits a polyhedral metric which is locally $\Cat{}{0}$ but it does not admit a Riemannian metric with nonpositive sectional curvature.

The construction described in this chapter also lead to so called \emph{hyperbolization}, a flexible tool to construct a aspherical spaces;
\cite{charney-davis} is a good survey in the subject.



