\mainmatter
\chapter{Manifesto}

Consider the space $\mathcal{M}_4$ of all isometry classes of 4-point metric space.
Each element in $\mathcal{M}_4$ can be described by 6 numbers 
 --- the distances between all 6 pairs of its points, say $\ell_{i,j}$ for $1\le i< j\le 4$ modulo permutations of $(1,2,3,4)$.
These 6 numbers is subject to 12 triangle inequalities; that is,
\[\ell_{i,j}+\ell_{j,k}\ge \ell_{i,k}\]
holds for all $i$, $j$ and $k$; here we assume that $\ell_{j,i}=\ell_{i,j}$ and $\ell_{i,i}=0$. 

Consider the subset $\mathcal{E}_4\subset \mathcal{M}_4$ of all isometry classes of metric spaces which admit an isometric embedding into Euclidean space.
The complement $\mathcal{M}_4\backslash \mathcal{E}_4$ has two connected components.

\begin{thm}{Exercise}
Prove the latter statement.
\end{thm}


One of the component will be denoted by $\mathcal{P}_4$ and the other by $\mathcal{N}_4$,
here $\mathcal{P}$ stays for \emph{positive} 
and $\mathcal{N}$ stays for \emph{negative}.
Say, assume $[x^1x^2x^3]$ is an equilateral triangle in the plane 
and $x^4$ be the midpoint of side $[x^2x^3]$.
Define the distances $\ell_{i,j}=\dist{x^i}{x^j}{\EE^2}$ for all $i<j$ except $\ell_{1,4}$.
\begin{itemize}
\item If one takes $\ell_{1,4}=\dist{x^1}{x^4}{\EE^2}$ then the obtained class of the metric space belongs to $\mathcal{E}_4$.
\item If one takes $\ell_{1,4}$ slightly smaller than $\dist{x^1}{x^4}{\EE^2}$ then the obtained class of the metric space belongs to $\mathcal{N}_4$.
\item If one takes $\ell_{1,4}$ slightly bigger than $\dist{x^1}{x^4}{\EE^2}$ then the obtained class of the metric space belongs to $\mathcal{P}_4$.
\end{itemize}



If a space with an intrinsic metric has no quadruples of points of type $\mathcal{P}_4$ or $\mathcal{N}_4$ 
are called Alexandrov space with non-positive or non-negative curvature correspondingly.

These two conditions could be understood as 4-point analog of triangle inequality.

Here is an exercise, solving which would force the reader to rebuild considrable part of the theory.

\begin{thm}{Exercise}
Assume $\spc{X}$ is a complete metric space with intrinsic metric
which contains only quadruples of type $\mathcal{E}_4$.
Show that $\spc{X}$ is isometric to the convex set in a Hilbert space.
\end{thm}

In fact it would be helpful to think about this exercise for a couple of days before you proceed on reading.

If instead of Euclidean space, one takes hyperbolic space,
one arrives to the definition of spaces with curvature bounded above or below by $-1$.
To define spaces with curvature bounded above or below by $1$
One has to take unit 3-dimesional sphere instead,
and allow to check only quadrouples of points such that each of the 4 triangles has perimeter at most $2\cdot\pi$.
The later condition could be thought as a part of \emph{spherical triangle inequality}.

\medskip

These lectures were a part of the mini-course in Summer School ``Algebra and Geometry'' held on 25--31 June 2016 in Yaroslavl, Russia.







