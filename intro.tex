%%!TEX root = invitation-springer.tex
\mainmatter

\chapter*{Preface}

These notes arise as an offshoot 
of the book on Alexandrov geometry we have been writing for  a number of years.
The notes were shaped in a number of lectures given by the third author
to undergraduate students 
at different  occasions at the
MASS program at Penn State University
and the Summer School ``Algebra and Geometry'' in Yaroslavl.

The idea is to demonstrate the beauty and power of Alexandrov geometry by reaching interesting applications and theorems with a minimum of preparation.

\medskip 

In Chapter~\ref{prelim}, we discuss necessary preliminaries.

In Chapter~\ref{chapter:gluing}, we discuss the Reshetnyak gluing theorem and apply it to a problem in billiards which was solved by Dmitri Burago, Serge Ferleger and Alexey Kononenko.

In Chapter~\ref{chapter:globalization}, we discuss the Hadamard--Cartan globalization theorem, and apply it to the construction of exotic aspherical manifolds introduced by Michael Davis.

In Chapter~\ref{chapter:shefel}, we discuss examples of Alexandrov spaces with curvature bounded above.
This chapter is based largely on work of Samuel Shefel on nonsmooth saddle surfaces.

\medskip

Here is a list of some sources providing a good introduction to Alexandrov spaces with curvature bounded above, which we recommend for further information;
we will not assume familiarity with any of these sources.

\begin{itemize}
\item The book by Martin Bridson and Andr\'e Haefliger \cite{BH};
\item Lecture notes of Werner Ballmann \cite{ballmann:lectures};
\item Chapter 9 in the book \cite{BBI} by Dmitri Burago, Yurii Burago and Sergei Ivanov.
\end{itemize}



\section*{Early history of Alexandov geometry}


The idea that the essence of curvature lies in a condition on quadruples of points apparently originated with Abraham Wald.
It is found in his publication on ``coordinate-free differential geometry'' \cite{wald} written under the supervision of Karl Menger;
the story of this discovery can be found in \cite{menger}.
In 1941, similar definitions were rediscovered independently by 
Alexandr Danilovich Alexandrov, %Alexandr is the right spelling
see~\cite{alexandrov:def}.
In Alexandrov's work the first fruitful applications of this approach were given.
Mainly:
\begin{itemize}
\item Alexandrov's embedding theorem --- 
\textit{metrics of non-negative curvature on the sphere, and only they, are isometric to closed convex surfaces in Euclidean 3-space}. 
\item Gluing theorem, which tells  when the sphere obtained by gluing of two discs along their boundaries has non-negative curvature in the sense of Alexandrov.
\end{itemize}
These two results together gave  a very intuitive geometric tool for studying  embeddings and bending of surfaces in  Euclidean space, and changed this subject dramatically.
They formed the foundation of the branch of geometry now called \emph{Alexandrov geometry}.

The study of  spaces with curvature bounded above started later.
The first paper on the subject was written by Alexandrov; it appeared in 1951, see \cite{alexandrov:strong-angle}.
It was based on work of Herbert Busemann, who studied spaces satisfying a weaker condition \cite{busemann-CBA}.

Yurii Grigorievich Reshetnyak proved fundamental results about general spaces with curvature bounded above, the most important of which is his gluing theorem.
An equally important theorem is the Hadamard--Cartan theorem (globalization theorem).
These theorems and their history are discussed in chapters \ref{chapter:gluing} and~\ref{chapter:globalization}.

Surfaces with upper curvature bounds were studied extensively in the 50-s and 60-s, and are by now well understood; see the survey \cite{reshetnyak:survey} and the references therein.


\section*{Manifesto of Alexandrov geometry}

Alexandrov geometry can use ``back to Euclid'' as a slogan.
Alexandrov spaces are defined via axioms similar to those given by Euclid,
but certain  equalities are changed to inequalities. 
Depending on the sign of the inequalities we get Alexandrov spaces with \emph{curvature bounded above} or \emph{curvature bounded below}.
The definitions of the two classes of spaces are similar, but their properties and known applications are quite different.


Consider the space $\mathcal{M}_4$ of all isometry classes of 4-point metric spaces.
Each element in $\mathcal{M}_4$ can be described by 6 numbers 
 --- the distances between all 6 pairs of its points, say $\ell_{i,j}$ for $1\le i< j\le 4$ modulo permutations of the index set $(1,2,3,4)$.
These 6 numbers are subject to 12 triangle inequalities; that is,
\[\ell_{i,j}+\ell_{j,k}\ge \ell_{i,k}\]
holds for all $i$, $j$ and $k$, where we assume that $\ell_{j,i}=\ell_{i,j}$ and $\ell_{i,i}=0$.

\begin{wrapfigure}[8]{r}{53mm}
\begin{lpic}[t(-4mm),b(-0mm),r(0mm),l(0mm)]{pics/NEP(1)}
\lbl{8,12;$\mathcal{N}_4$}
\lbl{26,12;{\color{white} $\mathcal{E}_4$}}
\lbl{42,12;$\mathcal{P}_4$}
\lbl[t]{26,0;$\mathcal{M}_4$}
\end{lpic}
\end{wrapfigure}

Consider the subset $\mathcal{E}_4\subset \mathcal{M}_4$ of all isometry classes of 4-point metric spaces that admit isometric embeddings into Euclidean space.
The complement $\mathcal{M}_4\backslash \mathcal{E}_4$ has two connected components.

\begin{thm}{Exercise}\label{ex:two-components-of-M4}
Prove the latter statement.
\end{thm}


One of the components will be denoted by $\mathcal{P}_4$ and the other by~$\mathcal{N}_4$.
Here $\mathcal{P}$ and $\mathcal{N}$ stand for {}\emph{positive} 
and {}\emph{negative curvature} because spheres have no quadruples of type $\mathcal{N}_4$ and 
hyperbolic space
has no quadruples of type~$\mathcal{P}_4$.

A metric space, with length metric, 
that has no quadruples of points of type $\mathcal{P}_4$ or $\mathcal{N}_4$
respectively 
is called an Alexandrov space with non-positive or non-negative curvature respectively.

Here is an exercise, solving which would force the reader to rebuild a considerable part of Alexandrov geometry.

\begin{thm}{Advanced exercise}\label{ex:convex-set}
Assume $\spc{X}$ is a complete metric space with length metric, 
containing only quadruples of type~$\mathcal{E}_4$.
Show that $\spc{X}$ is isometric to a convex set in a Hilbert space.
\end{thm}

In fact, it might be helpful to spend some time thinking about this exercise before proceeding.

In the definition above, 
instead of  Euclidean space 
one can take  
hyperbolic space of curvature~$-1$.
In this case,
one obtains the definition of spaces with curvature bounded above or below by~$-1$.

To define spaces with curvature bounded above or below by $1$,
one has to take the unit 3-sphere 
and specify that only the quadruples of points such that each of the four triangles has perimeter 
less than $2\cdot\pi$ are checked.
The latter condition could be considered as a part of the {}\emph{spherical triangle inequality}.


\section*{Acknowledgment}
We want to  thank 
David Berg,
Richard Bishop,
Yurii Burago,
Maxime Fortier Bourque,
Sergei Ivanov,
Michael Kapovich, 
Bernd Kirchheim, 
Bruce Kleiner,
Nikolai Kosovsky,
Greg Kuperberg,
Nina Lebedeva,
John Lott,
Alexander Lytchak,
Dmitri Panov,
Stephan Stadler,
Wilderich Tuschmann
and 
Sergio Zamora Barrera
for a number of discussions and suggestions.

We thank the mathematical institutions where we worked on this material,  including
BIRS, 
MFO, 
Henri Poincar\'{e} Institute,
University of Colone, 
Max Planck Institute for Mathematics. 

The first author was partially supported by  the Simons Foundation grant \#209053.
The second author was partially supported by a Discovery grant  from NSERC and by the Simons Foundation grant \#390117.
The third author was partially supported by the NSF grant DMS 1309340 and the Simons Foundation \#584781.

