\mainmatter

\chapter*{Instead of foreword}

These notes arise as a coproduct of the book on Alexandrov geometry
we have been writing for  a number of years.
The notes were shaped in a number of lectures given by Anton Petrunin 
to undergraduate students 
at different  occasions at the
MASS program at Penn State University
and the Summer School ``Algebra and Geometry'' in Yaroslavl.

Here is a list of available introductions to the Alexandrov spaces with curvature bounded above: 
\begin{itemize}
\item A book of Martin Bridson and Andr\'e Haefliger \cite{BH};
\item Lecture notes of Werner Ballmann \cite{ballmann:lectures};
\item Chapter 9 in the book \cite{BBI} by Dmitri Burago, Yurii Burago and Sergei Ivanov.
\end{itemize}

\section*{Early history of Alexandov geometry}

The first synthetic description of curvature is due to Abraham Wald; 
it was given in a lonely publication on a ``coordinateless description of Gauss surfaces'' published in 1936, see \cite{wald}.
In 1941, similar definitions were rediscovered independently by Aleksandr Danilovich Aleksandrov,
see \cite{alexandrov:def}.
In Alexandrov's work the first fruitful applications of this approach were given.
Mainly:
\begin{itemize}
\item Alexandrov's embedding theorem  --- 
\textit{metrics of non-negative curvature on the sphere, and only they, are isometric to closed convex surfaces in Euclidean 3-space}. 
\item Gluing theorem, which tells when the sphere obtained by gluing of two discs has non-negative curvature in the sense of Alexandrov.
\end{itemize}
These two results together gave  a very intuitive geometric tool to study embeddings and bending of surfaces in Euclidean space, and changed this subject dramatically.
They formed the foundation of branch of geometry now called \emph{Alexandrov geometry}.

Alexandrov geometry can use ``back to Euclid'' as a slogan.
Alexandrov spaces are defined via axioms similar to the one given by Euclid,
but certain  equalities exchanged to inequalities. 
Depending on the sign of inequalities we get Alexandrov spaces with \emph{curvature bounded above} and \emph{curvature bounded below};
more on this is in the manifesto below.
Althogh the definitions of two classes of spaces are similar, their properties and known applications are qute different.


The study of the spaces with curvature bounded above started later.
The first paper on the subject was written by Alexandrov, it appeared in 1951, see \cite{alexandrov:strong-angle}.
An analogous weaker definition was considered earlier by Busemann in \cite{busemann-CBA}.

The fundamental results in this direction were obtained by Yurii Grigorievich Reshetnyak.
It includes his majorization theorem and gluing theorem.
The gluing theorem states that if two non-positively curved spaces have isometric convex sets, then the space obtained by gluing these sets along the isometry is also non-positively curved.

\section*{Manifesto of Alexandov geometry}


Consider the space $\mathcal{M}_4$ of all isometry classes of 4-point metric spaces.
Each element in $\mathcal{M}_4$ can be described by 6 numbers 
 --- the distances between all 6 pairs of its points, say $\ell_{i,j}$ for $1\le i< j\le 4$ modulo permutations of $(1,2,3,4)$.
These 6 numbers are subject to 12 triangle inequalities; that is,
\[\ell_{i,j}+\ell_{j,k}\ge \ell_{i,k}\]
holds for all $i$, $j$ and $k$, where we assume that $\ell_{j,i}=\ell_{i,j}$ and $\ell_{i,i}=0$.

\begin{wrapfigure}[8]{r}{53mm}
\begin{lpic}[t(0mm),b(-0mm),r(0mm),l(0mm)]{pics/NEP(1)}
\lbl{8,12;$\mathcal{N}_4$}
\lbl{26,12;$\mathcal{E}_4$}
\lbl{42,12;$\mathcal{P}_4$}
\lbl[t]{26,0;$\mathcal{M}_4$}
\end{lpic}
\end{wrapfigure}

Consider the subset $\mathcal{E}_4\subset \mathcal{M}_4$ of all isometry classes of 4-point metric spaces which admit isometric embeddings into Euclidean space.
The complement $\mathcal{M}_4\backslash \mathcal{E}_4$ has two connected components.

\begin{thm}{Exercise}\label{ex:two-components-of-M4}
Prove the latter statement.
\end{thm}


One of the components will be denoted by $\mathcal{P}_4$ and the other by $\mathcal{N}_4$.
Here $\mathcal{P}$ and $\mathcal{N}$ stand for {}\emph{positive} 
and {}\emph{negative curvature} because spheres have no quadruples of type $\mathcal{N}_4$ and 
hyperbolic space
has no quadruples of type $\mathcal{P}_4$.

A metric space, with length metric, 
which has no quadruples of points of type $\mathcal{P}_4$ or $\mathcal{N}_4$ 
is called an Alexandrov space with non-positive or non-negative curvature respectively.

Here is an exercise, solving which would force the reader to rebuild a considerable part of Alexandrov geometry.

\begin{thm}{Advanced exercise}\label{ex:convex-set}
Assume $\spc{X}$ is a complete metric space with length metric, 
containing only quadruples of type $\mathcal{E}_4$.
Show that $\spc{X}$ is isometric to a convex set in a Hilbert space.
\end{thm}

In fact, it might be helpful to spend some time thinking about this exercise before proceeding.

In the definition above, 
instead of  Euclidean space 
one can take  
hyperbolic space of curvature $-1$.
In this case,
one obtains the definition of spaces with curvature bounded above or below by $-1$.

To define spaces with curvature bounded above or below by $1$,
one has to take the unit 3-sphere 
and specify that only the quadruples of points such that each of the four triangles has perimeter 
%??? less than?
 at most $2\cdot\pi$ are checked.
The latter condition could be thought as a part of the {}\emph{spherical triangle inequality}.


\section*{Acknowledgment}
We want to thank 
Richard Bishop,
Yurii Burago,
Sergei Ivanov,
Michael Kapovich, 
Bernd Kirchheim, 
Bruce Kleiner,
Nikolai Kosovsky
Greg Kuperberg,
Nina Lebedeva,
John Lott,
Alexander Lytchak,
Stephan Stadler
and 
Wilderich Tuschmann
%WHO ELSE???
for a number of discussions and suggestions.


%Yet special thanks to our non-mathematicician friends and relatives M.~Prelovskaya, J.~Tuschamnn, F.~Champong???; they made for us food, provide place to stay and did not ask stupid questions while this book was written.

We want to thank the mathematical institutions where we worked on this text inculding
BIRS, 
MFO, 
Henri Poincar\'{e} Institute,
University of Colone, 
Max Planck Institute for Mathematics.
%what else????










