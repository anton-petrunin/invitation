\mainmatter
\chapter*{Manifesto}

Consider the space $\mathcal{M}_4$ of all isometry classes of 4-point metric space.
Each element in $\mathcal{M}_4$ can be described by 6 numbers 
 --- the distances between all 6 pairs of its points, say $\ell_{i,j}$ for $1\le i< j\le 4$ modulo permutations of $(1,2,3,4)$.
These 6 numbers is subject to 12 triangle inequalities; that is,
\[\ell_{i,j}+\ell_{j,k}\ge \ell_{i,k}\]
holds for all $i$, $j$ and $k$; here we assume that $\ell_{j,i}=\ell_{i,j}$ and $\ell_{i,i}=0$.

\begin{wrapfigure}[8]{r}{53mm}
\begin{lpic}[t(0mm),b(-0mm),r(0mm),l(0mm)]{pics/NEP(1)}
\lbl{8,12;$\mathcal{N}_4$}
\lbl{26,12;$\mathcal{E}_4$}
\lbl{42,12;$\mathcal{P}_4$}
\lbl[t]{26,0;$\mathcal{M}_4$}
\end{lpic}
\end{wrapfigure}

Consider the subset $\mathcal{E}_4\subset \mathcal{M}_4$ of all isometry classes of metric spaces which admit an isometric embedding into Euclidean space.
The complement $\mathcal{M}_4\backslash \mathcal{E}_4$ has two connected components.

\begin{thm}{Exercise}\label{ex:two-components-of-M4}
Prove the latter statement.
\end{thm}


One of the component will be denoted by $\mathcal{P}_4$ and the other by $\mathcal{N}_4$.
Here $\mathcal{P}$ and $\mathcal{N}$ stay for \emph{positive} 
and \emph{negative curvature} because spheres have no quadruples of type $\mathcal{N}_4$ and hyperbolic plane has no quadruples of type $\mathcal{P}_4$.

A metric space with an intrinsic metric 
which has no quadruples of points of type $\mathcal{P}_4$ or $\mathcal{N}_4$ 
are called Alexandrov space with non-positive or non-negative curvature correspondingly.

Here is an exercise, solving which would force the reader to rebuild considerable part of the theory.

\begin{thm}{Advanced exercise}\label{ex:convex-set}
Assume $\spc{X}$ is a complete metric space with intrinsic metric
which contains only quadruples of type $\mathcal{E}_4$.
Show that $\spc{X}$ is isometric to the convex set in a Hilbert space.
\end{thm}

In fact it would be helpful to think about this exercise for a couple of days before you proceed on reading.

In the definition above, 
instead of Euclidean space, 
one can take the hyperbolic plane.
In this case
one arrives to the definition of spaces with curvature bounded above or below by $-1$.

To define spaces with curvature bounded above or below by $1$,
one has to take unit 3-sphere instead,
and allow to check only the quadruples of points such that each of the 4 triangles has perimeter at most $2\cdot\pi$.
The later condition could be thought as a part of \emph{spherical triangle inequality}.

\medskip
\noindent\rule{2cm}{0.4pt}

These notes arise as a coproduct of the book in Alexandrov geometry
we are writing for number of years.
The notes were shaped in number of lectures given by Anton Petrunin 
to undergraduate students 
at few occasions at the
MASS program at Penn State University
and Summer School ``Algebra and Geometry'' in Yaroslavl.

We want to thank 
Mikhail Kapovich, 
Alexander Lytchak,
Stephan Stadler
%WHO ELSE
for number discussions and suggestions.








