\part{Curvature bounded below}

\chapter{Isometric actions}

\section{4-point condition}

A quadruple of points $p,x,y,z$ in a metric space $\spc{X}$ satisfies 
$\CBB{}{0}$ comparison if
\[\angk pxy_{\EE^2}+\angk pyz_{\EE^2}+\angk pzx_{\EE^2}\le 2\cdot\pi.\]
If $p$ coincides with one of the points $x$, $y$ or $z$ then the left hand side is undefined;
in this case we assume that $\CBB{}{0}$ comparison holds automatically.

Taking hyperbolic plane $\HH^2$ or sphere $\SS^2$ instead of the Eulcidean plane $\EE^2$,
one defines $\CBB{}{-1}$ and $\CBB{}{1}$ comparison.

Namely, $p,x,y,z$  satisfies 
$\CBB{}{-1}$ comparison if 
\[\angk pxy_{\HH^2}+\angk pyz_{\HH^2}+\angk pzx_{\HH^2}\le 2\cdot\pi.\]

Further, $p,x,y,z$ satisfies $\CBB{}{1}$ comparison if 
\[\angk pxy_{\SS^2}+\angk pyz_{\SS^2}+\angk pzx_{\SS^2}\le 2\cdot\pi.\]
In the latter case the left hand side is undefined if $p$ coincides with 
one of $x$, $y$ or $z$ or one of the comparison triangles $\modtrig(pxy)_{\SS^2}$, $\modtrig(pyz)_{\SS^2}$ or $\modtrig(pzx)_{\SS^2}$ is undefined\footnote{That is 
\begin{align*}
\dist{p}{x}{}+\dist{p}{y}{}+\dist{x}{y}{}&\ge2\cdot\pi,\quad\text{or}
\\
\dist{p}{y}{}+\dist{p}{z}{}+\dist{y}{z}{}&\ge2\cdot\pi,\quad\text{or}
\\
\dist{p}{z}{}+\dist{p}{x}{}+\dist{z}{x}{}&\ge2\cdot\pi.
\end{align*}
};
in this case we also assume that $\CBB{}{0}$ comparison holds automatically.

We say that a metric space $\spc{L}$ is $\CBB{}{\kappa}$ if all quadruples of points in $\spc{L}$ satisfy $\CBB{}{\kappa}$ komaprison.

In the definition of spaces with curvature $\CBB{}{1}$, 
most authors assume in addition that $\diam \spc{L}\le \pi$. 
We do not make this assumption. 
In particular, the real line is a $\CBB{}{1}$ space for us.


\begin{thm}{Exercise}
Let $\spc{L}$ be a length space.
Show that $\spc{L}$ is a $\CBB{}0$ space
if and only if 
\[
\area\modtrig(xyz)_{\EE^2}
\le
\area\modtrig(pxy)_{\EE^2}+\area\modtrig(pyz)_{\EE^2}+\area\modtrig(pzx)_{\EE^2}
\]
for any 4 distinct points $p,x,y,z\in\spc{L}$.
\end{thm}

\begin{thm}{Exercise}\label{mink+alex=euclid} 
Assume that the metric $\rho$ on  $\RR^m$ is defined by a norm.
Show that $(\RR^m,\rho)$ is a $\CBB{}0$ space if and only if it is isometric to the Euclidean space.
\end{thm}

\section{Submetries}

A map $f\:\spc{X}\to\spc{Y}$ between metric spaces is called \index{submetry}\emph{submetry}
if it is 1-Lipscitz and 1-co-Lipschitz at the same time;
that is, if 
\[f(\oBall(x,R)_{\spc{X}})=\oBall(f(x),R)_{\spc{Y}}\]
for any $x\in \spc{X}$ and any value $R\ge 0$.

Note that according to the definition, any submetry is surjective.

\begin{thm}{Theorem}\label{thm:submetry}
Let $\spc{L}$ is a proper $\CBB{}\kappa$ space and $f\:\spc{L}\to\spc{N}$ is a submetry then $\spc{N}$ is a $\CBB{}\kappa$ space.
\end{thm}

Assume a group $G$ acts on a metric space $\spc{L}$ by isometries and each orbit of the action is closed.
Let us equip the space of $G$-orbits $\spc{L}/G$ with Hausdorff distance.
Then the quotient map $\spc{L}\to \spc{L}/G$ is a submetry.
Therfefore we get the following corollary of the theorem above.

\begin{thm}{Corollary}\label{cor:CBB/G}
Let $\spc{L}$ is a proper $\CBB{}\kappa$ space and the group $G$ acts on $\spc{L}$ by isometries 
and it has closed orbits.
Then the orbit space $\spc{L}/G$ is a $\CBB{}\kappa$ space. 
\end{thm}

\parit{Proof.}
Fix  a quadruple of points $p,x^1,x^2,x^3\in \spc{N}$.
Choose arbitrary $\hat p\in \spc{L}$ such that $f(\hat{p})=p$.

Since $f$ is a submetry, and $\spc{L}$ is proper,
we can choose the points $\hat{x}^1,\hat{x}^2,\hat{x}^3\in \spc{L}$.
such that $f(\hat x_i)=x_i$ and
\[\dist{\hat{p}}{\hat{x}^i}{\spc{L}}=
\dist{p}{x^i}{\spc{M}}\]
for all $i$.
Note that 
\[\dist{\hat{x}^i}{\hat{x}^j}{\spc{L}}
\ge
\dist{x^i}{x^j}{\spc{M}}\]
for all $i$ and $j$.
Therefore 
\[\angk {\hat{p}}{\hat{x}^1}{\hat{x}^2}
\le
\angk p{x^1}{x^2}
\eqlbl{eq:angles-M-L}\]
holds for all $i$ and $j$.

From $\CBB{}{\kappa}$ comparison in $\spc{L}$,
we have
\[\angk {\hat{p}}{\hat{x}^1}{\hat{x}^2}
+\angk {\hat{p}}{\hat{x}^2}{\hat{x}^3}
+\angk {\hat{p}}{\hat{x}^3}{\hat{x}^1}
\le 
2\cdot\pi\]
assuming all the angles on the left hand side are defined.
Applying  \ref{eq:angles-M-L}, 
we get 
\[\angk p{x^1}{x^2}
+\angk p{x^2}{x^3}
+\angk p{x^3}{x^1}
\le
2\cdot\pi.\]
\qedsf

%??? THE PROOF IS NOT CORRECT FOR KAPPA=1 SINCE THE COMPARISON IN L MIGHT HOLD BECAUSE THE PERIMERTER OF ONE TRIANGLE IS TOO BIG???

\section{Extremal sets}

Let $\spc{L}$ be a proper length $\CBB{}{\kappa}$ space.
A closed subset $A\subset\spc{L}$ is called \index{extremal set}\emph{extremal} if for any point $p\notin A$, 
any point $q\in A$ which minimize the distance to $p$ 
and any $x\ne q$ we have
\[\angk qpx\le \tfrac\pi2.\]

The whole space as well as the empty set are always extremal;
the other extremal sets are called \index{proper extremal set}\emph{proper}.

\begin{thm}{Exercise}
Show that a poin $p$ in a proper length $\CBB{}{\kappa}$ space forms an extremal set if and only if and only if $\diam\Sigma_p\le \tfrac\pi2$.
\end{thm}

\begin{thm}{Exercise}
Show that intersection and the union of two extremal sets is extremal. 
\end{thm}

Consider a Riemannian manifold $\spc{R}$ with sectional curvature $\ge \kappa$.
Toponiogov comparison theorem implies that $\spc{R}$ is a $\CBB{}{\kappa}$ space.

Let $G$ be a subgroup of the isometry group of $\spc{R}$.
We will deniote by $\spc{R}^G$ the set of points fixed by $G$.
Note that $\spc{R}^G$ is a totally geodesic subset in $\spc{R}$.

Given $p\in \spc{R}$ we will denote by $[p]_G$ the projection of $p$ to the orbit space $\spc{R}/G$.

If $G$ has closed orbits then according to Corollary~\ref{cor:CBB/G}, the orbit space $\spc{R}/G$
is a $\CBB{}{\kappa}$ space.
It opens a door for applicatuions comming from the isometric group actions.

Note that if the action $G\acts \spc{R}$ is free then the orbit space 
$\spc{R}/G$ is also a Riemannian manifold.
A typical action however has orbit space 
$\spc{R}/G$ with singularities.

\begin{thm}{Theorem}
Assume $\spc{R}$ is a Riemannian manifold with isometric action by a group $G$ with closed orbits.
Then for any subgroup $K<G$ the set $E_K=[\spc{R}^K]_G$ is an extremal subset in $\spc{R}/G$.
\end{thm}

From the construction of $E_K$ we get the following properties.
\begin{itemize}
\item If $K<L$ then $E_K\supseteq E_L$.
\item If $K=a\cdot L\cdot a^{-1}$ for some $a\in G$ then $E_K= E_L$.
\end{itemize}




\parit{Proof.}
Assume contrary; let $p\notin \spc{R}^G/G$, $q\in \spc{R}^G/G$ and $x$ be a triple of points in $\spc{R}/G$ violating the definition of extremal subset.
\qeds



\section{Erd\H{o}s' problem rediscovered}

Assume $\Gamma\acts \EE^m$ is a crystallographic action on the Euclidean space.
Note that every finite subgroup $F<\Gamma$ fix some affine subspace in $\EE^m$.
It follows that the corresponding extremal subset $E_F$ in $\spc{L}=\EE^m/\Gamma$ is proper --- not the whole space nor the empty set.

If $F$ is a maximal finite subgroup then the corresponging extremal set $E_F$ is primitive;
that is $E_F$ contains no other extreaml sets except itself and the empty set.
This disussion leads to the following observation.

\begin{thm}{Observation}
Let $\Gamma\acts \EE^m$ be a crystallographic action on the Euclidean space.
Then number $\#(\Gamma)$ of the maximal finite subgroups of $\Gamma$ up to conjugation 
equals to the number of primitive extreamal sets in the quatient space $\EE^m/\Gamma$.
\end{thm}

It expected that  $\#(\Gamma)\le 2^m$.
According to the observation above the latter would follow if 
the number of primitive extreaml sets in any $m$-dimensional compact length $\CBB{}{0}$ space does not exeed $2^m$.


It follows that in order to estimate number of maximal finite subgroups in $\Gamma$ it is sufficient to estimate the number of primitive extremal sets in an $m$-dimensional compact length $\CBB{}{0}$.

The following theorem was proved in ???.

\begin{thm}{Theorem}\label{thm:extr-point}
Let $\spc{L}$ is a $\CBB{}0$ space.
Then it has at most $2^m$ one-point extremal sets.
\end{thm}


The proof is a translation of proof of classical problem in combinatoric geometry to Alexandrov's language.

\begin{thm}{Erd\H{o}s' problem}
Let $F$ be a set of points in $\EE^m$ such that any triangle formed by three distinct points in $F$ has no obtuse angles.
Then number of elements in $F$ can not exeed $2^m$.

Moreover, if $|F|=2^m$ then $F$ consists of vertexes of right parallelepiped.
\end{thm}

This problem was posted by Paul Erd\H{o}s in \cite{erdos} and solved by Ludwig Danzer and Branko Gr\"unbaum in \cite{danzer-gruenbaum}.
The question of classifing all $\spc{L}$ with the maximal number one-point extremal sets turns out to be more delicate see \cite{lebedeva}.
Let us describe a class of examples.
Let $\TT^m$ be standart torus 
and $\Gamma$ be a subgroup generated by all coordinte reflections.
Clearly, $\Gamma$ is isomorphic to $(\ZZ/2\cdot\ZZ)^m$ and it has exactly $2^m$ fixed points in $\TT^m$.
Let $\Gamma'\subset\Gamma$ be a subgroup which has the same fixed point set as $\Gamma$ 
and $g$ be a flat $\Gamma'$-ivariant metric on $\TT^m$.
Then, $(\TT^m,g)/\Gamma'$ is a $\CBB{m}0$ space and it has exactly $2^m$ one-point extremal subsets; it follows from ??? and ???.
The above construction does not describes all $\CBB m0$ spaces with $2^m$ one-point extremal subsets, but it is close to the correct answer.

The finding resonable estimates for maximal number of extraml one-point subsets in a $\CBB m 1$ space is completely open;
it analogous to the follolwing problem: finding maximal number $n(m)$ such that there are points $p_1,p_2,\dots p_n$ in $\EE^m$ such that any triangle $\trig{p_i}{p_j}{p_\kay}$ is acute.
It is expected $n(m)\ll 2^m$, but so far it is only known that 
\begin{enumerate}
\item $n(m)\ge 2m$ --- the configuration s slight perturbation of vertexes of $m$-octahedra.
\item $n(m)\le ???$
\end{enumerate}


\parit{Proof of \ref{thm:extr-point}.}
Let $\{p_i\}$, $i\in\{1,2,\dots,N\}$ be the one-point extreaml sets.
For each $p_i$ consider its open Voronoi domain $V_i$; that is, 
\[V_i=\set{x\in \spc{L}}{\dist{p_i}{x}{}<\dist{p_j}{x}{}\ \t{for any}\ j\not=i}.\]
Clearly $V_i\cap V_j=\emptyset$ if $i\not=j$.
Note that $\vol_mV_i>\frac{1}{2^m}\vol_m \spc{L}$.

Indeed, fix $i$ and for given $\alpha\in(0,1)$, consider $\alpha$-homothety $\map_\alpha\:\spc{L}\to \spc{L}$ with center at $p_i$; 
that is, for each point $x\in \spc{L}$ choose a geodesic $[p_ix]$ and set
$\map_\alpha x=\geod_{[p_ix]}(\alpha\dist{p_i}{x}{})$.
From comparison we have that $\vol(\map_\alpha \spc{L})\ge\alpha^m\vol \spc{L}$.
For any $x\in \spc{L}$ and all $\alpha<\tfrac{1}{2}$ we have $\map_\alpha x\in V_i$.
Assume $x'=\map_\alpha x\notin V_i$,
then threre is $p_j$ such that $\dist{p_i}{x'}{}\ge\dist{p_j}{x'}{}$.
Then from comparison, we have $\angk{p_j}{p_i}{x}_{\EE^2}>\tfrac\pi2$;
that is, $p_j$ does not form a one-point extremal set.???
\qeds

\section{Circle actions on 4-dimensional manifolds}



The following result was obtained by Wu-Yi Hsiang and Bruce Kleiner in \cite{hsiang-kleiner}.
The original proof used Alexandrov geometry implicitly;
in fact at the moment 
Alexandrov spaces with lower curvature bound were defined only in dimension $2$.

\begin{thm}{Theorem}\label{thm:keliner}
Let $M$ be a connected complete Riemannian manifold 
and $\mathbb{S}^1\acts M$ be isometric effective circle action.
Then
\begin{subthm}{thm:keliner:nonneg}
If $M$ has nonnegative sectional curvature then the action has at most $4$ isolated fixed points.
\end{subthm}

\begin{subthm}{thm:keliner:positive}
If $M$ has positive sectional curvature then the action has at most $3$ isolated fixed points.
\end{subthm}

\end{thm}

\parit{Proof.}
Consider the quotient space $\spc{L}=M/\mathbb{S}^1$.

Assume $x$ is a fixed point of $\mathbb{S}^1$-action;
denote by $y$ the projection of $x$ in $\spc{L}$.
Note that

\begin{clm}{}\label{clm:Sigma=<sphere/2}
$\Sigma_y\le \tfrac12\cdot\SS^2$;
that is there is a noncontracting map $\Sigma_y\to \tfrac12\cdot\SS^2$.
In particular, 
\[\mangle\hinge y{z^1}{z^2}+\mangle\hinge y{z^2}{z^3}+\mangle\hinge y{z^3}{z^1}\le\pi\]
for any triple of points $z^1,z^2,z^3\in\spc{L}$.
\end{clm}

Indeed, $\Sigma_x\iso\SS^3$.
Therefore $\Sigma_y\iso\SS^3/\SS^1$, 
for an isometric $\SS^1$-action.
Note that we can identify $\SS^3$ as the unit sphere in $\CC^2$
in such a way that the $\SS^1$-action is given by diagonal matrices
$\left(\begin{smallmatrix}
z^p&0\\0&z^q
\end{smallmatrix}\right)$ for some realatively prime integers $p$, $q$
and $z\in\SS^1\subset\CC$.

Consider the torus action $\TT^2\acts\SS^3$ by diagonal matrices
$\left(\begin{smallmatrix}
v&0\\0&w
\end{smallmatrix}\right)$.
Note that $[0,\tfrac\pi2]\iso\SS^3/\TT^2$
and $\Sigma_w$ is isometric to the warped product $\SS^1\times_{f_{p,q}}[0,\tfrac\pi2]$ 
(compare with Exercise~\ref{ex:chohom-1=warped-product}).
Let us compute the warping function $f_{p,q}$. 
If $t\in [0,\tfrac\pi2]$
is the projection of $x\in \SS^3$
then
\begin{align*}
f_{p,q}(t)
&=\frac{\area [\TT^2\cdot x]}{\length[\SS^1\cdot x]}
\\
&=
\frac{\sin t\cdot\cos t}{\sqrt{(p\cdot\sin t)^2+(q\cdot\cos t)^2}}
\end{align*}


Note that 
$f_{1,1}(t)\ge f_{p,q}(t)$
for any pair $(p,q)$ of relatively prime inegers and any $t\in[0,\tfrac\pi2]$ and
\[\tfrac12\cdot\SS^2\iso \SS^1\times_{f_{1,1}}[0,\tfrac\pi2],\]
(In other words $\tfrac12\cdot\SS^2$ is isometric to the orbit space of Hopf fibration on $\SS^3$.)
Applying Exercise \ref{ex:warp=<}, we get \ref{clm:Sigma=<sphere/2}.
\claimqeds

\parit{(\ref{SHORT.thm:keliner:positive}).}
Assume the circle action has 4 isolated points.
Denote by $y_1,y_2,y_3,y_4$ their projections in $\spc{L}$.

According to ???, $\spc{L}$ is a $\CBB{}\kappa$ space for some $\kappa>0$.
According to Claim~{clm:Sigma=<sphere/2},
\[\mangle \hinge{y_i}{y_j}{y_k}\le\tfrac\pi2\]
for any distinct $i$, $j$ and $k$.
In particular non of 4 triangles $\trig{y^1}{y^2}{y^3}$,
$\trig{y^1}{y^2}{y^4}$,
$\trig{y^1}{y^3}{y^4}$,
$\trig{y^2}{y^3}{y^4}$
is degenerate.
Therefore 
The sum of the angles in each triuangle is strictly larger than $\pi$.
Denote by $\omega$ the sum of all 12 angles of these 4 triangles.
We get
\[\omega>4\cdot\pi.\eqlbl{eq:omega>4pi}\]

From Claim~\ref{clm:Sigma=<sphere/2}, 
the sum of 3 angles among these 12 which share one vertex
is at most $\pi$.
Therefore 
\[\omega\le 4\cdot\pi.\]
The later contradicts \ref{eq:omega>4pi}.

\parit{(\ref{SHORT.thm:keliner:nonneg}).}
Assume there are 5 fixed points of the circle action.
Denote by $y_1,y_2,y_3,y_4,y^5$ their projections in $\spc{L}$.

\qeds

The result above was used to classify positively and nonnegatively curved mainifolds which admit isometric circle action.
This classification was improved later by Karsten Grove and Burkhard Wilking in \cite{grove-wilking}.
Namely they show the following.

\begin{thm}{Theorem}
A closed nonnegatively curved complete simply connected 4-dimensional manifold $M$
with an isometric circle action is diffeomorphic to
$\mathbb{S}^4$,
$\CP^2$,
$\mathbb{S}^2\times\mathbb{S}^2$
or one of
$\CP^2\#(\pm\CP^2)$
and the action extends
to a smooth torus
action.

In particular, if $M$ is positively curved then the circle action is equivariantly diffeomorphic to a linear action on 
$\mathbb{S}^4$,
$\RP^2$
or
$\CP^2$.
\end{thm}

\begin{thm}{Proposition}
Let $\spc{L}$ be a $\CBB{}{\kappa}$ space and $E\subset\spc{L}$ be an extremal subset.
Assume $\tilde{\spc{L}}$ is a double metric cover of $\spc{L}$ 
branching around $E$;
that is $\tilde{\spc{L}}$ is the completion of a double metric cover of the complement $\spc{L}\backslash E$.
Then $\tilde{\spc{L}}$ is a $\CBB{}{\kappa}$ space.
\end{thm}
