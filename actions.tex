\part{Curvature bounded below}

\chapter{Isometric actions}

\section{Definitions}
\begin{thm}{Definition}
\label{df:1+3}
A quadruple of points $p,x^1,x^2,x^3$ in a metric space satisfies 
\emph{(1+3)-point $\kappa$-comparison}%
\index{(1+3)-point $\kappa$-comparison} 
if 
\[\angk p{x^1}{x^2}
+\angk p{x^2}{x^3}
+\angk p{x^3}{x^1}\le 2\cdot\pi.\eqlbl{Yup-kappa}\]
or at least one of the model angles $\angk p{x^i}{x^j}$ is not defined.
\end{thm}

\begin{thm}{Definition}
\label{df:cbb1+3}
A space $\spc{L}$ 
is called an Alexandrov space with curvature $\ge \kappa$ 
(briefly, $\spc{L}$ is a $\CBB{}{\kappa}$\index{$\CBB{}{}$} space) 
if any quadruple $p,x^1,x^2,x^3\in \spc{L}$ satisfies (1+3)-point comparison.

We say that $\spc{L}$ an Alexandrov space with curvature bounded below 
(briefly, $\spc{L}$ is a  $\CBB{}{}$ space) if there is $\kappa\in\RR$ such that $\spc{L}$ is a $\CBB{}{\kappa}$ space.
\end{thm}

\parbf{Remarks} 
\begin{itemize}
\item Alexandrov spaces with curvature bounded below usually will be denoted by $\spc{L}$, for $\spc{L}$ower curvature bound.
\item In the definition of spaces with curvature $\ge1$, 
most authors assume in addition that $\diam \spc{L}\le \pi$. 
We do not make this assumption. 
In particular, we consider the real line to have curvature $\ge 1$.
\end{itemize}

\begin{thm}{Exercise}
Let $\spc{L}$ be a length space.
Show that $\spc{L}$ is a $\CBB{}0$ space
if and only if 
\[
\area\modtrig(xyz))_{\EE^2}
\le
\area\modtrig(pxy)_{\EE^2}+\area\modtrig(pyz)_{\EE^2}+\area\modtrig(pzx)_{\EE^2}
\]
for any 4 distinct points $p,x,y,z\in\spc{L}$.
\end{thm}

\begin{thm}{Exercise}\label{mink+alex=euclid} 
Let $\spc{L}$ be  $\RR^m$ with the metric defined by a norm).
Show that $\spc{L}$ is a $\CBB{}0$ space if and only if $\spc{L}\iso\EE^m$.
\end{thm}


\begin{thm}{Theorem}\label{thm:CBB/G}
Let $\spc{L}$ is a $\CBB{}\kappa$ space and the group $G$ acts on $\spc{L}$ by isometries 
and it has closed orbits.
Then the orbit space $\spc{L}/G$ is a $\CBB{}\kappa$ space. 
\end{thm}

\parit{Proof of \ref{thm:submetry-CBB}.}
Fix $\delta>0$ and a quadruple of points $p,x^1,x^2,x^3\in \spc{M}$.
Choose arbitrary $\hat p\in \spc{L}$ such that $\sigma(\hat{p})=p$.
Next $\sigma$ is submetry, we can choose the points $\hat{x}^1,\hat{x}^2,\hat{x}^3\in \spc{M}$.
such that $\sigma(\hat x_i)=x_i$ and
\[\dist{\hat{p}}{\hat{x}^i}{\spc{L}}
\lege
\dist{p}{x^i}{\spc{M}}\pm\delta\]
for all $i$.
Note that 
\[\dist{\hat{x}^i}{\hat{x}^j}{\spc{L}}
>
\dist{x^i}{x^j}{\spc{M}}-2\cdot\delta\]
for all $i$ and $j$.
Therefore given $\eps>0$, the value $\delta$ above can be chousen in such a way that the inequality
\[\angk {\hat{p}}{\hat{x}^1}{\hat{x}^2}
<
\angk p{x^1}{x^2}+\eps
\eqlbl{eq:angles-M-L}\]
holds for all $i$ and $j$.

By (1+3)-point comparison in $\spc{L}$,
we have
\[\angk {\hat{p}}{\hat{x}^1}{\hat{x}^2}
+\angk {\hat{p}}{\hat{x}^2}{\hat{x}^3}
+\angk {\hat{p}}{\hat{x}^3}{\hat{x}^1}
\le 
2\cdot\pi\]
assuming all the angles on the left hand side are defined.
Applying  \ref{eq:angles-M-L}, 
we get 
\[\angk p{x^1}{x^2}
+\angk p{x^2}{x^3}
+\angk p{x^3}{x^1}
< 
2\cdot\pi+3\cdot\eps.\]

Since $\eps>0$ is arbitrary we have 
\[\angk p{x^1}{x^2}
+\angk p{x^2}{x^3}
+\angk p{x^3}{x^1}\le 2\cdot\pi;\]
that is,
the (1+3)-point $\kappa$-comparison holds in $\spc{M}$.
It remains to apply Proposition~\ref{prop:submet-length}.
\qeds

\section{Erd\H{o}s' problem rediscovered}

The following theorem was proved in ???.

\begin{thm}{Theorem}\label{thm:extr-point}
Let $\spc{L}$ is a $\CBB{}0$ space.
Then it has at most $2^m$ one-point extremal sets.
\end{thm}


The proof is a translation of proof of classical problem in combinatoric geometry to Alexandrov's language.

\begin{thm}{Erd\H{o}s' problem}
Let $F$ be a set of points in $\EE^m$ such that any triangle formed by three distinct points in $F$ has no obtuse angles.
Then number of elements in $F$ can not exeed $2^m$.

Moreover, if $|F|=2^m$ then $F$ consists of vertexes of right paralelepiped.
\end{thm}

This problem was posted in \cite{erdos} and solved in \cite{danzer-gruenbaum}.
The question of classifing all $\spc{L}$ with the maximal number one-point extremal sets turns out to be more delicate see \cite{lebedeva}.
Let us describe a class of examples.
Let $\TT^m$ be standart torus 
and $\Gamma$ be a subgroup generated by all coordinte reflections.
Clearly, $\Gamma$ is isomorphic to $(\ZZ/2\cdot\ZZ)^m$ and it has exactly $2^m$ fixed points in $\TT^m$.
Let $\Gamma'\subset\Gamma$ be a subgroup which has the same fixed point set as $\Gamma$ 
and $g$ be a flat $\Gamma'$-ivariant metric on $\TT^m$.
Then, $(\TT^m,g)/\Gamma'$ is a $\CBB{m}0$ space and it has exactly $2^m$ one-point extremal subsets; it follows from ??? and ???.
The above construction does not describes all $\CBB m0$ spaces with $2^m$ one-point extremal subsets, but it is close to the correct answer.

The finding resonable estimates for maximal number of extraml one-point subsets in a $\CBB m 1$ space is completely open;
it analogous to the follolwing problem: finding maximal number $n(m)$ such that there are points $p_1,p_2,\dots p_n$ in $\EE^m$ such that any triangle $\trig{p_i}{p_j}{p_\kay}$ is acute.
It is expected $n(m)\ll 2^m$, but so far it is only known that 
\begin{enumerate}
\item $n(m)\ge 2m$ --- the configuration s slight perturbation of vertexes of $m$-octahedra.
\item $n(m)\le ???$
\end{enumerate}


\parit{Proof of \ref{thm:extr-point}.}
Let $\{p_i\}$, $i\in\{1,2,\dots,N\}$ be the one-point extreaml sets.
For each $p_i$ consider its open Voronoi domain $V_i$; that is, 
\[V_i=\set{x\in \spc{L}}{\dist{p_i}{x}{}<\dist{p_j}{x}{}\ \t{for any}\ j\not=i}.\]
Clearly $V_i\cap V_j=\emptyset$ if $i\not=j$.
Note that $\vol_mV_i>\frac{1}{2^m}\vol_m \spc{L}$.

Indeed, fix $i$ and for given $\alpha\in(0,1)$, consider $\alpha$-homothety $\map_\alpha\:\spc{L}\to \spc{L}$ with center at $p_i$; 
that is, for each point $x\in \spc{L}$ choose a geodesic $[p_ix]$ and set
$\map_\alpha x=\geod_{[p_ix]}(\alpha\dist{p_i}{x}{})$.
From comparison we have that $\vol(\map_\alpha \spc{L})\ge\alpha^m\vol \spc{L}$.
For any $x\in \spc{L}$ and all $\alpha<\tfrac{1}{2}$ we have $\map_\alpha x\in V_i$.
Assume $x'=\map_\alpha x\notin V_i$,
then threre is $p_j$ such that $\dist{p_i}{x'}{}\ge\dist{p_j}{x'}{}$.
Then from comparison, we have $\angk{p_j}{p_i}{x}_{\EE^2}>\tfrac\pi2$;
that is, $p_j$ does not form a one-point extremal set.???
\qeds

\section{Circle actions on 4-dimensional manifolds}



The following result was obtained by Hsiang and Kleiner in \cite{hsiang-kleiner}.
The original proof used Alexandrov geometry implicitly;
in fact at the moment 
Alexandrov spaces with lower curvature bound were defined only in dimension $2$.

\begin{thm}{Theorem}\label{thm:keliner}
Let $M$ be a connected complete Riemannian manifold 
and $\mathbb{S}^1\acts M$ be isometric effective circle action.
Then
\begin{subthm}{thm:keliner:nonneg}
If $M$ has nonnegative sectional curvature then the action has at most $4$ isolated fixed points.
\end{subthm}

\begin{subthm}{thm:keliner:positive}
If $M$ has positive sectional curvature then the action has at most $3$ isolated fixed points.
\end{subthm}

\end{thm}

\parit{Proof.}
Consider the quotient space $\spc{L}=M/\mathbb{S}^1$.

Assume $x$ is a fixed point of $\mathbb{S}^1$-action;
denote by $y$ the projection of $x$ in $\spc{L}$.
Note that

\begin{clm}{}\label{clm:Sigma=<sphere/2}
$\Sigma_y\le \tfrac12\cdot\SS^2$;
that is there is a noncontracting map $\Sigma_y\to \tfrac12\cdot\SS^2$.
In particular, 
\[\mangle\hinge y{z^1}{z^2}+\mangle\hinge y{z^2}{z^3}+\mangle\hinge y{z^3}{z^1}\le\pi\]
for any triple of points $z^1,z^2,z^3\in\spc{L}$.
\end{clm}

Indeed, $\Sigma_x\iso\SS^3$.
Therefore $\Sigma_y\iso\SS^3/\SS^1$, 
for an isometric $\SS^1$-action.
Note that we can identify $\SS^3$ as the unit sphere in $\CC^2$
in such a way that the $\SS^1$-action is given by diagonal matrices
$\left(\begin{smallmatrix}
z^p&0\\0&z^q
\end{smallmatrix}\right)$ for some realatively prime integers $p$, $q$
and $z\in\SS^1\subset\CC$.

Consider the torus action $\TT^2\acts\SS^3$ by diagonal matrices
$\left(\begin{smallmatrix}
v&0\\0&w
\end{smallmatrix}\right)$.
Note that $[0,\tfrac\pi2]\iso\SS^3/\TT^2$
and $\Sigma_w$ is isometric to the warped product $\SS^1\times_{f_{p,q}}[0,\tfrac\pi2]$ 
(compare with Exercise~\ref{ex:chohom-1=warped-product}).
Let us compute the warping function $f_{p,q}$. 
If $t\in [0,\tfrac\pi2]$
is the projection of $x\in \SS^3$
then
\begin{align*}
f_{p,q}(t)
&=\frac{\area [\TT^2\cdot x]}{\length[\SS^1\cdot x]}
\\
&=
\frac{\sin t\cdot\cos t}{\sqrt{(p\cdot\sin t)^2+(q\cdot\cos t)^2}}
\end{align*}


Note that 
$f_{1,1}(t)\ge f_{p,q}(t)$
for any pair $(p,q)$ of relatively prime inegers and any $t\in[0,\tfrac\pi2]$ and
\[\tfrac12\cdot\SS^2\iso \SS^1\times_{f_{1,1}}[0,\tfrac\pi2],\]
(In other words $\tfrac12\cdot\SS^2$ is isometric to the orbit space of Hopf fibration on $\SS^3$.)
Applying Exercise \ref{ex:warp=<}, we get \ref{clm:Sigma=<sphere/2}.
\claimqeds

\parit{(\ref{SHORT.thm:keliner:positive}).}
Assume the circle action has 4 isolated points.
Denote by $y_1,y_2,y_3,y_4$ their projections in $\spc{L}$.

According to ???, $\spc{L}$ is a $\CBB{}\kappa$ space for some $\kappa>0$.
According to Claim~{clm:Sigma=<sphere/2},
\[\mangle \hinge{y_i}{y_j}{y_k}\le\tfrac\pi2\]
for any distinct $i$, $j$ and $k$.
In particular non of 4 triangles $[y^1y^2y^3]$,
$[y^1y^2y^4]$,
$[y^1y^3y^4]$,
$[y^2y^3y^4]$
is degenerate.
Therefore 
The sum of the angles in each triuangle is strictly larger than $\pi$.
Denote by $\omega$ the sum of all 12 angles of these 4 triangles.
We get
\[\omega>4\cdot\pi.\eqlbl{eq:omega>4pi}\]

From Claim~\ref{clm:Sigma=<sphere/2}, 
the sum of 3 angles among these 12 which share one vertex
is at most $\pi$.
Therefore 
\[\omega\le 4\cdot\pi.\]
The later contradicts \ref{eq:omega>4pi}.

\parit{(\ref{SHORT.thm:keliner:nonneg}).}
Assume there are 5 fixed points of the circle action.
Denote by $y_1,y_2,y_3,y_4,y^5$ their projections in $\spc{L}$.

\qeds

The result above was used to classify positively and nonnegatively curved mainifolds which admit isometric circle action.
This classification was improved later by Grove and Wilking in \cite{grove-wilking}.
Namely they show the following.

\begin{thm}{Theorem}
A closed nonnegatively curved complete simply connected 4-dimensional manifold $M$
with an isometric circle action is diffeomorphic to
$\mathbb{S}^4$,
$\CP^2$,
$\mathbb{S}^2\times\mathbb{S}^2$
or one of
$\CP^2\#(\pm\CP^2)$
and the action extends
to a smooth torus
action.

In particular, if $M$ is positively curved then the circle action is equivariantly diffeomorphic to a linear action on 
$\mathbb{S}^4$,
$\RP^2$
or
$\CP^2$.
\end{thm}

\begin{thm}{Proposition}
Let $\spc{L}$ be a $\CBB{}{\kappa}$ space and $E\subset\spc{L}$ be an extremal subset.
Assume $\tilde{\spc{L}}$ is a double metric cover of $\spc{L}$ 
branching around $E$;
that is $\tilde{\spc{L}}$ is the completion of a double metric cover of the complement $\spc{L}\backslash E$.
Then $\tilde{\spc{L}}$ is a $\CBB{}{\kappa}$ space.
\end{thm}
